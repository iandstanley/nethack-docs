% Created 2021-09-14 Tue 15:15
% Intended LaTeX compiler: pdflatex
\documentclass[11pt]{article}
\usepackage[utf8]{inputenc}
\usepackage[T1]{fontenc}
\usepackage{graphicx}
\usepackage{grffile}
\usepackage{longtable}
\usepackage{wrapfig}
\usepackage{rotating}
\usepackage[normalem]{ulem}
\usepackage{amsmath}
\usepackage{textcomp}
\usepackage{amssymb}
\usepackage{capt-of}
\usepackage{hyperref}
\author{Eric S. Raymond}
\date{\today}
\title{A Guide to the Mazes of Menace\\\medskip
\large Nethack Guidebook}
\hypersetup{
 pdfauthor={Eric S. Raymond},
 pdftitle={A Guide to the Mazes of Menace},
 pdfkeywords={},
 pdfsubject={},
 pdfcreator={Emacs 27.1 (Org mode 9.3)}, 
 pdflang={English}}
\begin{document}

\maketitle
\setcounter{tocdepth}{2}
\tableofcontents

\begin{center}
A Guide to the Mazes of Menace
   (Guidebook for NetHack)


	     Original version - Eric S. Raymond
(Edited and expanded for 3.6 by Mike Stephenson and others)

January 27, 2020
\end{center}

NetHack is a single player dungeon exploration game that runs on a
wide variety of computer systems, with a variety of graphical and text
interfaces all using the same game engine. Unlike many other Dungeons
\& Dragons-inspired games, the emphasis in NetHack is on discovering
the detail of the dungeon and not simply killing everything in sight -
in fact, killing everything in sight is a good way to die
quickly. Each game presents a different landscape - the random number
generator provides an essentially unlimited number of variations of
the dungeon and its denizens to be discovered by the player in one of
a number of characters: you can pick your race, your role, and your
gender. 



\section{Introduction}
\label{sec:org0e0a819}

Recently, you have begun to find yourself unfulfilled and distant in
your daily occupation.  Strange dreams of stealing, crusading, and
combat have haunted you in your sleep for many months, but you aren't
sure of the reason.  You wonder whether you have in fact been haviqng
those dreams all your life, and somehow managed to forget about them
until now.  Some nights you awaken suddenly and cry out, terrified at
the vivid recollection of the strange and powerful creatures that seem
to be lurking behind every corner of the dungeon in your dream.  Could
these details haunting your dreams be real?  As each night passes, you
feel the desire to enter the mysterious caverns near the ruins grow
stronger.  Each morning, however, you quickly put the idea out of your
head as you recall the tales of those who entered the caverns before
you and did not return.  Eventually you can resist the yearning to
seek out the fantastic place in your dreams no longer.  After all,
when other adventurers came back this way after spending time in the
caverns, they usually seemed better off than when they passed through
the first time.  And who was to say that all of those who did not
return had not just kept going?

Asking around, you hear about a bauble, called the Amulet of Yendor by
some, which, if you can find it, will bring you great wealth.  One
legend you were told even mentioned that the one who finds the amulet
will be granted immortality by the gods.  The amulet is rumored to be
somewhere beyond the Valley of Gehennom, deep within the Mazes of
Menace.  Upon hearing the legends, you immediately realize that there
is some profound and undiscovered reason that you are to descend into
the caverns and seek out that amulet of which they spoke.  Even if the
rumors of the amulet's powers are untrue, you decide that you should
at least be able to sell the tales of your adventures to the local
minstrels for a tidy sum, especially if you encounter any of the
terrifying and magical creatures of your dreams along the way.  You
spend one last night fortifying yourself at the local inn, becoming
more and more depressed as you watch the odds of your success being
posted on the inn's walls getting lower and lower.

In the morning you awake, collect your belongings, and set off for the
dungeon.  After several days of uneventful travel, you see the ancient
ruins that mark the entrance to the Mazes of Menace.  It is late at
night, so you make camp at the entrance and spend the night sleeping
under the open skies.  In the morning, you gather your gear, eat what
may be your last meal outside, and enter the dungeon\ldots{}


\section{What is going on here?}
\label{sec:org3d08c65}


You have just begun a game of NetHack.  Your goal is to grab as much
treasure as you can, retrieve the Amulet of Yendor, and escape the
Mazes of Menace alive.

Your abilities and strengths for dealing with the hazards of adventure
will vary with your background and training:

\begin{itemize}
\item Archeologists understand dungeons pretty well; this enables them to
move quickly and sneak up on the local nasties.  They start equipped
with the tools for a proper scientific expedition.

\item Barbarians are warriors out of the hinterland, hardened to
battle. They begin their quests with naught but uncommon strength, a
trusty hauberk, and a great two-handed sword.

\item Cavemen and Cavewomen start with exceptional strength but,
unfortunately, with neolithic weapons.

\item Healers are wise in medicine and apothecary.  They know the herbs
and simples that can restore vitality, ease pain, anesthetize, and
neutralize poisons; and with their instruments, they can divine a
being's state of health or sickness.  Their medical practice earns
them quite reasonable amounts of money, with which they enter the
dungeon.

\item Knights are distinguished from the common skirmisher by their
devotion to the ideals of chivalry and by the surpassing excellence
of their armor.

\item Monks are ascetics, who by rigorous practice of physical and mental
disciplines have become capable of fighting as effectively without
weapons as with.  They wear no armor but make up for it with
increased mobility.

\item Priests and Priestesses are clerics militant, crusaders advancing
the cause of righteousness with arms, armor, and arts thaumaturgic.
Their ability to commune with deities via prayer occasionally
extricates them from peril, but can also put them in it.

\item Rangers are most at home in the woods, and some say slightly out of
place in a dungeon.  They are, however, experts in archery as well
as tracking and stealthy movement.

\item Rogues are agile and stealthy thieves, with knowledge of locks,
traps, and poisons.  Their advantage lies in surprise, which they
employ to great advantage.

\item Samurai are the elite warriors of feudal Nippon.  They are lightly
armored and quick, and wear the dai-sho, two swords of the deadliest
keenness.

\item Tourists start out with lots of gold (suitable for shopping with), a
credit card, lots of food, some maps, and an expensive camera.  Most
monsters don't like being photographed.

\item Valkyries are hardy warrior women.  Their upbringing in the harsh
Northlands makes them strong, inures them to extremes of cold, and
instills in them stealth and cunning.

\item Wizards start out with a knowledge of magic, a selection of magical
items, and a particular affinity for dweomercraft.  Although
seemingly weak and easy to overcome at first sight, an experienced
Wizard is a deadly foe.
\end{itemize}

You may also choose the race of your character (within limits; most
roles have restrictions on which races are eligible for them):

\begin{itemize}
\item Dwarves are smaller than humans or elves, but are stocky and solid
individuals.  Dwarves' most notable trait is their great expertise
in mining and metalwork.  Dwarvish armor is said to be second in
quality not even to the mithril armor of the Elves.

\item Elves are agile, quick, and perceptive; very little of what goes on
will escape an Elf.  The quality of Elven craftsmanship often gives
them an advantage in arms and armor.

\item Gnomes are smaller than but generally similar to dwarves.  Gnomes
are known to be expert miners, and it is known that a secret
underground mine complex built by this race exists within the Mazes
of Menace, filled with both riches and danger.

\item Humans are by far the most common race of the surface world, and are
thus the norm to which other races are often compared.  Although
they have no special abilities, they can succeed in any role.

\item Orcs are a cruel and barbaric race that hate every living thing
(including other orcs).  Above all others, Orcs hate Elves with a
passion unequalled, and will go out of their way to kill one at any
opportunity.  The armor and weapons fashioned by the Orcs are
typically of inferior quality.
\end{itemize}


\section{What do all those things on the screen mean?}
\label{sec:org569a36b}

On the screen is kept a map of where you have been and what you have
seen on the current dungeon level; as you explore more of the level,
it appears on the screen in front of you.

When NetHack's ancestor rogue first appeared, its screen orientation
was almost unique among computer fantasy games.  Since then, screen
orientation has become the norm rather than the exception; NetHack
continues this fine tradition.  Unlike text adventure games that
accept commands in pseudo-English sentences and explain the results in
words, NetHack commands are all one or two keystrokes and the results
are displayed graphically on the screen.  A minimum screen size of 24
lines by 80 columns is recommended; if the screen is larger, only a
21x80 section will be used for the map.

NetHack can even be played by blind players, with the assistance of
Braille readers or speech synthesisers.  Instructions for configuring
NetHack for the blind are included later in this document.

NetHack generates a new dungeon every time you play it; even the
authors still find it an entertaining and exciting game despite having
won several times.

NetHack offers a variety of display options.  The options available to
you will vary from port to port, depending on the capabilities of your
hardware and software, and whether various compile-time options were
enabled when your executable was created.  The three possible display
options are: a monochrome character interface, a color character
interface, and a graphical interface using small pictures called
tiles.  The two character interfaces allow fonts with other characters
to be substituted, but the default assignments use standard ASCII
characters to represent everything.  There is no difference between
the various display options with respect to game play.  Because we
cannot reproduce the tiles or colors in the Guidebook, and because it
is common to all ports, we will use the default ASCII characters from
the monochrome character display when referring to things you might
see on the screen during your game.

In order to understand what is going on in NetHack, first you must
understand what NetHack is doing with the screen.  The NetHack screen
replaces the "You see \ldots{}" descriptions of text adventure
games. Figure 1 is a sample of what a NetHack screen might look like.
The way the screen looks for you depends on your platform.  

\begin{verbatim}
+----------------------------------------------------------------+
|The bat bites!                                                  |
|                                                                |
|    ------                                                      |
|    |....|    ----------                                        |
|    |.<..|####...@...$.|                                        |
|    |....-#   |...B....+                                        |
|    |....|    |.d......|                                        |
|    ------    -------|--                                        |
|                                                                |
|                                                                |
|                                                                |
|Player the Rambler    St:12 Dx:7 Co:18 In:11 Wi:9 Ch:15 Neutral |
|Dlvl:1 $:0 HP:9(12) Pw:3(3) AC:10 Exp:1/19 T:257 Weak           |
+---------------------------Figure-1-----------------------------+
\end{verbatim}

\subsection{The message line}
\label{sec:org037bfbb}

The top line of the screen is reserved for messages that describe
things that are impossible to represent visually.  If  you see a
"--More--" on the top line, this means that NetHack has another
message to display on the screen, but it wants to make certain that
you've read the one that is there first.  To read the next message,
just press the space bar.  

To change how and what messages are  shown  on  the  message line, see
"Configuring Message Types" and the verbose option. 

\subsection{The map}
\label{sec:orgb6fd488}

The rest  of the screen is the map of the level as you have explored
it so far.  Each symbol on the screen  represents  something.   You
can  set various graphics options to change some of the symbols the
game uses; otherwise, the game will  use  default symbols.  Here is a
list of what the default symbols mean: 


\begin{description}
\item[{- and |}] The walls of a room, or an open door.  Or a grave (|).

\item[{.}] The floor of a room, ice, or a doorless doorway.

\item[{\#}] A  corridor,  or iron bars, or a tree, or possibly a kitchen
sink (if your dungeon has sinks), or a drawbridge.

\item[{>}] Stairs down: a way to the next level.

\item[{<}] Stairs up: a way to the previous level.

\item[{+}] A closed door, or a spellbook containing a spell you may be
able to learn.

\item[{@}] Your character or a human.

\item[{\$}] A pile of gold.

\item[{\^{}}] A trap (once you have detected it).

\item[{)}] A weapon.

\item[{[}] A suit or piece of armor.

\item[{\%}] Something edible (not necessarily healthy).

\item[{?}] A scroll.

\item[{/}] A wand.

\item[{=}] A ring.

\item[{!}] A potion.

\item[{(}] A useful item (pick-axe, key, lamp\ldots{}).

\item[{"}] An amulet or a spider web.

\item[{*}] A gem or rock (possibly valuable, possibly worthless).

\item[{`}] A boulder or statue.

\item[{0}] An iron ball.

\item[{\_}] An altar, or an iron chain.

\item[{\{}] A fountain.

\item[{\}}] A pool of water or moat or a pool of lava.

\item[{$\backslash$}] An opulent throne.

\item[{a-zA-Z and other symbols}] Letters  and certain other symbols
represent the various inhabitants of the Mazes of Menace.  Watch
out,  they  can be nasty and vicious.  Sometimes, however, they can
be helpful.

\item[{I}] This marks the last known location of an invisible or otherwise
unseen monster.   Note  that  the  monster  could  have moved.  The
`F' and `m' commands may be useful here.
\end{description}


You  need  not  memorize  all these symbols; you can ask the game what
any symbol represents with the  `/'  command  (see  the next section
for more info). 

\subsection{The status lines}
\label{sec:org9697301}
The bottom two lines of the screen contain several cryptic pieces of
information describing your current status.  If either status line
becomes longer than the width of the screen, you might not see all of
it.  Here are explanations of what the various status items mean
(though your configuration may not have all the status items listed
below):

\begin{description}
\item[{Rank}] Your character's name and professional ranking (based on
the experience level, see below)

\item[{Strength}] A measure of your character's strength; one of your six
basic  attributes.   A  human character's attributes can range from
3 to 18 inclusive; non-humans may exceed  these  limits
(occasionally you may get super-strengths of the form 18/xx, and
magic can also cause attributes  to  exceed  the  normal limits).
The  higher  your strength, the stronger you are. Strength  affects
how  successfully  you  perform  physical tasks,  how  much damage
you do in combat, and how much loot you can carry.

\item[{Dexterity}] Dexterity affects your chances to hit in  combat,  to
avoid traps,  and do other tasks requiring agility or manipulation
of objects.

\item[{Constitution}] Constitution affects your ability to recover  from
injuries and  other strains on your stamina.  When strength is low
or modest, constitution also affects how much  you  can  carry. With
sufficiently high strength, the contribution to carrying capacity
from your constitution no longer matters.

\item[{Intelligence}] Intelligence affects your ability to cast  spells
and  read spellbooks.

\item[{Wisdom}] Wisdom comes from your practical experience (especially
when dealing with magic).  It affects your magical energy.

\item[{Charisma}] Charisma affects how certain creatures react toward you.
In particular, it can affect the prices shopkeepers offer you.

\item[{Alignment}] Lawful, Neutral, or Chaotic.  Often, Lawful is taken as
good and Chaotic as evil, but legal and ethical do not always co-
incide.   Your alignment influences how other monsters react toward
you.  Monsters of a like alignment are more likely to be
non-aggressive, while those of an opposing alignment are more likely
to be seriously offended at your presence.

\item[{Dungeon Level}] How deep you are in the dungeon.  You start at
level one and the  number  increases  as  you  go deeper into the
dungeon. Some levels are special, and are identified by  a  name
and not  a  number.  The Amulet of Yendor is reputed to be some-
where beneath the twentieth level.

\item[{Gold}] The number of gold pieces you  are  openly  carrying.   Gold
which you have concealed in containers is not counted.
\end{description}


\begin{description}
\item[{Hit Points}] Your  current  and  maximum hit points.  Hit points
indicate how much damage you can take before you die.  The  more
you get  hit in a fight, the lower they get.  You can regain hit
points by resting, or by  using  certain  magical  items  or spells.
The  number  in  parentheses is the maximum number your hit points
can reach.

\item[{Power}] Spell points.  This tells you how much mystic energy
(mana) you  have  available for spell casting.  Again, resting will
regenerate the amount available.

\item[{Armor Class}] A measure of how effectively your armor stops blows
from unfriendly  creatures.  The lower this number is, the more
effective the armor; it is quite possible to have negative armor
class.

\item[{Experience}] Your current experience level and experience points.
As you adventure, you gain experience points.  At  certain
experience  point  totals, you gain an experience level.  The more
experienced you are, the better you fight and withstand magical
attacks.  Many dungeons show only your experience level here.

\item[{Time}] The number of turns elapsed so far, displayed  if  you  have
the time option set.

\item[{Status }] \begin{itemize}
\item Hunger:  your  current  hunger status.  Values are Satiated, Not
Hungry (or Normal), Hungry,  Weak,  and  Fainting.   Not shown when
Normal.
\item Encumbrance:  an indication of how what you are carrying affects
your ability to move.  Values are Unencumbered, Encum- bered,
Stressed,  Strained, Overtaxed, and Overloaded.  Not shown when
Unencumbered.
\item Fatal conditions: Stone (aka Petrifying, turning to  stone), Slime
(turning into green slime), Strngl (being strangled), FoodPois
(suffering  from  acute  food  poisoning),  TermIll (suffering from
a terminal illness).
\item Non-fatal  conditions: Blind (can't see), Deaf (can't hear), Stun
(stunned), Conf (confused), Hallu (hallucinating).
\item Movement modifiers: Lev  (levitating),  Fly  (flying),  Ride (riding).
\item Other conditions and modifiers exist, but there isn't enough room
to display them with the other status fields.
\item The `\textsuperscript{X}' command shows all relevant status conditions.
\end{itemize}
\end{description}


\section{Commands}
\label{sec:orgec4be61}

Commands can be initiated by typing one or two characters to which the
command is bound to, or typing the command name in  the extended
commands  entry.   Some commands, like "search", do not require that
any more information be collected by NetHack.  Other commands  might
require additional information, for example a direction, or an
object to be used.  For those  commands  that  require  additional
information, NetHack will present you with either a menu of choices
or with a command line  prompt  requesting information.  Which you are
presented with will depend chiefly on how you have set the menustyle
option.  

For example, a common question, in the  form  "What  do  you want  to
use? [a-zA-Z ?*]", asks you to choose an object you are
carrying. Here, "a-zA-Z" are the inventory letters of your  possible
choices. Typing `?' gives you an inventory list of these items, so you
can see what each letter refers to.  In this  exam- ple, there is also
a `*' indicating that you may choose an object not on the list, if you
wanted to use something unexpected.  Typing  a `*' lists your entire
inventory, so you can see the inventory letters of every object you're
carrying.   Finally,  if  you change your mind and decide you don't
want to do this command after all, you can press the ESC key to abort
the command.   

You can put a number before some  commands  to  repeat  them that
many  times;  for example, "10s" will search ten times.  If you have
the number\textsubscript{pad} option set, you must type `n' to prefix a count,  so
the example above would be typed "n10s" instead.  Commands for which
counts make no sense ignore them.   In  addition, movement  commands
can  be prefixed for greater control (see be- low).  To cancel a count
or a prefix, press the ESC key. 

The list of commands is rather long, but it can be  read  at any  time
during the game through the `?' command, which accesses a menu of
helpful texts.  Here are the default key  bindings  for your
reference: 

\begin{description}
\item[{?}] Help menu:  display one of several help texts available.

\item[{/}] The "whatis" command, to tell what a symbol represents.  You
may choose to specify a location or type a symbol (or even a whole
word)  to  explain.  Specifying a location is done by moving the
cursor to a particular spot on the map  and  then pressing one of
`.', `,', `;', or `:'.  `.' will explain the symbol at the chosen
location, conditionally check for "More info?"  depending  upon
whether  the help option is on, and then you will be asked to pick
another  location;  `,'  will explain the symbol but skip any
additional information, then let you pick another location; `;'
will skip additional info and also not bother asking you to choose
another location to examine; `:' will show additional info, if any,
without ask- ing for confirmation.  When picking a location,
pressing the ESC key will terminate this command, or  pressing  `?'
will give a brief reminder about how it works. 

If  the  autodescribe  option  is on, a short description of what
you see at each location is shown as you move the  cursor.   Typing
`\#' while picking a location will toggle that option on or off.
The whatis\textsubscript{coord} option controls  whether the short description
includes map coordinates. 

Specifying  a  name  rather than a location always gives any
additional information available about that name. 

You may also request a description of nearby  monsters,  all
monsters  currently  displayed,  nearby  objects, or all objects.
The whatis\textsubscript{coord} option controls which format of map coordinate is
included with their descriptions.
\end{description}
\begin{description}
\item[{\&}] Tell what a command does.

\item[{<}] Go  up  to  the previous level (if you are on a staircase or ladder).

\item[{>}] Go down to the next level (if you are on a staircase or ladder).

\item[{[yuhjklbn]}] Go  one  step in the direction indicated (see Figure
2).  If you sense or remember a monster there, you  will  fight  the
monster  instead.   Only  these  one-step  movement commands cause
you to fight monsters; the others (below) are "safe."
\end{description}

\begin{verbatim}
y  k  u          7  8  9
 \ | /            \ | /
h- . -l          4- . -6
 / | \            / | \
b  j  n          1  2  3

 (if number_pad is set)

	Figure 2
\end{verbatim}

\begin{description}
\item[{[YUHJKLBN]}] Go in that direction until you hit a wall or run into
something.

\item[{m[yuhjklbn]}] Prefix:   move  without picking up objects or
fighting (even if you remember a monster there).  

A few non-movement commands use the `m'  prefix to request operating
via menu (to temporarily override the menustyle:Traditional option).
Primarily useful for `,' (pickup) when there is only one class of
objects present (where there won't be any "what kinds of objects?"
prompt, so no opportunity to answer `m' at that prompt). 

A few other commands (eat food, offer sacrifice, apply tinning-kit)
use the `m' prefix to skip checking for applicable objects on the
floor and go straight to checking inventory, or (for "\#loot" to
remove a saddle), skip containers and go straight to adjacent
monsters. The prefix will make "\#trav- el" command show a menu of
interesting targets in sight. In debug mode (aka "wizard mode"), the
`m' prefix may also be used with the "\#teleport" and "\#wizlevelport"
commands.

\item[{F[yuhjklbn]}] Prefix: fight a monster (even if you only guess one
is there).

\item[{M[yuhjklbn]}] Prefix: move far, no pickup.

\item[{g[yuhjklbn]}] Prefix: move until something interesting is found.

\item[{G[yuhjklbn] or <CONTROL->[yuhjklbn]}] Prefix:  same as `g', but
forking of corridors is not con- sidered interesting.

\item[{\_}] Travel to a map location via a shortest-path algorithm.

The shortest path is computed over map locations the hero knows
about (e.g. seen or previously traversed). If there is no known
path, a guess is made instead. Stops on most of the same conditions
as the `G' command, but without picking up objects, similar to the
`M' command.  For ports with mouse support, the command is also
invoked when a mouse-click takes place on a location other than the
current position.

\item[{.}] Wait or rest, do nothing for one turn.

\item[{a}] Apply (use) a tool (pick-axe, key, lamp\ldots{}). If used on a wand,
that wand will be broken, releasing its magic in the
process. Confirmation is required.

\item[{A}] Remove one or more worn items, such as armor. Use `T' (take
off) to take off only one piece of armor or `R' (remove) to take off
only one accessory.

\item[{\^{}A}] Redo the previous command.

\item[{c}] Close a door.

\item[{C}] Call (name) a monster, an individual object, or a type of
object. Same as extended command "\#name".

\item[{\^{}C}] Panic button. Quit the game.

\item[{d}] Drop something. For example "d7a" means drop seven items of
object a.

\item[{D}] Drop several things. In answer to the question

"What kinds of things do you want to drop? [!\%= BUCXaium]"

you should type zero or more object symbols possibly followed by `a'
and/or `i' and/or `u' and/or `m'. In addition, one or more of the
blessed/uncursed/cursed groups may be typed:  

\begin{itemize}
\item DB - drop all objects known to be blessed.
\item DU - drop all objects known to be uncursed.
\item DC - drop all objects known to be cursed.
\item DX - drop all objects of unknown B/U/C status.
\item Da - drop all objects, without asking for confirmation.
\item Di - examine your inventory before dropping anything.
\item Du - drop only unpaid objects (when in a shop).
\item Dm - use a menu to pick which object(s) to drop.
\item D\%u - drop only unpaid food.
\end{itemize}

The last example shows a combination. There are three categories of
object filtering: class (`!' for potions, `?' for scrolls, and so
on), shop status (`u' for unpaid, in other words, owned by the
shop), and bless/curse state (`B', `U', `C', and `X' as shown
above). If you specify more than one value in a category (such as
"!?" for potions and scrolls or "BU" for blessed and uncursed), an
inventory object will meet the criteria if it matches any of the
specified values (so "!?" means `!' or `?'). If you specify more
than one category, an inventory object must meet each of the
category criteria (so "\%u" means class `\%' and unpaid `u'). Lastly,
you may specify multiple values within multiple categories: "!?BU"
will select all potions and scrolls which are known to be blessed or
uncursed. (In versions prior to 3.6, filter combinations behaved
differently.)

\item[{\^{}D}] Kick something (usually a door).

\item[{e}] Eat food.

Normally checks for edible item(s) on the floor, then if none are
found or none are chosen, checks for edible item(s) in
inventory. Precede `e' with the `m' prefix to bypass attempting to
eat anything off the floor. 

If you attempt to eat while already satiated, you might choke to
death. If you risk it, you will be asked whether to "continue
eating?" if you survive the first bite.  You can set the
paranoid\textsubscript{confirmation}:eating option to require a response of yes
instead of just y.

\item[{E}] Engrave a message on the floor.

\item[{E-}] Write in the dust with your fingers.

Engraving the word "Elbereth" will cause most monsters to not attack
you hand-to-hand (but if you attack, you will rub it out); this is
often useful to give yourself a breather.

\item[{f}] Fire (shoot or throw) one of the objects placed in your quiver
(or quiver sack, or that you have at the ready). You may select
ammunition with a previous `Q' command, or let the computer pick
something appropriate if autoquiver is true. 

See also `t' (throw) for more general throwing and shooting.

\item[{i}] List your inventory (everything you're carrying).

\item[{I}] List selected parts of your inventory, usually be specifying
the character for a particular set of objects, like `[' for armor or
`!' for potions. 

\begin{itemize}
\item I* - list all gems in inventory;
\item Iu - list all unpaid items;
\item Ix - list all used up items that are on your shopping bill;
\item IB - list all items known to be blessed;
\item IU - list all items known to be uncursed;
\item IC - list all items known to be cursed;
\item IX - list all items whose bless/curse status is unknown;
\item I\$ - count your money.
\end{itemize}

\item[{o}] Open a door.

\item[{O}] Set options.

A menu showing the current option values will be displayed. You can
change most values simply by selecting the menu entry for the given
option (ie, by typing its letter or clicking upon it, depending on
your user interface). For the non-boolean choices, a further menu or
prompt will appear once you've closed this menu. The available
options are listed later in this Guidebook. Options are usually set
before the game rather than with the `O' command; see the section on
options below.

\item[{\^{}O}] Show overview.

Shortcut for "\#overview": list interesting dungeon levels
visited. (Prior to 3.6.0, `\textsuperscript{O}' was a debug mode command which listed
the placement of all special levels. Use "\#wizwhere" to run that
command.)

\item[{p}] Pay your shopping bill.

\item[{P}] Put on an accessory (ring, amulet, or blindfold).

This command may also be used to wear armor. The prompt for which
inventory item to use will only list accessories, but choosing an
unlisted item of armor will attempt to wear it. (See the `W' command
below. It lists armor as the inventory choices but will accept an
accessory and attempt to put that on.)

\item[{\^{}P}] Repeat previous message.

Subsequent `\textsuperscript{P}'s repeat earlier messages. For some interfaces, the
behavior can be varied via the msg\textsubscript{window} option.

\item[{q}] Quaff (drink) something (potion, water, etc).

\item[{Q}] Select an object for your quiver, quiver sack, or just
generally at the ready (only one of these is available at a time).
You can then throw this (or one of these) using the `f' command.

\item[{r}] Read a scroll or spellbook.

\item[{R}] Remove a worn accessory (ring, amulet, or blindfold).

If you're wearing more than one, you'll be prompted for which one to
 remove. When you're only wearing one, then by default it will be
 removed without asking, but you can set the paranoid\textsubscript{confirmation}
 option to require a prompt. 

This command may also be used to take off armor. The prompt for
which inventory item to remove only lists worn accessories, but an
item of worn armor can be chosen.  (See the `T' command below. It
lists armor as the inventory choices but will accept an accessory
and attempt to remove it.)

\item[{\^{}R}] Redraw the screen.

\item[{s}] Search for secret doors and traps around you.  It usually takes
several tries to find something. 

Can also be used to figure out whether there is still a monster at an
 adjacent "remembered, unseen monster" marker.

\item[{S}] Save the game (which suspends play and exits the program). The
saved game will be restored automatically the next time you play
using the same character name. 

In normal play, once a saved game is restored the file used to hold
 the saved data is deleted. In explore mode, once restoration is
 accomplished you are asked whether to keep or delete the
 file. Keeping the file makes it feasible to play for a while then
 quit without saving and later restore again.  

There is no "save current game state and keep playing" command, not
 even in explore mode where saved game files can be kept and
 re-used.

\item[{t}] Throw an object or shoot a projectile.

There's no separate "shoot" command. If you throw an arrow while
 wielding a bow, you are shooting that arrow and any weapon skill
 bonus or penalty for bow applies. If you throw an arrow while not
 wielding a bow, you are throwing it by hand and it will generally be
 less effective than when shot. 

See also `f' (fire) for throwing or shooting an item preselected via
 the `Q' (quiver) command.

\item[{T}] Take off armor.

If you're wearing more than one piece, you'll be prompted for which
 one to take off. (Note that this treats a cloak covering a suit
 and/or a shirt, or a suit covering a shirt, as if the underlying
 items weren't there.) When you're only wearing one, then by default
 it will be taken off without asking, but you can set the
 paranoid\textsubscript{confirmation} option to require a prompt. 

This command may also be used to remove accessories.  The prompt for
 which inventory item to take off only lists worn armor, but a worn
 accessory can be chosen.  (See the `R' command above. It lists
 accessories as the inventory choices but will accept an item of
 armor and attempt to take it off.)

\item[{\^{}T}] Teleport, if you have the ability.

\item[{v}] Display version number.

\item[{V}] Display the game history.

\item[{w}] Wield weapon.

\item[{w-}] wield nothing, use your bare (or gloved) hands.

Some characters can wield two weapons at once; use the `X' command
 (or the "\#twoweapon" extended command) to do so.

\item[{W}] Wear armor.

This command may also be used to put on an accessory (ring,
 amulet, or blindfold). The prompt for which inventory item to use
 will only list armor, but choosing an unlisted accessory will
 attempt to put it on. (See the `P' command above. It lists
 accessories as the inventory choices but will accept an item of
 armor and attempt to wear it.)

\item[{x}] Exchange your wielded weapon with the item in your alternate
weapon slot. 

The latter is used as your secondary weapon when engaging in
 two-weapon combat. Note that if one of these slots is empty, the
 exchange still takes place.

\item[{X}] Toggle two-weapon combat, if your character can do it. Also
available via the "\#twoweapon" extended command. 

(In versions prior to 3.6 this was the command to switch from normal
 play to "explore mode", also known as "discovery mode", which has
 now been moved to "\#exploremode".)

\item[{\^{}X}] Display basic information about your character.

Displays name, role, race, gender (unless role name makes that
 redundant, such as Caveman or Priestess), and alignment, along with
 your patron deity and his or her opposition. It also shows most of
 the various items of information from the status line(s) in a less
 terse form, including several additional things which don't appear
 in the normal status display due to space considerations. 

In normal play, that's all that `\textsuperscript{X}' displays. In explore mode, the
 role and status feedback is augmented by the information provided by
 enlightenment magic.

\item[{z}] Zap a wand.

\item[{z.}] to aim at yourself, use `.' for the direction.

\item[{Z}] Zap (cast) a spell.

\item[{Z.}] to cast at yourself, use `.' for the direction.

\item[{\^{}Z}] Suspend the game (UNIX\footnote{(R)UNIX is a registered trademark of The Open Group} versions with job control only).

\item[{:}] Look at what is here.

\item[{;}] Show what type of thing a visible symbol corresponds to.

\item[{,}] Pick up some things from the floor beneath you. May be preceded
by `m' to force a selection menu.

\item[{@}] Toggle the autopickup option on and off.

\item[{\^{}}] Ask for the type of an adjacent trap you found earlier.

\item[{)}] Tell what weapon you are wielding.

\item[{[}] Tell what armor you are wearing.

\item[{=}] Tell what rings you are wearing.

\item[{"}] Tell what amulet you are wearing.

\item[{(}] Tell what tools you are using.

\item[{*}] Tell what equipment you are using. Combines the preceding five
type-specific commands into one.

\item[{\$}] Count your gold pieces.

\item[{+}] List the spells you know.

Using this command, you can also rearrange the order in which
 your spells are listed, either by sorting the entire list or by
 picking one spell from the menu then picking another to swap places
 with it.  Swapping pairs of spells changes their casting letters, so
 the change lasts after the current `+' command finishes. Sorting the
 whole list is temporary.  To make the most recent sort order persist
 beyond the current `+' command, choose the sort option again and
 then pick "reassign casting letters".  (Any spells learned after
 that will be added to the end of the list rather than be inserted
 into the sorted ordering.)

\item[{$\backslash$}] Show what types of objects have been discovered.

\item[{`}] Show discovered types for one class of objects.

\item[{!}] Escape to a shell.

\item[{\#}] Perform an extended command.
\end{description}



   As you can see, the authors of NetHack used up all the let-
ters, so this is a way to introduce the less frequently used com-
mands. What extended commands are available depends on what fea-
tures the game was compiled with.

\#adjust
   Adjust inventory letters (most useful when the fixinv option
   is "on"). Autocompletes. Default key is `M-a'.

This command allows you to move an item from one particular
inventory slot to another so that it has a letter which is
more meaningful for you or that it will appear in a particu-
lar location when inventory listings are displayed. You can
move to a currently empty slot, or if the destination is oc-
cupied -- and won't merge -- the item there will swap slots
with the one being moved. "\#adjust" can also be used to
split a stack of objects; when choosing the item to adjust,
enter a count prior to its letter.

Adjusting without a count used to collect all compatible
stacks when moving to the destination.  That behavior has
been changed; to gather compatible stacks, "\#adjust" a stack
into its own inventory slot. If it has a name assigned,
other stacks with the same name or with no name will merge
provided that all their other attributes match. If it does
not have a name, only other stacks with no name are eligi-
ble. In either case, otherwise compatible stacks with a
different name will not be merged. This contrasts with us-
ing "\#adjust" to move from one slot to a different slot. In
that situation, moving (no count given) a compatible stack
will merge if either stack has a name when the other doesn't
and give that name to the result, while splitting (count
given) will ignore the source stack's name when deciding
whether to merge with the destination stack.

\#annotate
   Allows you to specify one line of text to associate with the
   current dungeon level. All levels with annotations are dis-
   played by the "\#overview" command. Autocompletes. Default
   key is `M-A', and also `\textsuperscript{N}' if number\textsubscript{pad} is on.

\#apply
   Apply (use) a tool such as a pick-axe, a key, or a lamp.
   Default key is `a'.

If the tool used acts on items on the floor, using the `m'
prefix skips those items.



NetHack 3.6                   January 27, 2020





NetHack Guidebook                       19



If used on a wand, that wand will be broken, releasing its
magic in the process. Confirmation is required.

\#attributes
   Show your attributes. Default key is `\textsuperscript{X}'.

\#autopickup
   Toggle the autopickup option on/off. Default key is `@'.

\#call
   Call (name) a monster, or an object in inventory, on the
   floor, or in the discoveries list, or add an annotation for
   the current level (same as "\#annotate"). Default key is
   `C'.

\#cast
   Cast a spell. Default key is `Z'.

\#chat
   Talk to someone. Default key is `M-c'.

\#close
   Close a door. Default key is `c'.

\#conduct
   List voluntary challenges you have maintained.  Autocom-
   pletes. Default key is `M-C'.

See the section below entitled "Conduct" for details.

\#dip
   Dip an object into something. Autocompletes. Default key
   is `M-d'.

\#down
   Go down a staircase. Default key is `>'.

\#drop
   Drop an item. Default key is `d'.

\#droptype
   Drop specific item types. Default key is `D'.

\#eat
   Eat something. Default key is `e'. The `m' prefix skips
   eating items on the floor.

\#engrave
   Engrave writing on the floor. Default key is `E'.

\#enhance
   Advance or check weapon and spell skills. Autocompletes.
   Default key is `M-e'.



NetHack 3.6                   January 27, 2020





NetHack Guidebook                       20



\#exploremode
   Enter the explore mode.

Requires confirmation; default response is n (no). To real-
ly switch to explore mode, respond with y. You can set the
paranoid\textsubscript{confirmation}:quit option to require a response of
yes instead.

\#fire
   Fire ammunition from quiver. Default key is `f'.

\#force
   Force a lock. Autocompletes. Default key is `M-f'.

\#glance
   Show what type of thing a map symbol corresponds to. De-
   fault key is `;'.

\#help
   Show the help menu. Default key is `?', and also `h' if
   number\textsubscript{pad} is on.

\#herecmdmenu
   Show a menu of possible actions in your current location.

\#history
   Show long version and game history. Default key is `V'.

\#inventory
   Show your inventory. Default key is `i'.

\#inventtype
   Inventory specific item types. Default key is `I'.

\#invoke
   Invoke an object's special powers. Autocompletes. Default
   key is `M-i'.

\#jump
   Jump to another location. Autocompletes.  Default key is
   `M-j', and also `j' if number\textsubscript{pad} is on.

\#kick
   Kick something. Default key is `\textsuperscript{D}', and `k' if number\textsubscript{pad}
   is on.

\#known
   Show what object types have been discovered. Default key is
   `$\backslash$'.

\#knownclass
   Show discovered types for one class of objects. Default key
   is ``'.



NetHack 3.6                   January 27, 2020





NetHack Guidebook                       21



\#levelchange
   Change your experience level.  Autocompletes.  Debug mode
   only.

\#lightsources
   Show mobile light sources. Autocompletes. Debug mode only.

\#look
   Look at what is here, under you. Default key is `:'.

\#loot
   Loot a box or bag on the floor beneath you, or the saddle
   from a steed standing next to you. Autocompletes.  Precede
   with the `m' prefix to skip containers at your location and
   go directly to removing a saddle. Default key is `M-l', and
   also `l' if number\textsubscript{pad} is on.

\#monster
   Use a monster's special ability (when polymorphed into mon-
   ster form). Autocompletes. Default key is `M-m'.

\#name
   Name a monster, an individual object, or a type of object.
   Same as "\#call". Autocompletes. Default keys are `N', `M-
   n', and `M-N'.

\#offer
   Offer a sacrifice to the gods. Autocompletes. Default key
   is `M-o'.

You'll need to find an altar to have any chance at success.
Corpses of recently killed monsters are the fodder of
choice.

The `m' prefix skips offering any items which are on the al-
tar.

\#open
   Open a door. Default key is `o'.

\#options
   Show and change option settings. Default key is `O'.

\#overview
   Display information you've discovered about the dungeon.
   Any visited level (unless forgotten due to amnesia) with an
   annotation is included, and many things (altars, thrones,
   fountains, and so on; extra stairs leading to another dun-
   geon branch) trigger an automatic annotation.  If dungeon
   overview is chosen during end-of-game disclosure, every vis-
   ited level will be included regardless of annotations.  Au-
   tocompletes. Default keys are `\textsuperscript{O}', and `M-O'.




NetHack 3.6                   January 27, 2020





NetHack Guidebook                       22



\#panic
   Test the panic routine. Terminates the current game. Auto-
   completes. Debug mode only.

Asks for confirmation; default is n (no); continue playing.
To really panic, respond with y. You can set the para-
noid\textsubscript{confirmation}:quit option to require a response of yes
instead.

\#pay
   Pay your shopping bill. Default key is `p'.

\#pickup
   Pick up things at the current location. Default key is `,'.
   The `m' prefix forces use of a menu.

\#polyself
   Polymorph self. Autocompletes. Debug mode only.

\#pray
   Pray to the gods for help. Autocompletes. Default key is
   `M-p'.

Praying too soon after receiving prior help is a bad idea.
(Hint: entering the dungeon alive is treated as having re-
ceived help. You probably shouldn't start off a new game by
praying right away.) Since using this command by accident
can cause trouble, there is an option to make you confirm
your intent before praying. It is enabled by default, and
you can reset the paranoid\textsubscript{confirmation} option to disable
it.

\#prevmsg
   Show previously displayed game messages.  Default key is
   `\textsuperscript{P}'.

\#puton
   Put on an accessory (ring, amulet, etc). Default key is
   `P'.

\#quaff
   Quaff (drink) something. Default key is `q'.

\#quit
   Quit the program without saving your game.  Autocompletes.
   Default key is `M-q'.

Since using this command by accident would throw away the
current game, you are asked to confirm your intent before
quitting. Default response is n (no); continue playing. To
really quit, respond with y. You can set the paranoid\textsubscript{con}-
firmation:quit option to require a response of yes instead.




NetHack 3.6                   January 27, 2020





NetHack Guidebook                       23



\#quiver
   Select ammunition for quiver. Default key is `Q'.

\#read
   Read a scroll, a spellbook, or something else. Default key
   is `r'.

\#redraw
   Redraw the screen. Default key is `\textsuperscript{R}', and also `\textsuperscript{L}' if
   number\textsubscript{pad} is on.

\#remove
   Remove an accessory (ring, amulet, etc). Default key is
   `R'.

\#ride
   Ride (or stop riding) a saddled creature.  Autocompletes.
   Default key is `M-R'.

\#rub
   Rub a lamp or a stone. Autocompletes. Default key is `M-
   r'.

\#save
   Save the game and exit the program. Default key is `S'.

\#search
   Search for traps and secret doors around you.  Default key
   is `s'.

\#seeall
   Show all equipment in use. Default key is `*'.

\#seeamulet
   Show the amulet currently worn. Default key is `"'.

\#seearmor
   Show the armor currently worn. Default key is `['.

\#seegold
   Count your gold. Default key is `\$'.

\#seenv
   Show seen vectors. Autocompletes. Debug mode only.

\#seerings
   Show the ring(s) currently worn. Default key is `='.

\#seespells
   List and reorder known spells. Default key is `+'.

\#seetools
   Show the tools currently in use. Default key is `('.



NetHack 3.6                   January 27, 2020





NetHack Guidebook                       24



\#seetrap
   Show the type of an adjacent trap. Default key is `\^{}'.

\#seeweapon
   Show the weapon currently wielded. Default key is `)'.

\#shell
   Do a shell escape. Default key is `!'.

\#sit
   Sit down. Autocompletes. Default key is `M-s'.

\#stats
   Show memory usage statistics. Autocompletes. Debug mode
   only.

\#suspend
   Suspend the game. Default key is `\textsuperscript{Z}'.

\#swap
   Swap wielded and secondary weapons. Default key is `x'.

\#takeoff
   Take off one piece of armor. Default key is `T'.

\#takeoffall
   Remove all armor. Default key is `A'.

\#teleport
   Teleport around the level. Default key is `\textsuperscript{T}'.

\#terrain
   Show bare map without displaying monsters, objects, or
   traps. Autocompletes.

\#therecmdmenu
   Show a menu of possible actions in a location next to you.

\#throw
   Throw something. Default key is `t'.

\#timeout
   Look at the timeout queue. Autocompletes. Debug mode only.

\#tip
   Tip over a container (bag or box) to pour out its contents.
   Autocompletes. Default key is `M-T'. The `m' prefix makes
   the command use a menu.

\#travel
   Travel to a specific location on the map. Default key is
   `\_'. Using the "request menu" prefix shows a menu of inter-
   esting targets in sight without asking to move the cursor.
   When picking a target with cursor and the autodescribe


NetHack 3.6                   January 27, 2020





NetHack Guidebook                       25



option is on, the top line will show "(no travel path)" if
your character does not know of a path to that location.

\#turn
   Turn undead away. Autocompletes. Default key is `M-t'.

\#twoweapon
   Toggle two-weapon combat on or off. Autocompletes. Default
   key is `X', and also `M-2' if number\textsubscript{pad} is off.

Note that you must use suitable weapons for this type of
combat, or it will be automatically turned off.

\#untrap
   Untrap something (trap, door, or chest). Default key is `M-
   u', and `u' if number\textsubscript{pad} is on.

In some circumstances it can also be used to rescue trapped
monsters.

\#up
   Go up a staircase. Default key is `<'.

\#vanquished
   List vanquished monsters. Autocompletes. Debug mode only.

\#version
   Print compile time options for this version of NetHack. Au-
   tocompletes. Default key is `M-v'.

\#versionshort
   Show version string. Default key is `v'.

\#vision
   Show vision array. Autocompletes. Debug mode only.

\#wait
   Rest one move while doing nothing. Default key is `.', and
   also ` ' if rest\textsubscript{on}\textsubscript{space} is on.

\#wear
   Wear a piece of armor. Default key is `W'.

\#whatdoes
   Tell what a key does. Default key is `\&'.

\#whatis
   Show what type of thing a symbol corresponds to.  Default
   key is `/'.

\#wield
   Wield a weapon. Default key is `w'.




NetHack 3.6                   January 27, 2020





NetHack Guidebook                       26



\#wipe
   Wipe off your face. Autocompletes. Default key is `M-w'.

\#wizbury
   Bury objects under and around you. Autocompletes. Debug
   mode only.

\#wizdetect
   Search for hidden things (secret doors or traps or unseen
   monsters) within a modest radius. Autocompletes. Debug
   mode only. Default key is `\textsuperscript{E}'.

\#wizgenesis
   Create a monster. May be prefixed by a count to create more
   than one. Autocompletes. Debug mode only. Default key is
   `\textsuperscript{G}'.

\#wizidentify
   Identify all items in inventory. Autocompletes. Debug mode
   only. Default key is `\textsuperscript{I}'.

\#wizintrinsic
   Set one or more intrinsic attributes. Autocompletes. Debug
   mode only.

\#wizlevelport
   Teleport to another level. Autocompletes. Debug mode only.
   Default key is `\textsuperscript{V}'.

\#wizmap
   Map the level.  Autocompletes. Debug mode only. Default
   key is `\textsuperscript{F}'.

\#wizrumorcheck
   Verify rumor boundaries. Autocompletes. Debug mode only.

\#wizsmell
   Smell monster. Autocompletes. Debug mode only.

\#wizwhere
   Show locations of special levels.  Autocompletes.  Debug
   mode only.

\#wizwish
   Wish for something. Autocompletes. Debug mode only. De-
   fault key is `\textsuperscript{W}'.

\#wmode
   Show wall modes. Autocompletes. Debug mode only.

\#zap
   Zap a wand. Default key is `z'.




NetHack 3.6                   January 27, 2020





NetHack Guidebook                       27



\#?
   Help menu: get the list of available extended commands.



   If your keyboard has a meta key (which, when pressed in com-
bination with another key, modifies it by setting the "meta"
[8th, or "high"] bit), you can invoke many extended commands by
meta-ing the first letter of the command.

   In NT, OS/2, PC and ST NetHack, the "Alt" key can be used in
this fashion; on the Amiga, set the altmeta option to get this
behavior.  On other systems, if typing "Alt" plus another key
transmits a two character sequence consisting of an Escape fol-
lowed by the other key, you may set the altmeta option to have
NetHack combine them into meta+key.

M-? \#? (not supported by all platforms)

M-2 \#twoweapon (unless the number\textsubscript{pad} option is enabled)

M-a \#adjust

M-A \#annotate

M-c \#chat

M-C \#conduct

M-d \#dip

M-e \#enhance

M-f \#force

M-i \#invoke

M-j \#jump

M-l \#loot

M-m \#monster

M-n \#name

M-o \#offer

M-O \#overview

M-p \#pray

M-q \#quit




NetHack 3.6                   January 27, 2020





NetHack Guidebook                       28



M-r \#rub

M-R \#ride

M-s \#sit

M-t \#turn

M-T \#tip

M-u \#untrap

M-v \#version

M-w \#wipe



   If the number\textsubscript{pad} option is on, some additional letter com-
mands are available:

h  \#help

j  \#jump

k  \#kick

l  \#loot

N  \#name

u  \#untrap

\section{5. Rooms and corridors}
\label{sec:org9173b22}

   Rooms and corridors in the dungeon are either lit or dark.
Any lit areas within your line of sight will be displayed; dark
areas are only displayed if they are within one space of you.
Walls and corridors remain on the map as you explore them.

   Secret corridors are hidden. You can find them with the `s'
(search) command.


\section{5.1. Doorways}
\label{sec:orgbeb0541}

   Doorways connect rooms and corridors. Some doorways have no
doors; you can walk right through. Others have doors in them,
which may be open, closed, or locked. To open a closed door, use
the `o' (open) command; to close it again, use the `c' (close)
command.

   You can get through a locked door by using a tool to pick
the lock with the `a' (apply) command, or by kicking it open with
the `\textsuperscript{D}' (kick) command.


NetHack 3.6                   January 27, 2020





NetHack Guidebook                       29



   Open doors cannot be entered diagonally; you must approach
them straight on, horizontally or vertically.  Doorways without
doors are not restricted in this fashion.

   Doors can be useful for shutting out monsters. Most mon-
sters cannot open doors, although a few don't need to (for exam-
ple, ghosts can walk through doors).

   Secret doors are hidden.  You can find them with the `s'
(search) command. Once found they are in all ways equivalent to
normal doors.

5.2. Traps (`\^{}')

   There are traps throughout the dungeon to snare the unwary
delver. For example, you may suddenly fall into a pit and be
stuck for a few turns trying to climb out. Traps don't appear on
your map until you see one triggered by moving onto it, see some-
thing fall into it, or you discover it with the `s' (search) com-
mand. Monsters can fall prey to traps, too, which can be a very
useful defensive strategy.

   There is a special pre-mapped branch of the dungeon based on
the classic computer game "Sokoban." The goal is to push the
boulders into the pits or holes. With careful foresight, it is
possible to complete all of the levels according to the tradi-
tional rules of Sokoban. Some allowances are permitted in case
the player gets stuck; however, they will lower your luck.

5.3. Stairs and ladders (`<', `>')

   In general, each level in the dungeon will have a staircase
going up (`<') to the previous level and another going down (`>')
to the next level. There are some exceptions though.  For in-
stance, fairly early in the dungeon you will find a level with
two down staircases, one continuing into the dungeon and the oth-
er branching into an area known as the Gnomish Mines. Those
mines eventually hit a dead end, so after exploring them (if you
choose to do so), you'll need to climb back up to the main dun-
geon.

   When you traverse a set of stairs, or trigger a trap which
sends you to another level, the level you're leaving will be de-
activated and stored in a file on disk. If you're moving to a
previously visited level, it will be loaded from its file on disk
and reactivated. If you're moving to a level which has not yet
been visited, it will be created (from scratch for most random
levels, from a template for some "special" levels, or loaded from
the remains of an earlier game for a "bones" level as briefly de-
scribed below). Monsters are only active on the current level;
those on other levels are essentially placed into stasis.

   Ordinarily when you climb a set of stairs, you will arrive
on the corresponding staircase at your destination.  However,


NetHack 3.6                   January 27, 2020





NetHack Guidebook                       30



pets (see below) and some other monsters will follow along if
they're close enough when you travel up or down stairs, and occa-
sionally one of these creatures will displace you during the
climb. When that occurs, the pet or other monster will arrive on
the staircase and you will end up nearby.

   Ladders serve the same purpose as staircases, and the two
types of inter-level connections are nearly indistinguishable
during game play.

5.4. Shops and shopping

   Occasionally you will run across a room with a shopkeeper
near the door and many items lying on the floor.  You can buy
items by picking them up and then using the `p' command. You can
inquire about the price of an item prior to picking it up by us-
ing the "\#chat" command while standing on it. Using an item pri-
or to paying for it will incur a charge, and the shopkeeper won't
allow you to leave the shop until you have paid any debt you owe.

   You can sell items to a shopkeeper by dropping them to the
floor while inside a shop. You will either be offered an amount
of gold and asked whether you're willing to sell, or you'll be
told that the shopkeeper isn't interested (generally, your item
needs to be compatible with the type of merchandise carried by
the shop).

   If you drop something in a shop by accident, the shopkeeper
will usually claim ownership without offering any compensation.
You'll have to buy it back if you want to reclaim it.

   Shopkeepers sometimes run out of money. When that happens,
you'll be offered credit instead of gold when you try to sell
something. Credit can be used to pay for purchases, but it is
only good in the shop where it was obtained; other shopkeepers
won't honor it. (If you happen to find a "credit card" in the
dungeon, don't bother trying to use it in shops; shopkeepers will
not accept it.)

   The `\$' command, which reports the amount of gold you are
carrying (in inventory, not inside bags or boxes), will also show
current shop debt or credit, if any. The "Iu" command lists un-
paid items (those which still belong to the shop) if you are car-
rying any. The "Ix" command shows an inventory-like display of
any unpaid items which have been used up, along with other shop
fees, if any.

5.4.1. Shop idiosyncrasies

Several aspects of shop behavior might be unexpected.

\begin{itemize}
\item The price of a given item can vary due to a variety of factors.
\end{itemize}




NetHack 3.6                   January 27, 2020





NetHack Guidebook                       31



\begin{itemize}
\item A shopkeeper treats the spot immediately inside the door as if
it were outside the shop.

\item While the shopkeeper watches you like a hawk, he will generally
ignore any other customers.

\item If a shop is "closed for inventory," it will not open of its
own accord.

\item Shops do not get restocked with new items, regardless of inven-
tory depletion.

\item Monsters

Monsters you cannot see are not displayed on the screen.
\end{itemize}
Beware!  You may suddenly come upon one in a dark place. Some
magic items can help you locate them before they locate you
(which some monsters can do very well).

   The commands `/' and `;' may be used to obtain information
about those monsters who are displayed on the screen.  The com-
mand "\#name" (by default bound to `C'), allows you to assign a
name to a monster, which may be useful to help distinguish one
from another when multiple monsters are present. Assigning a
name which is just a space will remove any prior name.

   The extended command "\#chat" can be used to interact with an
adjacent monster. There is no actual dialog (in other words, you
don't get to choose what you'll say), but chatting with some mon-
sters such as a shopkeeper or the Oracle of Delphi can produce
useful results.

6.1. Fighting

   If you see a monster and you wish to fight it, just attempt
to walk into it.  Many monsters you find will mind their own
business unless you attack them. Some of them are very dangerous
when angered. Remember: discretion is the better part of valor.

   In most circumstances, if you attempt to attack a peaceful
monster by moving into its location, you'll be asked to confirm
your intent.  By default an answer of `y' acknowledges that in-
tent, which can be error prone if you're using `y' to move.  You
can set the paranoid\textsubscript{confirmation} option to require a response of
"yes" instead.

   If you can't see a monster (if it is invisible, or if you
are blinded), the symbol `I' will be shown when you learn of its
presence. If you attempt to walk into it, you will try to fight
it just like a monster that you can see; of course, if the mon-
ster has moved, you will attack empty air. If you guess that the
monster has moved and you don't wish to fight, you can use the
`m' command to move without fighting; likewise, if you don't re-
member a monster but want to try fighting anyway, you can use the


NetHack 3.6                   January 27, 2020





NetHack Guidebook                       32



`F' command.

6.2. Your pet

   You start the game with a little dog (`d'), kitten (`f'), or
pony (`u'), which follows you about the dungeon and fights mon-
sters with you. Like you, your pet needs food to survive.  Dogs
and cats usually feed themselves on fresh carrion and other
meats; horses need vegetarian food which is harder to come by.
If you're worried about your pet or want to train it, you can
feed it, too, by throwing it food. A properly trained pet can be
very useful under certain circumstances.

   Your pet also gains experience from killing monsters, and
can grow over time, gaining hit points and doing more damage.
Initially, your pet may even be better at killing things than
you, which makes pets useful for low-level characters.

   Your pet will follow you up and down staircases if it is
next to you when you move. Otherwise your pet will be stranded
and may become wild. Similarly, when you trigger certain types
of traps which alter your location (for instance, a trap door
which drops you to a lower dungeon level), any adjacent pet will
accompany you and any non-adjacent pet will be left behind. Your
pet may trigger such traps itself; you will not be carried along
with it even if adjacent at the time.

6.3. Steeds

   Some types of creatures in the dungeon can actually be rid-
den if you have the right equipment and skill. Convincing a wild
beast to let you saddle it up is difficult to say the least.
Many a dungeoneer has had to resort to magic and wizardry in or-
der to forge the alliance. Once you do have the beast under your
control however, you can easily climb in and out of the saddle
with the "\#ride" command. Lead the beast around the dungeon when
riding, in the same manner as you would move yourself. It is the
beast that you will see displayed on the map.

   Riding skill is managed by the "\#enhance" command. See the
section on Weapon proficiency for more information about that.

   Use the `a' (apply) command and pick a saddle in your inven-
tory to attempt to put that saddle on an adjacent creature. If
successful, it will be transferred to that creature's inventory.

   Use the "\#loot" command while adjacent to a saddled creature
to try to remove the saddle from that creature. If successful,
it will be transferred to your inventory.

6.4. Bones levels

   You may encounter the shades and corpses of other adventur-
ers (or even former incarnations of yourself!) and their personal


NetHack 3.6                   January 27, 2020





NetHack Guidebook                       33



effects. Ghosts are hard to kill, but easy to avoid, since
they're slow and do little damage. You can plunder the deceased
adventurer's possessions; however, they are likely to be cursed.
Beware of whatever killed the former player; it is probably still
lurking around, gloating over its last victory.

6.5. Persistence of Monsters

   Monsters (a generic reference which also includes humans and
pets) are only shown while they can be seen or otherwise sensed.
Moving to a location where you can't see or sense a monster any
more will result in it disappearing from your map, similarly if
it is the one who moved rather than you.

   However, if you encounter a monster which you can't see or
sense -- perhaps it is invisible and has just tapped you on the
noggin -- a special "remembered, unseen monster" marker will be
displayed at the location where you think it is. That will per-
sist until you have proven that there is no monster there, even
if the unseen monster moves to another location or you move to a
spot where the marker's location ordinarily wouldn't be seen any
more.
\begin{enumerate}
\item Objects

   When you find something in the dungeon, it is common to want
to pick it up. In NetHack, this is accomplished automatically by
walking over the object (unless you turn off the autopickup op-
tion (see below), or move with the `m' prefix (see above)), or
manually by using the `,' command.

   If you're carrying too many items, NetHack will tell you so
and you won't be able to pick up anything more.  Otherwise, it
will add the object(s) to your pack and tell you what you just
picked up.

   As you add items to your inventory, you also add the weight
of that object to your load. The amount that you can carry de-
pends on your strength and your constitution. The stronger and
sturdier you are, the less the additional load will affect you.
There comes a point, though, when the weight of all of that stuff
you are carrying around with you through the dungeon will encum-
ber you. Your reactions will get slower and you'll burn calories
faster, requiring food more frequently to cope with it. Eventu-
ally, you'll be so overloaded that you'll either have to discard
some of what you're carrying or collapse under its weight.

   NetHack will tell you how badly you have loaded yourself.
If you are encumbered, one of the conditions  "Burdened",
"Stressed", "Strained", "Overtaxed" or "Overloaded" will be shown
on the bottom line status display.

   When you pick up an object, it is assigned an inventory let-
ter.  Many commands that operate on objects must ask you to find
\end{enumerate}


NetHack 3.6                   January 27, 2020





NetHack Guidebook                       34



out which object you want to use.  When NetHack asks you to
choose a particular object you are carrying, you are usually pre-
sented with a list of inventory letters to choose from (see Com-
mands, above).

   Some objects, such as weapons, are easily differentiated.
Others, like scrolls and potions, are given descriptions which
vary according to type. During a game, any two objects with the
same description are the same type.  However, the descriptions
will vary from game to game.

   When you use one of these objects, if its effect is obvious,
NetHack will remember what it is for you. If its effect isn't
extremely obvious, you will be asked what you want to call this
type of object so you will recognize it later. You can also use
the "\#name" command, for the same purpose at any time, to name
all objects of a particular type or just an individual object.
When you use "\#name" on an object which has already been named,
specifying a space as the value will remove the prior name in-
stead of assigning a new one.

7.1. Curses and Blessings

   Any object that you find may be cursed, even if the object
is otherwise helpful. The most common effect of a curse is being
stuck with (and to) the item. Cursed weapons weld themselves to
your hand when wielded, so you cannot unwield them.  Any cursed
item you wear is not removable by ordinary means. In addition,
cursed arms and armor usually, but not always, bear negative en-
chantments that make them less effective in combat. Other cursed
objects may act poorly or detrimentally in other ways.

   Objects can also be blessed.  Blessed items usually work
better or more beneficially than normal uncursed items. For ex-
ample, a blessed weapon will do more damage against demons.

   Objects which are neither cursed nor blessed are referred to
as uncursed.  They could just as easily have been described as
unblessed, but the uncursed designation is what you will see
within the game.  A "glass half full versus glass half empty"
situation; make of that what you will.

   There are magical means of bestowing or removing curses upon
objects, so even if you are stuck with one, you can still have
the curse lifted and the item removed. Priests and Priestesses
have an innate sensitivity to this property in any object, so
they can more easily avoid cursed objects than other character
roles.

   An item with unknown status will be reported in your inven-
tory with no prefix. An item which you know the state of will be
distinguished in your inventory by the presence of the word
"cursed", "uncursed" or "blessed" in the description of the item.
In some cases "uncursed" will be omitted as being redundant when


NetHack 3.6                   January 27, 2020





NetHack Guidebook                       35



enough other information is displayed. The implicit\textsubscript{uncursed} op-
tion can be used to control this; toggle it off to have "un-
cursed" be displayed even when that can be deduced from other at-
tributes.

7.2. Weapons (`)')

   Given a chance, most monsters in the Mazes of Menace will
gratuitously try to kill you. You need weapons for self-defense
(killing them first).  Without a weapon, you do only 1-2 hit
points of damage (plus bonuses, if any). Monk characters are an
exception; they normally do more damage with bare (or gloved)
hands than they do with weapons.

   There are wielded weapons, like maces and swords, and thrown
weapons, like arrows and spears. To hit monsters with a weapon,
you must wield it and attack them, or throw it at them. You can
simply elect to throw a spear. To shoot an arrow, you should
first wield a bow, then throw the arrow. Crossbows shoot cross-
bow bolts. Slings hurl rocks and (other) stones (like gems).

   Enchanted weapons have a "plus" (or "to hit enhancement"
which can be either positive or negative) that adds to your
chance to hit and the damage you do to a monster. The only way
to determine a weapon's enchantment is to have it magically iden-
tified somehow. Most weapons are subject to some type of damage
like rust. Such "erosion" damage can be repaired.

   The chance that an attack will successfully hit a monster,
and the amount of damage such a hit will do, depends upon many
factors. Among them are: type of weapon, quality of weapon (en-
chantment and/or erosion), experience level, strength, dexterity,
encumbrance, and proficiency (see below).  The monster's armor
class -- a general defense rating, not necessarily due to wearing
of armor -- is a factor too; also, some monsters are particularly
vulnerable to certain types of weapons.

   Many weapons can be wielded in one hand; some require both
hands. When wielding a two-handed weapon, you can not wear a
shield, and vice versa. When wielding a one-handed weapon, you
can have another weapon ready to use by setting things up with
the `x' command, which exchanges your primary (the one being
wielded) and alternate weapons. And if you have proficiency in
the "two weapon combat" skill, you may wield both weapons simul-
taneously as primary and secondary; use the `X' command to engage
or disengage that.  Only some types of characters (barbarians,
for instance) have the necessary skill available. Even with that
skill, using two weapons at once incurs a penalty in the chance
to hit your target compared to using just one weapon at a time.

   There might be times when you'd rather not wield any weapon
at all. To accomplish that, wield `-', or else use the `A' com-
mand which allows you to unwield the current weapon in addition
to taking off other worn items.


NetHack 3.6                   January 27, 2020





NetHack Guidebook                       36



   Those of you in the audience who are AD\&D players, be aware
that each weapon which existed in AD\&D does roughly the same dam-
age to monsters in NetHack. Some of the more obscure weapons
(such as the aklys, lucern hammer, and bec-de-corbin) are defined
in an appendix to Unearthed Arcana, an AD\&D supplement.

   The commands to use weapons are `w' (wield), `t' (throw),
`f' (fire, an alternate way of throwing), `Q' (quiver), `x' (ex-
change), `X' (twoweapon), and "\#enhance" (see below).

7.2.1. Throwing and shooting

   You can throw just about anything via the `t' command. It
will prompt for the item to throw; picking `?' will list things
in your inventory which are considered likely to be thrown, or
picking `*' will list your entire inventory. After you've chosen
what to throw, you will be prompted for a direction rather than
for a specific target. The distance something can be thrown de-
pends mainly on the type of object and your strength. Arrows can
be thrown by hand, but can be thrown much farther and will be
more likely to hit when thrown while you are wielding a bow.

   You can simplify the throwing operation by using the `Q'
command to select your preferred "missile", then using the `f'
command to throw it.  You'll be prompted for a direction as
above, but you don't have to specify which item to throw each
time you use `f'. There is also an option, autoquiver, which has
NetHack choose another item to automatically fill your quiver (or
quiver sack, or have at the ready) when the inventory slot used
for `Q' runs out.

   Some characters have the ability to fire a volley of multi-
ple items in a single turn. Knowing how to load several rounds
of ammunition at once -- or hold several missiles in your hand --
and still hit a target is not an easy task. Rangers are among
those who are adept at this task, as are those with a high level
of proficiency in the relevant weapon skill (in bow skill if
you're wielding one to shoot arrows, in crossbow skill if you're
wielding one to shoot bolts, or in sling skill if you're wielding
one to shoot stones). The number of items that the character has
a chance to fire varies from turn to turn. You can explicitly
limit the number of shots by using a numeric prefix before the
`t' or `f' command. For example, "2f" (or "n2f" if using num-
ber\textsubscript{pad} mode) would ensure that at most 2 arrows are shot even if
you could have fired 3.  If you specify a larger number than
would have been shot ("4f" in this example), you'll just end up
shooting the same number (3, here) as if no limit had been speci-
fied. Once the volley is in motion, all of the items will travel
in the same direction; if the first ones kill a monster, the oth-
ers can still continue beyond that spot.






NetHack 3.6                   January 27, 2020





NetHack Guidebook                       37



7.2.2. Weapon proficiency

   You will have varying degrees of skill in the weapons avail-
able.  Weapon proficiency, or weapon skills, affect how well you
can use particular types of weapons, and you'll be able to im-
prove your skills as you progress through a game, depending on
your role, your experience level, and use of the weapons.

   For the purposes of proficiency, weapons have been divided
up  into various groups such as daggers, broadswords, and
polearms. Each role has a limit on what level of proficiency a
character can achieve for each group. For instance, wizards can
become highly skilled in daggers or staves but not in swords or
bows.

   The "\#enhance" extended command is used to review current
weapons proficiency (also spell proficiency) and to choose which
skill(s) to improve when you've used one or more skills enough to
become eligible to do so. The skill rankings are "none" (some-
times also referred to as "restricted", because you won't be able
to advance), "unskilled", "basic", "skilled", and "expert".  Re-
stricted skills simply will not appear in the list shown by "\#en-
hance".  (Divine intervention might unrestrict a particular
skill, in which case it will start at unskilled and be limited to
basic.) Some characters can enhance their barehanded combat or
martial arts skill beyond expert to "master" or "grand master".

   Use of a weapon in which you're restricted or unskilled will
incur a modest penalty in the chance to hit a monster and also in
the amount of damage done when you do hit; at basic level, there
is no penalty or bonus; at skilled level, you receive a modest
bonus in the chance to hit and amount of damage done; at expert
level, the bonus is higher. A successful hit has a chance to
boost your training towards the next skill level (unless you've
already reached the limit for this skill).  Once such training
reaches the threshold for that next level, you'll be told that
you feel more confident in your skills. At that point you can
use "\#enhance" to increase one or more skills. Such skills are
not increased automatically because there is a limit to your to-
tal overall skills, so you need to actively choose which skills
to enhance and which to ignore.

7.2.3. Two-Weapon combat

   Some characters can use two weapons at once. Setting things
up to do so can seem cumbersome but becomes second nature with
use. To wield two weapons, you need to use the "\#twoweapon" com-
mand.  But first you need to have a weapon in each hand. (Note
that your two weapons are not fully equal; the one in the hand
you normally wield with is considered primary and the other one
is considered secondary. The most noticeable difference is after
you stop -- or before you begin, for that matter -- wielding two
weapons at once. The primary is your wielded weapon and the sec-
ondary is just an item in your inventory that's been designated


NetHack 3.6                   January 27, 2020





NetHack Guidebook                       38



as alternate weapon.)

   If your primary weapon is wielded but your off hand is empty
or has the wrong weapon, use the sequence `x', `w', `x' to first
swap your primary into your off hand, wield whatever you want as
secondary weapon, then swap them both back into the intended
hands. If your secondary or alternate weapon is correct but your
primary one is not, simply use `w' to wield the primary. Lastly,
if neither hand holds the correct weapon, use `w', `x', `w' to
first wield the intended secondary, swap it to off hand, and then
wield the primary.

   The whole process can be simplified via use of the push-
weapon option. When it is enabled, then using `w' to wield some-
thing causes the currently wielded weapon to become your alter-
nate weapon. So the sequence `w', `w' can be used to first wield
the weapon you intend to be secondary, and then wield the one you
want as primary which will push the first into secondary posi-
tion.

   When in two-weapon combat mode, using the `X' command tog-
gles back to single-weapon mode. Throwing or dropping either of
the weapons or having one of them be stolen or destroyed will al-
so make you revert to single-weapon combat.

7.3. Armor (`[')

   Lots of unfriendly things lurk about; you need armor to pro-
tect yourself from their blows. Some types of armor offer better
protection than others.  Your armor class is a measure of this
protection. Armor class (AC) is measured as in AD\&D, with 10 be-
ing the equivalent of no armor, and lower numbers meaning better
armor. Each suit of armor which exists in AD\&D gives the same
protection in NetHack. Here is an (incomplete) list of the armor
classes provided by various suits of armor:
	  dragon scale mail     1
	  plate mail        3
	  crystal plate mail    3
	  bronze plate mail     4
	  splint mail        4
	  banded mail        4
	  dwarvish mithril-coat   4
	  elven mithril-coat    5
	  chain mail        5
	  orcish chain mail     6
	  scale mail        6
	  dragon scales       7
	  studded leather armor   7
	  ring mail         7
	  orcish ring mail     8
	  leather armor       8
	  leather jacket      9
	  no armor         10



NetHack 3.6                   January 27, 2020





NetHack Guidebook                       39



   You can also wear other pieces of armor (for example hel-
mets, boots, shields, cloaks) to lower your armor class even fur-
ther, but you can only wear one item of each category (one suit
of armor, one cloak, one helmet, one shield, and so on) at a
time.

   If a piece of armor is enchanted, its armor protection will
be better (or worse) than normal, and its "plus" (or minus) will
subtract from your armor class. For example, a +1 chain mail
would give you better protection than normal chain mail, lowering
your armor class one unit further to 4. When you put on a piece
of armor, you immediately find out the armor class and any
"plusses" it provides. Cursed pieces of armor usually have nega-
tive enchantments (minuses) in addition to being unremovable.

   Many types of armor are subject to some kind of damage like
rust. Such damage can be repaired. Some types of armor may in-
hibit spell casting.

   The commands to use armor are `W' (wear) and `T' (take off).
The `A' command can also be used to take off armor as well as
other worn items.

7.4. Food (`\%')

   Food is necessary to survive. If you go too long without
eating you will faint, and eventually die of starvation.  Some
types of food will spoil, and become unhealthy to eat, if not
protected. Food stored in ice boxes or tins ("cans") will usual-
ly stay fresh, but ice boxes are heavy, and tins take a while to
open.

   When you kill monsters, they usually leave corpses which are
also "food."  Many, but not all, of these are edible; some also
give you special powers when you eat them. A good rule of thumb
is "you are what you eat."

   Some character roles and some monsters are vegetarian. Veg-
etarian monsters will typically never eat animal corpses, while
vegetarian players can, but with some rather unpleasant side-ef-
fects.

   You can name one food item after something you like to eat
with the fruit option.

The command to eat food is `e'.

7.5. Scrolls (`?')

   Scrolls are labeled with various titles, probably chosen by
ancient wizards for their amusement value (for example "READ ME,"
or "THANX MAUD" backwards).  Scrolls disappear after you read
them (except for blank ones, without magic spells on them).



NetHack 3.6                   January 27, 2020





NetHack Guidebook                       40



   One of the most useful of these is the scroll of identify,
which can be used to determine what another object is, whether it
is cursed or blessed, and how many uses it has left.  Some ob-
jects of subtle enchantment are difficult to identify without
these.

   A mail daemon may run up and deliver mail to you as a scroll
of mail (on versions compiled with this feature). To use this
feature on versions where NetHack mail delivery is triggered by
electronic mail appearing in your system mailbox, you must let
NetHack know where to look for new mail by setting the "MAIL" en-
vironment variable to the file name of your mailbox. You may al-
so want to set the "MAILREADER" environment variable to the file
name of your favorite reader, so NetHack can shell to it when you
read the scroll. On versions of NetHack where mail is randomly
generated internal to the game, these environment variables are
ignored. You can disable the mail daemon by turning off the mail
option.

The command to read a scroll is `r'.

7.6. Potions (`!')

   Potions are distinguished by the color of the liquid inside
the flask. They disappear after you quaff them.

   Clear potions are potions of water.  Sometimes these are
blessed or cursed, resulting in holy or unholy water. Holy water
is the bane of the undead, so potions of holy water are good
things to throw (`t') at them. It is also sometimes very useful
to dip ("\#dip") an object into a potion.

The command to drink a potion is `q' (quaff).

7.7. Wands (`/')

   Wands usually have multiple magical charges. Some types of
wands require a direction in which to zap them. You can also zap
them at yourself (just give a `.' or `s' for the direction).  Be
warned, however, for this is often unwise. Other types of wands
don't require a direction. The number of charges in a wand is
random and decreases by one whenever you use it.

   When the number of charges left in a wand becomes zero, at-
tempts to use the wand will usually result in nothing happening.
Occasionally, however, it may be possible to squeeze the last few
mana points from an otherwise spent wand, destroying it in the
process.  A wand may be recharged by using suitable magic, but
doing so runs the risk of causing it to explode. The chance for
such an explosion starts out very small and increases each time
the wand is recharged.

   In a truly desperate situation, when your back is up against
the wall, you might decide to go for broke and break your wand.


NetHack 3.6                   January 27, 2020





NetHack Guidebook                       41



This is not for the faint of heart. Doing so will almost cer-
tainly cause a catastrophic release of magical energies.

   When you have fully identified a particular wand, inventory
display will include additional information in parentheses: the
number of times it has been recharged followed by a colon and
then by its current number of charges. A current charge count of
-1 is a special case indicating that the wand has been cancelled.

   The command to use a wand is `z' (zap). To break one, use
the `a' (apply) command.

7.8. Rings (`=')

   Rings are very useful items, since they are relatively per-
manent magic, unlike the usually fleeting effects of potions,
scrolls, and wands.

   Putting on a ring activates its magic. You can wear only
two rings, one on each ring finger.

   Most rings also cause you to grow hungry more rapidly, the
rate varying with the type of ring.

The commands to use rings are `P' (put on) and `R' (remove).

7.9. Spellbooks (`+')

   Spellbooks are tomes of mighty magic. When studied with the
`r' (read) command, they transfer to the reader the knowledge of
a spell (and therefore eventually become unreadable) -- unless
the attempt backfires.  Reading a cursed spellbook or one with
mystic runes beyond your ken can be harmful to your health!

   A spell (even when learned) can also backfire when you cast
it.  If you attempt to cast a spell well above your experience
level, or if you have little skill with the appropriate spell
type, or cast it at a time when your luck is particularly bad,
you can end up wasting both the energy and the time required in
casting.

   Casting a spell calls forth magical energies and focuses
them with your naked mind. Some of the magical energy released
comes from within you. Casting temporarily drains your magical
power, which will slowly be recovered, and causes you to need ad-
ditional food.  Casting of spells also requires practice. With
practice, your skill in each category of spell casting will im-
prove.  Over time, however, your memory of each spell will dim,
and you will need to relearn it.

   Some spells require a direction in which to cast them, simi-
lar to wands.  To cast one at yourself, just give a `.' or `s'
for the direction. A few spells require you to pick a target lo-
cation rather than just specify a particular direction. Other


NetHack 3.6                   January 27, 2020





NetHack Guidebook                       42



spells don't require any direction or target.

   Just as weapons are divided into groups in which a character
can become proficient (to varying degrees), spells are similarly
grouped. Successfully casting a spell exercises its skill group;
using the "\#enhance" command to advance a sufficiently exercised
skill will affect all spells within the group.  Advanced skill
may increase the potency of spells, reduce their risk of failure
during casting attempts, and improve the accuracy of the estimate
for how much longer they will be retained in your memory. Skill
slots are shared with weapons skills. (See also the section on
"Weapon proficiency".)

   Casting a spell also requires flexible movement, and wearing
various types of armor may interfere with that.

   The command to read a spellbook is the same as for scrolls,
`r' (read). The `+' command lists each spell you know along with
its level, skill category, chance of failure when casting, and an
estimate of how strongly it is remembered. The `Z' (cast) com-
mand casts a spell.

7.10. Tools (`(')

   Tools are miscellaneous objects with various purposes. Some
tools have a limited number of uses, akin to wand charges. For
example, lamps burn out after a while. Other tools are contain-
ers, which objects can be placed into or taken out of.

The command to use a tool is `a' (apply).

7.10.1. Containers

   You may encounter bags, boxes, and chests in your travels.
A tool of this sort can be opened with the "\#loot" extended com-
mand when you are standing on top of it (that is, on the same
floor spot), or with the `a' (apply) command when you are carry-
ing it.  However, chests are often locked, and are in any case
unwieldy objects. You must set one down before unlocking it by
using a key or lock-picking tool with the `a' (apply) command, by
kicking it with the `\textsuperscript{D}' command, or by using a weapon to force
the lock with the "\#force" extended command.

   Some chests are trapped, causing nasty things to happen when
you unlock or open them. You can check for and try to deactivate
traps with the "\#untrap" extended command.

7.11. Amulets (`"')

   Amulets are very similar to rings, and often more powerful.
Like rings, amulets have various magical properties, some benefi-
cial, some harmful, which are activated by putting them on.




NetHack 3.6                   January 27, 2020





NetHack Guidebook                       43



Only one amulet may be worn at a time, around your neck.

   The commands to use amulets are the same as for rings, `P'
(put on) and `R' (remove).

7.12. Gems (`*')

   Some gems are valuable, and can be sold for a lot of gold.
They are also a far more efficient way of carrying your riches.
Valuable gems increase your score if you bring them with you when
you exit.

   Other small rocks are also categorized as gems, but they are
much less valuable. All rocks, however, can be used as projec-
tile weapons (if you have a sling). In the most desperate of
cases, you can still throw them by hand.

7.13. Large rocks (``')

   Statues and boulders are not particularly useful, and are
generally heavy.  It is rumored that some statues are not what
they seem.

   Very large humanoids (giants and their ilk) have been known
to use boulders as weapons.

   For some configurations of the program, statues are no
longer shown as ``' but by the letter representing the monster
they depict instead.

7.14. Gold (`\$')

   Gold adds to your score, and you can buy things in shops
with it. There are a number of monsters in the dungeon that may
be influenced by the amount of gold you are carrying (shopkeepers
aside).

7.15. Persistence of Objects

   Normally, if you have seen an object at a particular map lo-
cation and move to another location where you can't directly see
that object any more, if will continue to be displayed on your
map.  That remains the case even if it is not actually there any
more -- perhaps a monster has picked it up or it has rotted away
-- until you can see or feel that location again. One notable
exception is that if the object gets covered by the "remembered,
unseen monster" marker. When that marker is later removed after
you've verified that no monster is there, you will forget that
there was any object there regardless of whether the unseen mon-
ster actually took the object. If the object is still there,
then once you see or feel that location again you will re-discov-
er the object and resume remembering it.




NetHack 3.6                   January 27, 2020





NetHack Guidebook                       44



   The situation is the same for a pile of objects, except that
only the top item of the pile is displayed. The hilite\textsubscript{pile} op-
tion can be enabled in order to show an item differently when it
is the top one of a pile.

\begin{enumerate}
\item Conduct

As if winning NetHack were not difficult enough, certain
\end{enumerate}
players seek to challenge themselves by imposing restrictions on
the way they play the game. The game automatically tracks some
of these challenges, which can be checked at any time with the
\#conduct command or at the end of the game. When you perform an
action which breaks a challenge, it will no longer be listed.
This gives players extra "bragging rights" for winning the game
with these challenges. Note that it is perfectly acceptable to
win the game without resorting to these restrictions and that it
is unusual for players to adhere to challenges the first time
they win the game.

   Several of the challenges are related to eating behavior.
The most difficult of these is the foodless challenge.  Although
creatures can survive long periods of time without food, there is
a physiological need for water; thus there is no restriction on
drinking beverages, even if they provide some minor food bene-
fits. Calling upon your god for help with starvation does not
violate any food challenges either.

   A strict vegan diet is one which avoids any food derived
from animals. The primary source of nutrition is fruits and veg-
etables. The corpses and tins of blobs (`b'), jellies (`j'), and
fungi (`F') are also considered to be vegetable matter.  Certain
human food is prepared without animal products; namely, lembas
wafers, cram rations, food rations (gunyoki), K-rations, and C-
rations.  Metal or another normally indigestible material eaten
while polymorphed into a creature that can digest it is also con-
sidered vegan food.  Note however that eating such items still
counts against foodless conduct.

   Vegetarians do not eat animals; however, they are less se-
lective about eating animal byproducts than vegans. In addition
to the vegan items listed above, they may eat any kind of pudding
(`P') other than the black puddings, eggs and food made from eggs
(fortune cookies and pancakes), food made with milk (cream pies
and candy bars), and lumps of royal jelly. Monks are expected to
observe a vegetarian diet.

   Eating any kind of meat violates the vegetarian, vegan, and
foodless conducts.  This includes tripe rations, the corpses or
tins of any monsters not mentioned above, and the various other
chunks of meat found in the dungeon. Swallowing and digesting a
monster while polymorphed is treated as if you ate the creature's
corpse.  Eating leather, dragon hide, or bone items while poly-
morphed into a creature that can digest it, or eating monster
brains while polymorphed into a mind flayer, is considered eating


NetHack 3.6                   January 27, 2020





NetHack Guidebook                       45



an animal, although wax is only an animal byproduct.

   Regardless of conduct, there will be some items which are
indigestible, and others which are hazardous to eat. Using a
swallow-and-digest attack against a monster is equivalent to eat-
ing the monster's corpse. Please note that the term "vegan" is
used here only in the context of diet. You are still free to
choose not to use or wear items derived from animals (e.g.
leather, dragon hide, bone, horns, coral), but the game will not
keep track of this for you. Also note that "milky" potions may
be a translucent white, but they do not contain milk, so they are
compatible with a vegan diet.  Slime molds or player-defined
"fruits", although they could be anything from "cherries" to
"pork chops", are also assumed to be vegan.

   An atheist is one who rejects religion. This means that you
cannot \#pray, \#offer sacrifices to any god, \#turn undead, or
\#chat with a priest. Particularly selective readers may argue
that playing Monk or Priest characters should violate this con-
duct; that is a choice left to the player. Offering the Amulet
of Yendor to your god is necessary to win the game and is not
counted against this conduct. You are also not penalized for be-
ing spoken to by an angry god, priest(ess), or other religious
figure; a true atheist would hear the words but attach no special
meaning to them.

   Most players fight with a wielded weapon (or tool intended
to be wielded as a weapon). Another challenge is to win the game
without using such a wielded weapon. You are still permitted to
throw, fire, and kick weapons; use a wand, spell, or other type
of item; or fight with your hands and feet.

   In NetHack, a pacifist refuses to cause the death of any
other monster (i.e. if you would get experience for the death).
This is a particularly difficult challenge, although it is still
possible to gain experience by other means.

   An illiterate character cannot read or write. This includes
reading a scroll, spellbook, fortune cookie message, or t-shirt;
writing a scroll; or making an engraving of anything other than a
single "X" (the traditional signature of an illiterate person).
Reading an engraving, or any item that is absolutely necessary to
win the game, is not counted against this conduct. The identity
of scrolls and spellbooks (and knowledge of spells) in your
starting inventory is assumed to be learned from your teachers
prior to the start of the game and isn't counted.

   There are several other challenges tracked by the game. It
is possible to eliminate one or more species of monsters by geno-
cide; playing without this feature is considered a challenge.
When the game offers you an opportunity to genocide monsters, you
may respond with the monster type "none" if you want to decline.
You can change the form of an item into another item of the same
type ("polypiling") or the form of your own body into another


NetHack 3.6                   January 27, 2020





NetHack Guidebook                       46



creature ("polyself") by wand, spell, or potion of polymorph;
avoiding these effects are each considered challenges. Polymor-
phing monsters, including pets, does not break either of these
challenges.  Finally, you may sometimes receive wishes; a game
without an attempt to wish for any items is a challenge, as is a
game without wishing for an artifact (even if the artifact imme-
diately disappears). When the game offers you an opportunity to
make a wish for an item, you may choose "nothing" if you want to
decline.

\begin{enumerate}
\item Options

Due to variations in personal tastes and conceptions of how
\end{enumerate}
NetHack should do things, there are options you can set to change
how NetHack behaves.

9.1. Setting the options

   Options may be set in a number of ways.  Within the game,
the `O' command allows you to view all options and change most of
them. You can also set options automatically by placing them in
a configuration file, or in the NETHACKOPTIONS environment vari-
able. Some versions of NetHack also have front-end programs that
allow you to set options before starting the game or a global
configuration for system administrators.

9.2. Using a configuration file

   The default name of the configuration file varies on differ-
ent operating systems.

   On UNIX, Linux, and Mac OS X it is ".nethackrc" in the us-
er's home directory. The file may not exist, but it is a normal
ASCII text file and can be created with any text editor.

   On Windows, it is ".nethackrc" in the folder "$\backslash$%USERPRO-
FILE\%\NetHack$\backslash$3.6". The file may not exist, but it is a normal
ASCII text file can can be created with any text editor. After
running NetHack for the first time, you should find a default
template for the configuration file named ".nethackrc.template"
in "$\backslash$%USERPROFILE\%\NetHack$\backslash$3.6". If you had not created the con-
figuration file, NetHack will create the configuration file for
you using the default template file.

   On MS-DOS, it is "defaults.nh" in the same folder as
nethack.exe.

   Any line in the configuration file starting with `\#' is
treated as a comment. Empty lines are ignored.

   Any line beginning with `[' and ending in `]' is considered
a section marker.  The text between the square brackets is the
section name. Lines after a section marker belong to that sec-
tion, and are ignored unless a CHOOSE statement was used to


NetHack 3.6                   January 27, 2020





NetHack Guidebook                       47



select that section. Section names are case insensitive.

   You can use different configuration statements in the file,
some of which can be used multiple times. In general, the state-
ments are written in capital letters, followed by an equals sign,
followed by settings particular to that statement.

Here is a list of allowed statements:

OPTIONS
 There are two types of options, boolean and compound options.
 Boolean options toggle a setting on or off, while compound op-
 tions take more diverse values. Prefix a boolean option with
 "no" or `!' to turn it off. For compound options, the option
 name and value are separated by a colon. Some options are per-
 sistent, and apply only to new games. You can specify multiple
 OPTIONS statements, and multiple options separated by commas in
 a single OPTIONS statement. (Comma separated options are pro-
 cessed from right to left.)

Example:

OPTIONS=dogname:Fido
OPTIONS=!legacy,autopickup,pickup\textsubscript{types}:\$"=/!?+

HACKDIR
 Default location of files NetHack needs. On Windows HACKDIR
 defaults to the location of the NetHack.exe or NetHackw.exe
 file so setting HACKDIR to override that is not usually neces-
 sary or recommended.

LEVELDIR
 The location that in-progress level files are stored. Defaults
 to HACKDIR, must be writable.

SAVEDIR
 The location where saved games are kept. Defaults to HACKDIR,
 must be writable.

BONESDIR
 The location that bones files are kept. Defaults to HACKDIR,
 must be writable.

LOCKDIR
 The location that file synchronization locks are stored.
 Defaults to HACKDIR, must be writable.

TROUBLEDIR
 The location that a record of game aborts and self-diagnosed
 game problems is kept. Defaults to HACKDIR, must be writable.

AUTOCOMPLETE
 Enable or disable an extended command autocompletion. Autocom-
 pletion has no effect for the X11 windowport. You can specify


NetHack 3.6                   January 27, 2020





NetHack Guidebook                       48



multiple autocompletions.  To enable autocompletion, list the
extended command. Prefix the command with "!" to disable the
autocompletion for that command.

Example:

AUTOCOMPLETE=zap,!annotate

AUTOPICKUP\textsubscript{EXCEPTION}
 Set exceptions to the pickup\textsubscript{types} option. See the "Configur-
 ing Autopickup Exceptions" section.

BINDINGS
 Change the key bindings of some special keys, menu accelera-
 tors, or extended commands. You can specify multiple bindings.
 Format is key followed by the command, separated by a colon.
 See the "Changing Key Bindings" section for more information.

Example:

BIND=\textsuperscript{X}:getpos.autodescribe

CHOOSE
 Chooses at random one of the comma-separated parameters as an
 active section name. Lines in other sections are ignored.

Example:

OPTIONS=color
CHOOSE=char A,char B
[char A]
OPTIONS=role:arc,race:dwa,align:law,gender:fem
[char B]
OPTIONS=role:wiz,race:elf,align:cha,gender:mal

MENUCOLOR
 Highlight menu lines with different colors. See the "Configur-
 ing Menu Colors" section.

MSGTYPE
 Change the way messages are shown in the top status line. See
 the "Configuring Message Types" section.

ROGUESYMBOLS
 Custom symbols for for the rogue level's symbol set. See SYM-
 BOLS below.

SOUND
 Define a sound mapping. See the "Configuring User Sounds" sec-
 tion.

SOUNDDIR
 Define the directory that contains the sound files.  See the
 "Configuring User Sounds" section.


NetHack 3.6                   January 27, 2020





NetHack Guidebook                       49



SYMBOLS
 Override one or more symbols in the symbol set used for all
 dungeon levels except for the special rogue level.  See the
 "Modifying NetHack Symbols" section.

Example:

SYMBOLS=S\textsubscript{boulder}:0,S\textsubscript{golem}:7

WIZKIT
 Debug mode only:  extra items to add to initial inventory.
 Value is the name of a text file containing a list of item
 names, one per line, up to a maximum of 128 lines. Each line
 is processed by the function that handles wishing.

Example:

WIZKIT=\textasciitilde{}/wizkit.txt



Here is an example of configuration file contents:

OPTIONS=role:Valkyrie, race:Human, gender:female, align:lawful

OPTIONS=autopickup,pickup\textsubscript{types}:\$"=/!?+

OPTIONS=color      \# Display things in color if possible
OPTIONS=lit\textsubscript{corridor}  \# Show lit corridors differently
OPTIONS=hilite\textsubscript{pet,hilite}\textsubscript{pile}

SYMBOLS=S\textsubscript{boulder}:0,S\textsubscript{golem}:7

OPTIONS=!splash\textsubscript{screen}



9.3. Using the NETHACKOPTIONS environment variable

   The NETHACKOPTIONS variable is a comma-separated list of
initial values for the various options. Some can only be turned
on or off. You turn one of these on by adding the name of the
option to the list, and turn it off by typing a `!' or "no" be-
fore the name. Others take a character string as a value.  You
can set string options by typing the option name, a colon or
equals sign, and then the value of the string. The value is ter-
minated by the next comma or the end of string.




NetHack 3.6                   January 27, 2020





NetHack Guidebook                       50



   For example, to set up an environment variable so that color
is on, legacy is off, character name is set to "Blue Meanie", and
named fruit is set to "lime", you would enter the command

\% setenv NETHACKOPTIONS "color,$\backslash$!leg,name:Blue Meanie,fruit:lime"

in csh (note the need to escape the `!' since it's special to
that shell), or the pair of commands

\$ NETHACKOPTIONS="color,!leg,name:Blue Meanie,fruit:lime"
\$ export NETHACKOPTIONS

in sh, ksh, or bash.

   The NETHACKOPTIONS value is effectively the same as a single
OPTIONS statement in a configuration file. The "OPTIONS=" prefix
is implied and comma separated options are processed from right
to left. Other types of configuration statements such as BIND or
MSGTYPE are not allowed.

   Instead of a comma-separated list of options, NETHACKOPTIONS
can be set to the full name of a configuration file you want to
use. If that full name doesn't start with a slash, precede it
with `@' (at-sign) to let NetHack know that the rest is intended
as a file name. If it does start with `/', the at-sign is op-
tional.

9.4. Customization options

   Here are explanations of what the various options do. Char-
acter strings that are too long may be truncated.  Some of the
options listed may be inactive in your dungeon.

   Some options are persistent, and are saved and reloaded
along with the game. Changing a persistent option in the config-
uration file applies only to new games.

acoustics
 Enable messages about what your character hears (default on).
 Note that this has nothing to do with your computer's audio ca-
 pabilities. Persistent.

align
 Your  starting  alignment (align:lawful, align:neutral, or
 align:chaotic). You may specify just the first letter.  The
 default is to randomly pick an appropriate alignment. If you
 prefix the value with `!' or "no", you will exclude that align-
 ment from being picked randomly. Cannot be set with the `O'
 command. Persistent.

autodescribe
 Automatically describe the terrain under cursor when asked to
 get a location on the map (default true). The whatis\textsubscript{coord} op-
 tion controls whether the description includes map coordinates.


NetHack 3.6                   January 27, 2020





NetHack Guidebook                       51



autodig
 Automatically dig if you are wielding a digging tool and moving
 into a place that can be dug (default false). Persistent.

autoopen
 Walking into a door attempts to open it (default true). Persis-
 tent.

autopickup
 Automatically pick up things onto which you move (default on).
 Persistent. See pickup\textsubscript{types} to refine the behavior.

autoquiver
 This option controls what happens when you attempt the `f'
 (fire) command when nothing is quivered or readied (default
 false).  When true, the computer will fill your quiver or
 quiver sack or make ready some suitable weapon. Note that it
 will not take into account the blessed/cursed status, enchant-
 ment, damage, or quality of the weapon; you are free to manual-
 ly fill your quiver or quiver sack or make ready with the `Q'
 command instead. If no weapon is found or the option is false,
 the `t' (throw) command is executed instead. Persistent.

blind
 Start the character permanently blind (default false). Persis-
 tent.

bones
 Allow saving and loading bones files (default true).  Persis-
 tent.

boulder
 Set the character used to display boulders (default is the
 "large rock" class symbol, ``').

catname
 Name your starting cat (for example "catname:Morris").  Cannot
 be set with the `O' command.

character
 Synonym for "role" to pick the type of your character (for ex-
 ample "character:Monk"). See role for more details.

checkpoint
 Save game state after each level change, for possible recovery
 after program crash (default on). Persistent.

checkspace
 Check free disk space before writing files to disk (default
 on). You may have to turn this off if you have more than 2 GB
 free space on the partition used for your save and level files
 (because too much space might overflow the calculation and end
 up looking like insufficient space). Only applies when MFLOPPY
 was defined during compilation.


NetHack 3.6                   January 27, 2020





NetHack Guidebook                       52



clicklook
 Allows looking at things on the screen by navigating the mouse
 over them and clicking the right mouse button (default off).

cmdassist
 Have the game provide some additional command assistance for
 new players if it detects some anticipated mistakes (default
 on).

confirm
 Have user confirm attacks on pets, shopkeepers, and other
 peaceable creatures (default on). Persistent.

dark\textsubscript{room}
 Show out-of-sight areas of lit rooms (default on). Persistent.

disclose
 Controls what information the program reveals when the game
 ends.  Value is a space separated list of prompting/category
 pairs (default is "ni na nv ng nc no", prompt with default re-
 sponse of `n' for each candidate). Persistent. The possibili-
 ties are:

i - disclose your inventory;
a - disclose your attributes;
v - summarize monsters that have been vanquished;
g - list monster species that have been genocided;
c - display your conduct;
o - display dungeon overview.

Each disclosure possibility can optionally be preceded by a
prefix which lets you refine how it behaves. Here are the
valid prefixes:

y - prompt you and default to yes on the prompt;
n - prompt you and default to no on the prompt;
\begin{itemize}
\item - disclose it without prompting;
\item - do not disclose it and do not prompt.
\end{itemize}

The listing of vanquished monsters can be sorted, so there are
two additional choices for `v':

? - prompt you and default to ask on the prompt;

Asking refers to picking one of the orderings from a menu. The
`+' disclose without prompting choice, or being prompted and
answering `y' rather than `a', will default to showing monsters
in the traditional order, from high level to low level.

Omitted categories are implicitly added with `n' prefix. Spec-
ified categories with omitted prefix implicitly use `+' prefix.
Order of the disclosure categories does not matter, program
display for end-of-game disclosure follows a set sequence.


NetHack 3.6                   January 27, 2020


NetHack Guidebook                       53



(for example "disclose:yi na +v -g o") The example sets inven-
tory to prompt and default to yes, attributes to prompt and de-
fault to no, vanquished to disclose without prompting, genocid-
ed to not disclose and not prompt, conduct to implicitly prompt
and default to no, and overview to disclose without prompting.

Note that the vanquished monsters list includes all monsters
killed by traps and each other as well as by you. And the dun-
geon overview shows all levels you had visited but does not re-
veal things about them that you hadn't discovered.

dogname
 Name your starting dog (for example "dogname:Fang"). Cannot be
 set with the `O' command.

extmenu
 Changes the extended commands interface to pop-up a menu of
 available commands. It is keystroke compatible with the tradi-
 tional interface except that it does not require that you hit
 Enter. It is implemented for the tty interface (default off).

For the X11 interface, which always uses a menu for choosing an
extended command, it controls whether the menu shows all avail-
able commands (on) or just the subset of commands which have
traditionally been considered extended ones (off).

female
 An obsolete synonym for "gender:female". Cannot be set with
 the `O' command.

fixinv
 An object's inventory letter sticks to it when it's dropped
 (default on).  If this is off, dropping an object shifts all
 the remaining inventory letters. Persistent.

force\textsubscript{invmenu}
 Commands asking for an inventory item show a menu instead of a
 text query with possible menu letters. Default is off.

fruit
 Name a fruit after something you enjoy eating (for example
 "fruit:mango") (default "slime mold"). Basically a nostalgic
 whimsy that NetHack uses from time to time. You should set
 this to something you find more appetizing than slime mold.
 Apples, oranges, pears, bananas, and melons already exist in
 NetHack, so don't use those.

gender
 Your starting gender (gender:male or gender:female).  You may
 specify just the first letter. Although you can still denote
 your gender using the "male" and "female" options, the "gender"
 option will take precedence. The default is to randomly pick
 an appropriate gender. If you prefix the value with `!' or
 "no", you will exclude that gender from being picked randomly.


NetHack 3.6                   January 27, 2020





NetHack Guidebook                       54



Cannot be set with the `O' command. Persistent.

goldX
 When filtering objects based on bless/curse state (BUCX),
 whether to treat gold pieces as X (unknown bless/curse state,
 when "on") or U (known to be uncursed, when "off", the de-
 fault).  Gold is never blessed or cursed, but it is not de-
 scribed as "uncursed" even when the implicit\textsubscript{uncursed} option is
 "off".

help
 If more information is available for an object looked at with
 the `/' command, ask if you want to see it (default on). Turn-
 ing help off makes just looking at things faster, since you
 aren't interrupted with the "More info?" prompt, but it also
 means that you might miss some interesting and/or important in-
 formation. Persistent.

herecmd\textsubscript{menu}
 When using a windowport that supports mouse and clicking on
 yourself or next to you, show a menu of possible actions for
 the location. Same as "\#herecmdmenu" and "\#therecmdmenu" com-
 mands.

hilite\textsubscript{pet}
 Visually distinguish pets from similar animals (default off).
 The behavior of this option depends on the type of windowing
 you use. In text windowing, text highlighting or inverse video
 is often used; with tiles, generally displays a heart symbol
 near pets.

With the curses interface, the petattr option controls how to
highlight pets and setting it will turn the hilite\textsubscript{pet} option
on or off as warranted.

hilite\textsubscript{pile}
 Visually distinguish piles of objects from individual objects
 (default off). The behavior of this option depends on the type
 of windowing you use. In text windowing, text highlighting or
 inverse video is often used; with tiles, generally displays a
 small plus-symbol beside the object on the top of the pile.

hitpointbar
 Show a hit point bar graph behind your name and title. Only
 available for TTY and Windows GUI, and only when statushilites
 is on.

horsename
 Name your starting horse (for example "horsename:Trigger").
 Cannot be set with the `O' command.

ignintr
 Ignore interrupt signals, including breaks (default off). Per-
 sistent.


NetHack 3.6                   January 27, 2020





NetHack Guidebook                       55



implicit\textsubscript{uncursed}
 Omit "uncursed" from inventory lists, if possible (default on).

legacy
 Display an introductory message when starting the game (default
 on). Persistent.

lit\textsubscript{corridor}
 Show corridor squares seen by night vision or a light source
 held by your character as lit (default off). Persistent.

lootabc
 When using a menu to interact with a container, use the old
 `a', `b', and `c' keyboard shortcuts rather than the mnemonics
 `o', `i', and `b' (default off). Persistent.

mail
 Enable mail delivery during the game (default on). Persistent.

male
 An obsolete synonym for "gender:male". Cannot be set with the
 `O' command.

mention\textsubscript{walls}
 Give feedback when walking against a wall (default off).

menucolors
 Enable coloring menu lines (default off).  See "Configuring
 Menu Colors" on how to configure the colors.

menustyle
 Controls the interface used when you need to choose various ob-
 jects (in response to the Drop command, for instance).  The
 value specified should be the first letter of one of the fol-
 lowing: traditional, combination, full, or partial.  Tradi-
 tional was the only interface available for early versions; it
 consists of a prompt for object class characters, followed by
 an object-by-object prompt for all items matching the selected
 object class(es). Combination starts with a prompt for object
 class(es) of interest, but then displays a menu of matching ob-
 jects rather than prompting one-by-one. Full displays a menu
 of object classes rather than a character prompt, and then a
 menu of matching objects for selection. Partial skips the ob-
 ject class filtering and immediately displays a menu of all ob-
 jects. Persistent.

menu\textsubscript{deselect}\textsubscript{all}
 Menu character accelerator to deselect all items in a menu.
 Implemented by the Amiga, Gem, X11 and tty ports. Default `-'.

menu\textsubscript{deselect}\textsubscript{page}
 Menu character accelerator to deselect all items on this page
 of a menu. Implemented by the Amiga, Gem and tty ports.  De-
 fault `$\backslash$'.


NetHack 3.6                   January 27, 2020





NetHack Guidebook                       56



menu\textsubscript{first}\textsubscript{page}
 Menu character accelerator to jump to the first page in a menu.
 Implemented by the Amiga, Gem and tty ports. Default `\^{}'.

menu\textsubscript{headings}
 Controls how the headings in a menu are highlighted.  Values
 are "none", "bold", "dim", "underline", "blink", or "inverse".
 Not all ports can actually display all types.

menu\textsubscript{invert}\textsubscript{all}
 Menu character accelerator to invert all items in a menu.  Im-
 plemented by the Amiga, Gem, X11 and tty ports. Default `@'.

menu\textsubscript{invert}\textsubscript{page}
 Menu character accelerator to invert all items on this page of
 a menu. Implemented by the Amiga, Gem and tty ports.  Default
 `\textasciitilde{}'.

menu\textsubscript{last}\textsubscript{page}
 Menu character accelerator to jump to the last page in a menu.
 Implemented by the Amiga, Gem and tty ports. Default `|'.

menu\textsubscript{next}\textsubscript{page}
 Menu character accelerator to goto the next menu page.  Imple-
 mented by the Amiga, Gem and tty ports. Default `>'.

menu\textsubscript{objsyms}
 Show object symbols in menu headings in menus where the object
 symbols act as menu accelerators (default off).

menu\textsubscript{overlay}
 Do not clear the screen before drawing menus, and align menus
 to the right edge of the screen. Only for the tty port. (de-
 fault on)

menu\textsubscript{previous}\textsubscript{page}
 Menu character accelerator to goto the previous menu page. Im-
 plemented by the Amiga, Gem and tty ports. Default `<'.

menu\textsubscript{search}
 Menu character accelerator to search for a menu item. Imple-
 mented by the Amiga, Gem, X11 and tty ports. Default `:'.

menu\textsubscript{select}\textsubscript{all}
 Menu character accelerator to select all items in a menu.  Im-
 plemented by the Amiga, Gem, X11 and tty ports. Default `.'.

menu\textsubscript{select}\textsubscript{page}
 Menu character accelerator to select all items on this page of
 a menu. Implemented by the Amiga, Gem and tty ports.  Default
 `,'.

monpolycontrol
 Prompt for new form whenever any monster changes shape (default


NetHack 3.6                   January 27, 2020





NetHack Guidebook                       57



off). Debug mode only.

mouse\textsubscript{support}
 Allow use of the mouse for input and travel.  Valid settings
 are:

0 - disabled
1 - enabled and make OS adjustments to support mouse use
2 - like 1 but does not make any OS adjustments

Omitting a value is the same as specifying 1 and negating
mouse\textsubscript{support} is the same as specifying 0.

msghistory
 The number of top line messages to keep (and be able to recall
 with `\textsuperscript{P}') (default 20). Cannot be set with the `O' command.

msg\textsubscript{window}
 Allows you to change the way recalled messages are displayed.
 Currently it is only supported for tty (all four choices) and
 for curses (`f' and `r' choices, default `r'). The possible
 values are:

s - single message (default; only choice prior to 3.4.0);
c - combination, two messages as "single", then as "full";
f - full window, oldest message first;
r - full window reversed, newest message first.

For backward compatibility, no value needs to be specified
(which defaults to "full"), or it can be negated (which
defaults to "single").

name
 Set your character's name (defaults to your user name).  You
 can also set your character's role by appending a dash and one
 or more letters of the role (that is, by suffixing one of -A -B
 -C -H -K -M -P -Ra -Ro -S -T -V -W). If -@ is used for the
 role, then a random one will be automatically chosen.  Cannot
 be set with the `O' command.

news
 Read the NetHack news file, if present (default on). Since the
 news is shown at the beginning of the game, there's no point in
 setting this with the `O' command.

nudist
 Start the character with no armor (default false). Persistent.

null
 Send padding nulls to the terminal (default on). Persistent.

number\textsubscript{pad}
 Use digit keys instead of letters to move (default 0 or off).
 Valid settings are:


NetHack 3.6                   January 27, 2020





NetHack Guidebook                       58



 0 - move by letters; "yuhjklbn"
 1 - move by numbers; digit `5' acts as `G' movement prefix
 2 - like 1 but `5' works as `g' prefix instead of as `G'
 3 - by numbers using phone key layout; 123 above, 789 below
 4 - combines 3 with 2; phone layout plus MS-DOS compatibility
-1 - by letters but use `z' to go northwest, `y' to zap wands

For backward compatibility, omitting a value is the same as
specifying 1 and negating number\textsubscript{pad} is the same as specifying
\begin{enumerate}
\item (Settings 2 and 4 are for compatibility with MS-DOS or old
\end{enumerate}
PC Hack; in addition to the different behavior for `5', `Alt-5'
acts as `G' and `Alt-0' acts as `I'. Setting -1 is to accommo-
date some QWERTZ keyboards which have the location of the `y'
and `z' keys swapped.) When moving by numbers, to enter a
count prefix for those commands which accept one (such as "12s"
to search twelve times), precede it with the letter `n'
("n12s").

packorder
 Specify  the  order  to  list  object types in (default
 "")[\%?+!=/(*`0\_"). The value of this option should be a string
 containing the symbols for the various object types. Any omit-
 ted types are filled in at the end from the previous order.

paranoid\textsubscript{confirmation}
 A space separated list of specific situations where alternate
 prompting  is desired.  The default is paranoid\textsubscript{confirma}-
 tion:pray.

Confirm   - for any prompts which are set to require "yes"
       rather than `y', also require "no" to reject in-
       stead of accepting any non-yes response as no
quit    - require "yes" rather than `y' to confirm quitting
       the game or switching into non-scoring explore
       mode;
die     - require "yes" rather than `y' to confirm dying
       (not useful in normal play; applies to explore
       mode);
bones    - require "yes" rather than `y' to confirm saving
       bones data when dying in debug mode;
attack   - require "yes" rather than `y' to confirm attack-
       ing a peaceful monster;
wand-break - require "yes" rather than `y' to confirm breaking
       a wand;
eating   - require "yes" rather than `y' to confirm whether
       to continue eating;
Were-change - require "yes" rather than `y' to confirm changing
       form due to lycanthropy when hero has polymorph
       control;
pray    - require `y' to confirm an attempt to pray rather
       than immediately praying; on by default;
Remove   - require selection from inventory for `R' and `T'
       commands even when wearing just one applicable
       item.


NetHack 3.6                   January 27, 2020





NetHack Guidebook                       59



all     - turn on all of the above.

By default, the pray choice is enabled, the others disabled.
To disable it without setting any of the other choices, use
"paranoid\textsubscript{confirmation}:none". To keep it enabled while setting
any of the others, include it in the list, such as "para-
noid\textsubscript{confirmation}:attack pray Remove".

perm\textsubscript{invent}
 If true, always display your current inventory in a window.
 This only makes sense for windowing system interfaces that im-
 plement this feature.

petattr
 Specifies one or more text highlighting attributes to use when
 showing pets on the map.  Effectively a superset of the
 hilite\textsubscript{pet} boolean option. Curses interface only; value is one
 or more of the following letters.

n - Normal text (no highlighting)
i - Inverse video (default)
b - Bold text
u - Underlined text
k - blinKing text
d - Dim text
t - iTalic text
l - Left line indicator
r - Right line indicator

Some of those choices might not work, particularly the final
three, depending upon terminal hardware or terminal emulation
software.

Currently multiple highlight-style letters can be combined by
simply stringing them together (for example, "bk"), but in the
future they might require being separated by plus signs (such
as "b+k", which works already). When using the `n' choice, it
should be specified on its own, not in combination with any of
the other letters.

pettype
 Specify the type of your initial pet, if you are playing a
 character class that uses multiple types of pets; or choose to
 have no initial pet at all. Possible values are "cat", "dog",
 "horse", and "none". If the choice is not allowed for the role
 you are currently playing, it will be silently ignored. For
 example, "horse" will only be honored when playing a knight.
 Cannot be set with the `O' command.

pickup\textsubscript{burden}
 When you pick up an item that would exceed this encumbrance
 level (Unencumbered, Burdened, streSsed, straiNed, overTaxed,
 or overLoaded), you will be asked if you want to continue.
 (Default `S'). Persistent.


NetHack 3.6                   January 27, 2020





NetHack Guidebook                       60



pickup\textsubscript{thrown}
 If this option is on and autopickup is also on, try to pick up
 things that you threw, even if they aren't in pickup\textsubscript{types} or
 match an autopickup exception. Default is on. Persistent.

pickup\textsubscript{types}
 Specify the object types to be picked up when autopickup is on.
 Default is all types. You can use autopickup\textsubscript{exception} config-
 uration file lines to further refine autopickup behavior. Per-
 sistent.

pile\textsubscript{limit}
 When walking across a pile of objects on the floor, threshold
 at which the message "there are few/several/many objects here"
 is given instead of showing a popup list of those objects. A
 value of 0 means "no limit" (always list the objects); a value
 of 1 effectively means "never show the objects" since the pile
 size will always be at least that big; default value is 5.
 Persistent.

playmode
 Values are "normal", "explore", or "debug". Allows selection
 of explore mode (also known as discovery mode) or debug mode
 (also known as wizard mode) instead of normal play. Debug mode
 might only be allowed for someone logged in under a particular
 user name (on multi-user systems) or specifying a particular
 character name (on single-user systems) or it might be disabled
 entirely.  Requesting it when not allowed or not possible re-
 sults in explore mode instead. Default is normal play.

pushweapon
 Using the `w' (wield) command when already wielding something
 pushes the old item into your alternate weapon slot (default
 off). Likewise for the `a' (apply) command if it causes the
 applied item to become wielded. Persistent.

race
 Selects your race (for example, "race:human"). Default is ran-
 dom. If you prefix the value with `!' or "no", you will ex-
 clude that race from being picked randomly. Cannot be set with
 the `O' command. Persistent.

rest\textsubscript{on}\textsubscript{space}
 Make the space bar a synonym for the `.' (\#wait) command (de-
 fault off). Persistent.

role
 Pick your type of character (for example "role:Samurai"); syn-
 onym for "character". See "name" for an alternate method of
 specifying your role.  Normally only the first letter of the
 value is examined; `r' is an exception with "Rogue", "Ranger",
 and "random" values. If you prefix the value with `!' or "no",
 you will exclude that role from being picked randomly.  Cannot
 be set with the `O' command. Persistent.


NetHack 3.6                   January 27, 2020





NetHack Guidebook                       61



roguesymset
 This option may be used to select one of the named symbol sets
 found within "symbols" to alter the symbols displayed on the
 screen on the rogue level.

rlecomp
 When writing out a save file, perform run length compression of
 the map. Not all ports support run length compression. It has
 no effect on reading an existing save file.

runmode
 Controls the amount of screen updating for the map window when
 engaged in multi-turn movement (running via shift+direction or
 control+direction and so forth, or via the travel command or
 mouse click). The possible values are:

teleport - update the map after movement has finished;
run   - update the map after every seven or so steps;
walk   - update the map after each step;
crawl  - like walk, but pause briefly after each step.

This option only affects the game's screen display, not the ac-
tual results of moving. The default is "run"; versions prior
to 3.4.1 used "teleport" only. Whether or not the effect is
noticeable will depend upon the window port used or on the type
of terminal. Persistent.

safe\textsubscript{pet}
 Prevent you from (knowingly) attacking your pets (default on).
 Persistent.

sanity\textsubscript{check}
 Evaluate monsters, objects, and map prior to each turn (default
 off). Debug mode only.

scores
 Control what parts of the score list you are shown at the end
 (for  example "scores:5 top scores/4 around my score/own
 scores"). Only the first letter of each category (`t', `a', or
 `o') is necessary. Persistent.

showexp
 Show your accumulated experience points on bottom line (default
 off). Persistent.

showrace
 Display yourself as the glyph for your race, rather than the
 glyph for your role (default off). Note that this setting af-
 fects only the appearance of the display, not the way the game
 treats you. Persistent.

showscore
 Show your approximate accumulated score on bottom line (default
 off). Persistent.


NetHack 3.6                   January 27, 2020





NetHack Guidebook                       62



silent
 Suppress terminal beeps (default on). Persistent.

sortloot
 Controls the sorting behavior of the pickup lists for inventory
 and \#loot commands and some others. Persistent. The possible
 values are:

full - always sort the lists;
loot - only sort the lists that don't use inventory letters,
    like with the \#loot and pickup commands;
none - show lists the traditional way without sorting.

sortpack
 Sort the pack contents by type when displaying inventory (de-
 fault on). Persistent.

sparkle
 Display a sparkly effect when a monster (including yourself) is
 hit by an attack to which it is resistant (default on). Per-
 sistent.

standout
 Boldface monsters and "--More--" (default off). Persistent.

statushilites
 Controls how many turns status hilite behaviors highlight the
 field.  If negated or set to zero, disables status hiliting.
 See "Configuring Status Hilites" for further information.

status\textsubscript{updates}
 Allow updates to the status lines at the bottom of the screen
 (default true).

suppress\textsubscript{alert}
 This option may be set to a NetHack version level to suppress
 alert notification messages about feature changes for that and
 prior versions (for example "suppress\textsubscript{alert}:3.3.1").

symset
 This option may be used to select one of the named symbol sets
 found within "symbols" to alter the symbols displayed on the
 screen.  Use "symset:default" to explicitly select the default
 symbols.

time
 Show the elapsed game time in turns on bottom line (default
 off). Persistent.

timed\textsubscript{delay}
 When pausing momentarily for display effect, such as with ex-
 plosions and moving objects, use a timer rather than sending
 extra characters to the screen. (Applies to "tty" interface
 only; "X11" interface always uses a timer based delay.  The


NetHack 3.6                   January 27, 2020





NetHack Guidebook                       63



default is on if configured into the program.) Persistent.

tombstone
 Draw a tombstone graphic upon your death (default on). Persis-
 tent.

toptenwin
 Put the ending display in a NetHack window instead of on stdout
 (default off). Setting this option makes the score list visi-
 ble when a windowing version of NetHack is started without a
 parent window, but it no longer leaves the score list around
 after game end on a terminal or emulating window.

travel
 Allow the travel command (default on). Turning this option off
 will prevent the game from attempting unintended moves if you
 make inadvertent mouse clicks on the map window. Persistent.

verbose
 Provide more commentary during the game (default on).  Persis-
 tent.

whatis\textsubscript{coord}
 When using the `/' or `;' commands to look around on the map
 with autodescribe on, display coordinates after the descrip-
 tion.  Also works in other situations where you are asked to
 pick a location.

The possible settings are:

c - compass ("east" or "3s" or "2n,4w");
f - full compass ("east" or "3south" or "2north,4west");
m - map <x,y> (map column x=0 is not used);
s - screen [row,column] (row is offset to match tty usage);
n - none (no coordinates shown) [default].

The whatis\textsubscript{coord} option is also used with the "\emph{m", "/M", "/o",
and "/O" sub-commands of `}', where the "none" setting is over-
ridden with "map".

whatis\textsubscript{filter}
 When getting a location on the map, and using the keys to cycle
 through next and previous targets, allows filtering the possi-
 ble targets.

n - no filtering [default]
v - in view only
a - in same area only

The area-filter tries to be slightly predictive -- if you're
standing on a doorway, it will consider the area on the side of
the door you were last moving towards.




NetHack 3.6                   January 27, 2020





NetHack Guidebook                       64



Filtering can also be changed when getting a location with the
"getpos.filter" key.

whatis\textsubscript{menu}
 When getting a location on the map, and using a key to cycle
 through next and previous targets, use a menu instead to pick a
 target. (default off)

whatis\textsubscript{moveskip}
 When getting a location on the map, and using shifted movement
 keys or meta-digit keys to fast-move, instead of moving 8 units
 at a time, move by skipping the same glyphs. (default off)

windowtype
 When the program has been built to support multiple interfaces,
 select which one to use, such as "tty" or "X11" (default de-
 pends on build-time settings; use "\#version" to check). Cannot
 be set with the `O' command.

When used, it should be the first option set since its value
might enable or disable the availability of various other op-
tions. For multiple lines in a configuration file, that would
be the first non-comment line. For a comma-separated list in
NETHACKOPTIONS or an OPTIONS line in a configuration file, that
would be the rightmost option in the list.

wizweight
 Augment object descriptions with their objects' weight (default
 off). Debug mode only.

zerocomp
 When writing out a save file, perform zero-comp compression of
 the contents. Not all ports support zero-comp compression. It
 has no effect on reading an existing save file.

9.5. Window Port Customization options

   Here are explanations of the various options that are used
to customize and change the characteristics of the windowtype
that you have chosen. Character strings that are too long may be
truncated.  Not all window ports will adjust for all settings
listed here. You can safely add any of these options to your
configuration file, and if the window port is capable of adjust-
ing to suit your preferences, it will attempt to do so. If it
can't it will silently ignore it. You can find out if an option
is supported by the window port that you are currently using by
checking to see if it shows up in the Options list. Some options
are dynamic and can be specified during the game with the `O'
command.

align\textsubscript{message}
 Where to align or place the message window (top, bottom, left,
 or right)



NetHack 3.6                   January 27, 2020





NetHack Guidebook                       65



align\textsubscript{status}
 Where to align or place the status window (top, bottom, left,
 or right).

ascii\textsubscript{map}
 If NetHack can, it should display an ascii character map if it
 can.

color
 If NetHack can, it should display color if it can for different
 monsters, objects, and dungeon features.

eight\textsubscript{bit}\textsubscript{tty}
 If NetHack can, it should pass eight-bit character values (for
 example, specified with the traps option) straight through to
 your terminal (default off).

font\textsubscript{map}
 if NetHack can, it should use a font by the chosen name for the
 map window.

font\textsubscript{menu}
 If NetHack can, it should use a font by the chosen name for
 menu windows.

font\textsubscript{message}
 If NetHack can, it should use a font by the chosen name for the
 message window.

font\textsubscript{status}
 If NetHack can, it should use a font by the chosen name for the
 status window.

font\textsubscript{text}
 If NetHack can, it should use a font by the chosen name for
 text windows.

font\textsubscript{size}\textsubscript{map}
 If NetHack can, it should use this size font for the map win-
 dow.

font\textsubscript{size}\textsubscript{menu}
 If NetHack can, it should use this size font for menu windows.

font\textsubscript{size}\textsubscript{message}
 If NetHack can, it should use this size font for the message
 window.

font\textsubscript{size}\textsubscript{status}
 If NetHack can, it should use this size font for the status
 window.

font\textsubscript{size}\textsubscript{text}
 If NetHack can, it should use this size font for text windows.


NetHack 3.6                   January 27, 2020





NetHack Guidebook                       66



fullscreen
 If NetHack can, it should try and display on the entire screen
 rather than in a window.

guicolor
 Use color text and/or highlighting attributes when displaying
 some non-map data (such as menu selector letters). Curses in-
 terface only; default is on.

large\textsubscript{font}
 If NetHack can, it should use a large font.

map\textsubscript{mode}
 If NetHack can, it should display the map in the manner speci-
 fied.

player\textsubscript{selection}
 If NetHack can, it should pop up dialog boxes, or use prompts
 for character selection.

popup\textsubscript{dialog}
 If NetHack can, it should pop up dialog boxes for input.

preload\textsubscript{tiles}
 If NetHack can, it should preload tiles into memory. For exam-
 ple, in the protected mode MS-DOS version, control whether
 tiles get pre-loaded into RAM at the start of the game.  Doing
 so enhances performance of the tile graphics, but uses more
 memory. (default on). Cannot be set with the `O' command.

scroll\textsubscript{amount}
 If NetHack can, it should scroll the display by this number of
 cells when the hero reaches the scroll\textsubscript{margin}.

scroll\textsubscript{margin}
 If NetHack can, it should scroll the display when the hero or
 cursor is this number of cells away from the edge of the win-
 dow.

selectsaved
 If NetHack can, it should display a menu of existing saved
 games for the player to choose from at game startup, if it can.
 Not all ports support this option.

softkeyboard
 Display an onscreen keyboard.  Handhelds are most likely to
 support this option.

splash\textsubscript{screen}
 If NetHack can, it should display an opening splash screen when
 it starts up (default yes).

statuslines
 Number of lines for traditional below-the-map status display.


NetHack 3.6                   January 27, 2020





NetHack Guidebook                       67



Acceptable values are 2 and 3 (default is 2). Curses and tty
interfaces only.

term\textsubscript{cols} and

term\textsubscript{rows}
 Curses interface only. Number of columns and rows to use for
 the display. Curses will attempt to resize to the values spec-
 ified but will settle for smaller sizes if they are too big.
 Default is the current window size.

tiled\textsubscript{map}
 If NetHack can, it should display a tiled map if it can.

tile\textsubscript{file}
 Specify the name of an alternative tile file to override the
 default.

tile\textsubscript{height}
 Specify the preferred height of each tile in a tile capable
 port.

tile\textsubscript{width}
 Specify the preferred width of each tile in a tile capable port

use\textsubscript{darkgray}
 Use bold black instead of blue for black glyphs (TTY only).

use\textsubscript{inverse}
 If NetHack can, it should display inverse when the game speci-
 fies it.

vary\textsubscript{msgcount}
 If NetHack can, it should display this number of messages at a
 time in the message window.

windowborders
 Whether to draw boxes around the map, status area, message
 area, and persistent inventory window if enabled. Curses in-
 terface only. Acceptable values are

0 - off, never show borders
1 - on, always show borders
2 - auto, on if display is at least (24+2)x(80+2) (default)

(The 26x82 size threshold for `2' refers to number of rows and
columns of the display.  A width of at least 110 columns
(80+2+26+2) is needed for align\textsubscript{status} set to left or right.)

windowcolors
 If NetHack can, it should display windows with the specified
 foreground/background colors. Windows GUI only. The format is




NetHack 3.6                   January 27, 2020





NetHack Guidebook                       68



OPTION=windowcolors:wintype foreground/background

   where wintype is one of "menu", "message", "status", or
"text", and foreground and background are colors, either a hexa-
decimal $\backslash$'\#rrggbb', one of the named colors (black, red, green,
brown, blue, magenta, cyan, orange, brightgreen, yellow, bright-
blue, brightmagenta, brightcyan, white, trueblack, gray, purple,
silver, maroon, fuchsia, lime, olive, navy, teal, aqua), or one
of Windows UI colors (activeborder, activecaption, appworkspace,
background, btnface, btnshadow, btntext, captiontext, graytext,
greytext, highlight, highlighttext, inactiveborder, inactivecap-
tion, menu, menutext, scrollbar, window, windowframe, window-
text).

wraptext
 If NetHack can, it should wrap long lines of text if they don't
 fit in the visible area of the window.

9.6. Platform-specific Customization options

   Here are explanations of options that are used by specific
platforms or ports to customize and change the port behavior.

altkeyhandler
 Select an alternate keystroke handler dll to load (Win32 tty
 NetHack only). The name of the handler is specified without
 the .dll extension and without any path information. Cannot be
 set with the `O' command.

altmeta
 On Amiga, this option controls whether typing "Alt" plus anoth-
 er key functions as a meta-shift for that key (default on).

altmeta
 On other (non-Amiga) systems where this option is available, it
 can be set to tell NetHack to convert a two character sequence
 beginning with ESC into a meta-shifted version of the second
 character (default off).

This conversion is only done for commands, not for other input
prompts. Note that typing one or more digits as a count prefix
prior to a command -- preceded by n if the number\textsubscript{pad} option is
set -- is also subject to this conversion, so attempting to
abort the count by typing ESC will leave NetHack waiting for
another character to complete the two character sequence. Type
a second ESC to finish cancelling such a count.  At other
prompts a single ESC suffices.

BIOS
 Use BIOS calls to update the screen display quickly and to read
 the keyboard (allowing the use of arrow keys to move) on ma-
 chines with an IBM PC compatible BIOS ROM (default off, OS/2,
 PC, and ST NetHack only).



NetHack 3.6                   January 27, 2020





NetHack Guidebook                       69



flush
 (default off, Amiga NetHack only).

MACgraphics
 (default on, Mac NetHack only).

page\textsubscript{wait}
 (default on, Mac NetHack only).

rawio
 Force raw (non-cbreak) mode for faster output and more bullet-
 proof input (MS-DOS sometimes treats `\textsuperscript{P}' as a printer toggle
 without it) (default off, OS/2, PC, and ST NetHack only).
 Note:  DEC Rainbows hang if this is turned on. Cannot be set
 with the `O' command.

soundcard
 (default on, PC NetHack only). Cannot be set with the `O' com-
 mand.

subkeyvalue
 (Win32 tty NetHack only). May be used to alter the value of
 keystrokes that the operating system returns to NetHack to help
 compensate for international keyboard issues. OPTIONS=subkey-
 value:171/92 will return 92 to NetHack, if 171 was originally
 going to be returned. You can use multiple subkeyvalue state-
 ments in the configuration file if needed. Cannot be set with
 the `O' command.

video
 Set the video mode used (PC NetHack only). Values are "autode-
 tect", "default", or "vga".  Setting "vga" (or "autodetect"
 with vga hardware present) will cause the game to display
 tiles. Cannot be set with the `O' command.

videocolors
 Set the color palette for PC systems using NO\textsubscript{TERMS} (default
 4-2-6-1-5-3-15-12-10-14-9-13-11, (PC NetHack only). The order
 of colors is red, green, brown,  blue,  magenta,  cyan,
 bright.white, bright.red, bright.green, yellow, bright.blue,
 bright.magenta, and bright.cyan. Cannot be set with the `O'
 command.

videoshades
 Set the intensity level of the three gray scales available (de-
 fault dark normal light, PC NetHack only). If the game display
 is difficult to read, try adjusting these scales; if this does
 not correct the problem, try !color. Cannot be set with the
 `O' command.

9.7. Regular Expressions

   Regular expressions are normally POSIX extended regular ex-
pressions. It is possible to compile NetHack without regular


NetHack 3.6                   January 27, 2020





NetHack Guidebook                       70



expression support on a platform where there is no regular ex-
pression library. While this is not true of any modern platform,
if your NetHack was built this way, patterns are instead glob
patterns. This applies to Autopickup exceptions, Message types,
Menu colors, and User sounds.

9.8. Configuring Autopickup Exceptions

   You can further refine the behavior of the autopickup option
beyond what is available through the pickup\textsubscript{types} option.

   By placing autopickup\textsubscript{exception} lines in your configuration
file, you can define patterns to be checked when the game is
about to autopickup something.

autopickup\textsubscript{exception}
 Sets an exception to the pickup\textsubscript{types} option.  The autopick-
 up\textsubscript{exception} option should be followed by a regular expression
 to be used as a pattern to match against the singular form of
 the description of an object at your location.

In addition, some characters are treated specially if they oc-
cur as the first character in the pattern, specifically:

< - always pickup an object that matches rest of pattern;
> - never pickup an object that matches rest of pattern.

The autopickup\textsubscript{exception} rules are processed in the order in
which they appear in your configuration file, thus allowing a
later rule to override an earlier rule.

Exceptions can be set with the `O' command, but because they
are not included in your configuration file, they won't be in
effect if you save and then restore your game.  autopickup\textsubscript{ex}-
ception rules and not saved with the game.

Here are some examples:

autopickup\textsubscript{exception}="<*arrow"
autopickup\textsubscript{exception}=">*corpse"
autopickup\textsubscript{exception}=">* cursed*"

   The first example above will result in autopickup of any
type of arrow. The second example results in the exclusion of
any corpse from autopickup. The last example results in the ex-
clusion of items known to be cursed from autopickup.

9.9. Changing Key Bindings

   It is possible to change the default key bindings of some
special commands, menu accelerator keys, and extended commands,
by using BIND stanzas in the configuration file. Format is key,
followed by the command to bind to, separated by a colon. The
key can be a single character ("x"), a control key ("\textsuperscript{X}", "C-x"),


NetHack 3.6                   January 27, 2020





NetHack Guidebook                       71



a meta key ("M-x"), or a three-digit decimal ASCII code.

For example:

BIND=\textsuperscript{X}:getpos.autodescribe
BIND=\{:menu\textsubscript{first}\textsubscript{page}
BIND=v:loot

Extended command keys
 You can bind multiple keys to the same extended command. Un-
 bind a key by using "nothing" as the extended command to bind
 to.  You can also bind the "<esc>", "<enter>", and "<space>"
 keys.

Menu accelerator keys
 The menu control or accelerator keys can also be rebound via
 OPTIONS lines in the configuration file. You cannot bind ob-
 ject symbols into menu accelerators.

Special command keys
 Below are the special commands you can rebind.  Some of them
 can be bound to same keys with no problems, others are in the
 same "context", and if bound to same keys, only one of those
 commands will be available. Special command can only be bound
 to a single key.

count
 Prefix key to start a count, to repeat a command this many
 times. With number\textsubscript{pad} only. Default is `n'.

doinv
 Show inventory. With number\textsubscript{pad} only. Default is `0'.

fight
 Prefix key to force fight a direction. Default is `F'.

fight.numpad
 Prefix key to force fight a direction. With number\textsubscript{pad} only.
 Default is `-'.

getdir.help
 When asked for a direction, the key to show the help.  Default
 is `?'.

getdir.self
 When asked for a direction, the key to target yourself. De-
 fault is `.'.

getdir.self2
 When asked for a direction, the key to target yourself.  De-
 fault is `s'.

getpos.autodescribe
 When asked for a location, the key to toggle autodescribe.


NetHack 3.6                   January 27, 2020




NetHack Guidebook                       72



Default is `\#'.

getpos.all.next
 When asked for a location, the key to go to next closest inter-
 esting thing. Default is `a'.

getpos.all.prev
 When asked for a location, the key to go to previous closest
 interesting thing. Default is `A'.

getpos.door.next
 When asked for a location, the key to go to next closest door
 or doorway. Default is `d'.

getpos.door.prev
 When asked for a location, the key to go to previous closest
 door or doorway. Default is `D'.

getpos.help
 When asked for a location, the key to show help.  Default is
 `?'.

getpos.mon.next
 When asked for a location, the key to go to next closest mon-
 ster. Default is `m'.

getpos.mon.prev
 When asked for a location, the key to go to previous closest
 monster. Default is `M'.

getpos.obj.next
 When asked for a location, the key to go to next closest ob-
 ject. Default is `o'.

getpos.obj.prev
 When asked for a location, the key to go to previous closest
 object. Default is `O'.

getpos.menu
 When asked for a location, and using one of the next or previ-
 ous keys to cycle through targets, toggle showing a menu in-
 stead. Default is `!'.

getpos.moveskip
 When asked for a location, and using the shifted movement keys
 or meta-digit keys to fast-move around, move by skipping the
 same glyphs instead of by 8 units. Default is `*'.

getpos.filter
 When asked for a location, change the filtering mode when using
 one of the next or previous keys to cycle through targets.
 Toggles between no filtering, in view only, and in the same
 area only. Default is `"'.



NetHack 3.6                   January 27, 2020





NetHack Guidebook                       73



getpos.pick
 When asked for a location, the key to choose the location, and
 possibly ask for more info. Default is `.'.

getpos.pick.once
 When asked for a location, the key to choose the location, and
 skip asking for more info. Default is `,'.

getpos.pick.quick
 When asked for a location, the key to choose the location, skip
 asking for more info, and exit the location asking loop. De-
 fault is `;'.

getpos.pick.verbose
 When asked for a location, the key to choose the location, and
 show more info without asking. Default is `:'.

getpos.self
 When asked for a location, the key to go to your location. De-
 fault is `@'.

getpos.unexplored.next
 When asked for a location, the key to go to next closest unex-
 plored location. Default is `x'.

getpos.unexplored.prev
 When asked for a location, the key to go to previous closest
 unexplored location. Default is `X'.

getpos.valid
 When asked for a location, the key to go to show valid target
 locations. Default is `\$'.

getpos.valid.next
 When asked for a location, the key to go to next closest valid
 location. Default is `z'.

getpos.valid.prev
 When asked for a location, the key to go to previous closest
 valid location. Default is `Z'.

nopickup
 Prefix key to move without picking up items. Default is `m'.

redraw
 Key to redraw the screen. Default is `\textsuperscript{R}'.

redraw.numpad
 Key to redraw the screen. With number\textsubscript{pad} only. Default is
 `\textsuperscript{L}'.

repeat
 Key to repeat previous command. Default is `\textsuperscript{A}'.



NetHack 3.6                   January 27, 2020





NetHack Guidebook                       74



reqmenu
 Prefix key to request menu from some commands. Default is `m'.

run
 Prefix key to run towards a direction. Default is `G'.

run.nopickup
 Prefix key to run towards a direction without picking up items
 on the way. Default is `M'.

run.numpad
 Prefix key to run towards a direction. With number\textsubscript{pad} only.
 Default is `5'.

rush
 Prefix key to rush towards a direction. Default is `g'.

9.10. Configuring Message Types

   You can change the way the messages are shown in the message
area, when the message matches a user-defined pattern.

   In general, the configuration file entries to describe the
message types look like this: MSGTYPE=type "pattern"

type  - how the message should be shown;
pattern - the pattern to match.

The pattern should be a regular expression.

Allowed types are:

show - show message normally;
hide - never show the message;
stop - wait for user with more-prompt;
norep - show the message once, but not again if no other mes-
    sage is shown in between.

Here's an example of message types using NetHack's internal
pattern matching facility:

MSGTYPE=stop "You feel hungry."
MSGTYPE=hide "You displaced *."

specifies that whenever a message "You feel hungry" is shown,
the user is prompted with more-prompt, and a message matching
"You displaced <something>." is not shown at all.

The order of the defined MSGTYPE lines is important; the last
matching rule is used. Put the general case first, exceptions
below them.





NetHack 3.6                   January 27, 2020





NetHack Guidebook                       75



9.11. Configuring Menu Colors

   Some platforms allow you to define colors used in menu lines
when the line matches a user-defined pattern. At this time the
tty, curses, win32tty and win32gui interfaces support this.

   In general, the configuration file entries to describe the
menu color mappings look like this:

MENUCOLOR="pattern"=color\&attribute

pattern  - the pattern to match;
color   - the color to use for lines matching the pat-
      tern;
attribute - the attribute to use for lines matching the
      pattern. The attribute is optional, and if
      left out, you must also leave out the preced-
      ing ampersand.  If no attribute is defined,
      no attribute is used.

The pattern should be a regular expression.

Allowed colors are black, red, green, brown, blue, magenta,
cyan, gray, orange, light-green, yellow, light-blue, light-ma-
genta, light-cyan, and white. And no-color, the default fore-
ground color, which isn't necessarily the same as any of the
other colors.

Allowed attributes are none, bold, dim, underline, blink, and
inverse.  "Normal" is a synonym for "none". Note that the
platform used may interpret the attributes any way it wants.

Here's an example of menu colors using NetHack's internal pat-
tern matching facility:

MENUCOLOR="* blessed \textbf{"\texttt{green
         MENUCOLOR}"} cursed \textbf{"\texttt{red
         MENUCOLOR}"} cursed *(being worn)"=red\&underline

specifies that any menu line with " blessed " contained in it
will be shown in green color, lines with " cursed " will be
shown in red, and lines with " cursed " followed by "(being
worn)" on the same line will be shown in red color and under-
lined. You can have multiple MENUCOLOR entries in your config-
uration file, and the last MENUCOLOR line that matches a menu
line will be used for the line.

   Note that if you intend to have one or more color specifica-
tions match " uncursed ", you will probably want to turn the im-
plicit\textsubscript{uncursed} option off so that all items known to be uncursed
are actually displayed with the "uncursed" description.





NetHack 3.6                   January 27, 2020





NetHack Guidebook                       76



9.12. Configuring User Sounds

   Some platforms allow you to define sound files to be played
when a message that matches a user-defined pattern is delivered
to the message window. At this time the Qt port and the win32tty
and win32gui ports support the use of user sounds.

   The following configuration file entries are relevant to
mapping user sounds to messages:

SOUNDDIR
 The directory that houses the sound files to be played.

SOUND
 An entry that maps a sound file to a user-specified message
 pattern.  Each SOUND entry is broken down into the following
 parts:

MESG    - message window mapping (the only one supported in
       3.6);
pattern  - the pattern to match;
sound file - the sound file to play;
volume   - the volume to be set while playing the sound file.

The pattern should be a POSIX extended regular expression.

9.13. Configuring Status Hilites

   Your copy of NetHack may have been compiled with support for
"Status Hilites". If so, you can customize your game display by
setting thresholds to change the color or appearance of fields in
the status display.

The format for defining status colors is:

OPTION=hilite\textsubscript{status}:field-name/behavior/color\&attributes

   For example, the following line in your configuration file
will cause the hitpoints field to display in the color red if
your hitpoints drop to or below a threshold of 30\%:

OPTION=hilite\textsubscript{status}:hitpoints/<=30\%/red/normal

(That example is actually specifying red\&normal for <=30\% and no-
color\&normal for >30\%.)

   For another example, the following line in your configura-
tion file will cause wisdom to be displayed red if it drops and
green if it rises:

OPTION=hilite\textsubscript{status}:wisdom/down/red/up/green

   Allowed colors are black, red, green, brown, blue, magenta,
cyan, gray, orange, light-green, yellow, light-blue, light-


NetHack 3.6                   January 27, 2020





NetHack Guidebook                       77



magenta, light-cyan, and white.  And "no-color", the default
foreground color on the display, which is not necessarily the
same as black or white or any of the other colors.

   Allowed attributes are none, bold, dim, underline, blink,
and inverse. "Normal" is a synonym for "none"; they should not
be used in combination with any of the other attributes.

   To specify both a color and an attribute, use `\&' to combine
them. To specify multiple attributes, use `+' to combine those.
For example: "magenta\&inverse+dim".

   Note that the display may substitute or ignore particular
attributes depending upon its capabilities, and in general may
interpret the attributes any way it wants. For example, on some
display systems a request for bold might yield blink or vice ver-
sa. On others, issuing an attribute request while another is al-
ready set up will replace the earlier attribute rather than com-
bine with it.  Since NetHack issues attribute requests sequen-
tially (at least with the tty interface) rather than all at once,
the only way a situation like that can be controlled is to speci-
fy just one attribute.

   You can adjust the appearance of the following status
fields:
      title    dungeon-level  experience-level
     strength     gold      experience
     dexterity    hitpoints      HD
    constitution  hitpoints-max     time
    intelligence    power      hunger
      wisdom    power-max   carrying-capacity
     charisma   armor-class    condition
     alignment              score

The pseudo-field "characteristics" can be used to set all six
of Str, Dex, Con, Int, Wis, and Cha at once.  "HD" is "hit
dice", an approximation of experience level displayed when
polymorphed. "experience", "time", and "score" are condition-
ally displayed depending upon your other option settings.

Instead of a behavior, "condition" takes the following condi-
tion flags: stone, slime, strngl, foodpois, termill, blind,
deaf, stun, conf, hallu, lev, fly, and ride. You can use "ma-
jor\textsubscript{troubles}" as an alias for stone through termill, "mi-
nor\textsubscript{troubles}" for blind through hallu, "movement" for lev, fly,
and ride, and "all" for every condition.

Allowed behaviors are "always", "up", "down", "changed", a per-
centage or absolute number threshold, or text to match against.

\begin{itemize}
\item "always" will set the default attributes for that field.

\item "up", "down" set the field attributes for when the field
value changes upwards or downwards. This attribute times
\end{itemize}


NetHack 3.6                   January 27, 2020





NetHack Guidebook                       78



out after statushilites turns.

\begin{itemize}
\item "changed" sets the field attribute for when the field val-
ue changes. This attribute times out after statushilites
turns.  (If a field has both a "changed" rule and an "up"
or "down" rule which matches a change in the field's val-
ue, the "up" or "down" one takes precedence.)

\item percentage sets the field attribute when the field value
matches the percentage. It is specified as a number be-
tween 0 and 100, followed by `\%' (percent sign). If the
percentage is prefixed with `<=' or `>=', it also matches
when value is below or above the percentage. Use prefix
`<' or `>' to match when strictly below or above.  (The
numeric limit is relaxed slightly for those: >-1\% and
<101\% are allowed.) Only four fields support percentage
rules.  Percentages for "hitpoints" and "power" are
straightforward; they're based on the corresponding maxi-
mum field.  Percentage highlight rules are also allowed
for "experience level" and "experience points" (valid when
the showexp option is enabled). For those, the percentage
is based on the progress from the start of the current ex-
perience level to the start of the next level. So if lev-
el 2 starts at 20 points and level 3 starts at 40 points,
having 30 points is 50\% and 35 points is 75\%. 100\% is
unattainable for experience because you'll gain a level
and the calculations will be reset for that new level, but
a rule for =100\% is allowed and matches the special case
of being exactly 1 experience point short of the next lev-
el.

\item absolute value sets the attribute when the field value
matches that number. The number must be 0 or higher, ex-
cept for "armor-class' which allows negative values, and
may optionally be preceded by `='. If the number is pre-
ceded by `<=' or `>=' instead, it also matches when value
is below or above.  If the prefix is `<' or `>', only
match when strictly above or below.

\item text match sets the attribute when the field value matches
the text. Text matches can only be used for "alignment",
"carrying-capacity", "hunger", "dungeon-level", and "ti-
tle".  For title, only the role's rank title is tested;
the character's name is ignored.
\end{itemize}

   The in-game options menu can help you determine the correct
syntax for a configuration file.

   The whole feature can be disabled by setting option sta-
tushilites to 0.

Example hilites:




NetHack 3.6                   January 27, 2020





NetHack Guidebook                       79



OPTION=hilite\textsubscript{status}: gold/up/yellow/down/brown
OPTION=hilite\textsubscript{status}: characteristics/up/green/down/red
OPTION=hilite\textsubscript{status}: hitpoints/100\%/gray\&normal
OPTION=hilite\textsubscript{status}: hitpoints/<100\%/green\&normal
OPTION=hilite\textsubscript{status}: hitpoints/<66\%/yellow\&normal
OPTION=hilite\textsubscript{status}: hitpoints/<50\%/orange\&normal
OPTION=hilite\textsubscript{status}: hitpoints/<33\%/red\&bold
OPTION=hilite\textsubscript{status}: hitpoints/<15\%/red\&inverse
OPTION=hilite\textsubscript{status}: condition/major/orange\&inverse
OPTION=hilite\textsubscript{status}: condition/lev+fly/red\&inverse

9.14. Modifying NetHack Symbols

NetHack can load entire symbol sets from the symbol file.

   The options that are used to select a particular symbol set
from the symbol file are:

symset
 Set the name of the symbol set that you want to load.

roguesymset
 Set the name of the symbol set that you want to load for dis-
 play on the rogue level.

   You can also override one or more symbols using the SYMBOLS
and ROGUESYMBOLS configuration file options. Symbols are speci-
fied as name:value pairs. Note that NetHack escape-processes the
value string in conventional C fashion. This means that $\backslash$ is a
prefix to take the following character literally. Thus $\backslash$ needs
to be represented as $\backslash$\. The special prefix form \m switches on
the meta bit in the symbol value, and the \^{} prefix causes the
following character to be treated as a control character.

NetHack Symbols
  Symbol Name      Description

\noindent\rule{\textwidth}{0.5pt}
  S\textsubscript{air}         (air)
\_ S\textsubscript{altar}        (altar)
" S\textsubscript{amulet}        (amulet)
A S\textsubscript{angel}        (angelic being)
a S\textsubscript{ant}         (ant or other insect)
\^{} S\textsubscript{anti}\textsubscript{magic}\textsubscript{trap}   (anti-magic field)
[ S\textsubscript{armor}        (suit or piece of armor)
[ S\textsubscript{armour}        (suit or piece of armor)
\^{} S\textsubscript{arrow}\textsubscript{trap}      (arrow trap)
0 S\textsubscript{ball}         (iron ball)

B S\textsubscript{bat}         (bat or bird)
\^{} S\textsubscript{bear}\textsubscript{trap}      (bear trap)
\begin{itemize}
\item S\textsubscript{blcorn}        (bottom left corner)
\end{itemize}
b S\textsubscript{blob}         (blob)
\begin{itemize}
\item S\textsubscript{book}         (spellbook)
\end{itemize}



NetHack 3.6                   January 27, 2020





NetHack Guidebook                       80



) S\textsubscript{boomleft}       (boomerang open left)
( S\textsubscript{boomright}      (boomerang open right)
` S\textsubscript{boulder}       (boulder)
\begin{itemize}
\item S\textsubscript{brcorn}        (bottom right corner)
\end{itemize}
C S\textsubscript{centaur}       (centaur)
\_ S\textsubscript{chain}        (iron chain)

c S\textsubscript{cockatrice}      (cockatrice)
\$ S\textsubscript{coin}         (pile of coins)

\begin{itemize}
\item S\textsubscript{crwall}        (wall)
\end{itemize}

\^{} S\textsubscript{dart}\textsubscript{trap}      (dart trap)
\& S\textsubscript{demon}        (major demon)
\begin{itemize}
\item S\textsubscript{digbeam}       (dig beam)
\end{itemize}
> S\textsubscript{dnladder}       (ladder down)
> S\textsubscript{dnstair}       (staircase down)
d S\textsubscript{dog}         (dog or other canine)
D S\textsubscript{dragon}        (dragon)
; S\textsubscript{eel}         (sea monster)
E S\textsubscript{elemental}      (elemental)
/ S\textsubscript{explode1}       (explosion top left)
\begin{itemize}
\item S\textsubscript{explode2}       (explosion top center)
\end{itemize}
$\backslash$ S\textsubscript{explode3}       (explosion top right)
\begin{center}
\begin{tabular}{l}
S\textsubscript{explode4}       (explosion middle left)\\
\end{tabular}
\end{center}
S\textsubscript{explode5}       (explosion middle center)
\begin{center}
\begin{tabular}{l}
S\textsubscript{explode6}       (explosion middle right)\\
\end{tabular}
\end{center}
$\backslash$ S\textsubscript{explode7}       (explosion bottom left)
\begin{itemize}
\item S\textsubscript{explode8}       (explosion bottom center)
\end{itemize}
/ S\textsubscript{explode9}       (explosion bottom right)
e S\textsubscript{eye}         (eye or sphere)
\^{} S\textsubscript{falling}\textsubscript{rock}\textsubscript{trap}  (falling rock trap)
f S\textsubscript{feline}        (cat or other feline)
\^{} S\textsubscript{fire}\textsubscript{trap}      (fire trap)
! S\textsubscript{flashbeam}      (flash beam)
\% S\textsubscript{food}         (piece of food)
\{ S\textsubscript{fountain}       (fountain)
F S\textsubscript{fungus}        (fungus or mold)
\begin{itemize}
\item S\textsubscript{gem}         (gem or rock)
S\textsubscript{ghost}        (ghost)
\end{itemize}
H S\textsubscript{giant}        (giant humanoid)
G S\textsubscript{gnome}        (gnome)
' S\textsubscript{golem}        (golem)
\begin{center}
\begin{tabular}{l}
S\textsubscript{grave}        (grave)\\
\end{tabular}
\end{center}
g S\textsubscript{gremlin}       (gremlin)
\begin{itemize}
\item S\textsubscript{hbeam}        (horizontal beam [zap animation])
\end{itemize}

\begin{itemize}
\item S\textsubscript{hcdoor}        (closed door in horizontal wall)
\end{itemize}
. S\textsubscript{hodbridge}      (horizontal lowered drawbridge)
\begin{center}
\begin{tabular}{l}
S\textsubscript{hodoor}        (open door in horizontal wall)\\
\end{tabular}
\end{center}
\^{} S\textsubscript{hole}         (hole)
@ S\textsubscript{human}        (human or elf)
h S\textsubscript{humanoid}       (humanoid)



NetHack 3.6                   January 27, 2020





NetHack Guidebook                       81



\begin{itemize}
\item S\textsubscript{hwall}        (horizontal wall)
\end{itemize}
. S\textsubscript{ice}         (ice)
i S\textsubscript{imp}         (imp or minor demon)
I S\textsubscript{invisible}      (invisible monster)
J S\textsubscript{jabberwock}      (jabberwock)
j S\textsubscript{jelly}        (jelly)
k S\textsubscript{kobold}        (kobold)
K S\textsubscript{kop}         (Keystone Kop)
\^{} S\textsubscript{land}\textsubscript{mine}      (land mine)
\} S\textsubscript{lava}         (molten lava)
l S\textsubscript{leprechaun}      (leprechaun)
\^{} S\textsubscript{level}\textsubscript{teleporter}   (level teleporter)
L S\textsubscript{lich}         (lich)
y S\textsubscript{light}        (light)

\begin{verbatim}
S_lizard        (lizard)
\end{verbatim}

$\backslash$ S\textsubscript{lslant}        (diagonal beam [zap animation])
\^{} S\textsubscript{magic}\textsubscript{portal}     (magic portal)
\^{} S\textsubscript{magic}\textsubscript{trap}      (magic trap)
m S\textsubscript{mimic}        (mimic)
] S\textsubscript{mimic}\textsubscript{def}      (mimic)
M S\textsubscript{mummy}        (mummy)
N S\textsubscript{naga}         (naga)
. S\textsubscript{ndoor}        (doorway without door)
n S\textsubscript{nymph}        (nymph)
O S\textsubscript{ogre}         (ogre)
o S\textsubscript{orc}         (orc)
p S\textsubscript{piercer}       (piercer)
\^{} S\textsubscript{pit}         (pit)

\^{} S\textsubscript{polymorph}\textsubscript{trap}    (polymorph trap)
\} S\textsubscript{pool}         (water)
! S\textsubscript{potion}        (potion)
P S\textsubscript{pudding}       (pudding or ooze)
q S\textsubscript{quadruped}      (quadruped)
Q S\textsubscript{quantmech}      (quantum mechanic)
= S\textsubscript{ring}         (ring)
` S\textsubscript{rock}         (boulder or statue)
r S\textsubscript{rodent}        (rodent)
\^{} S\textsubscript{rolling}\textsubscript{boulder}\textsubscript{trap} (rolling boulder trap)
. S\textsubscript{room}         (floor of a room)
/ S\textsubscript{rslant}        (diagonal beam [zap animation])
\^{} S\textsubscript{rust}\textsubscript{trap}      (rust trap)
R S\textsubscript{rustmonst}      (rust monster or disenchanter)
? S\textsubscript{scroll}        (scroll)

\^{} S\textsubscript{sleeping}\textsubscript{gas}\textsubscript{trap}  (sleeping gas trap)
S S\textsubscript{snake}        (snake)
s S\textsubscript{spider}        (arachnid or centipede)
\^{} S\textsubscript{spiked}\textsubscript{pit}      (spiked pit)
\^{} S\textsubscript{squeaky}\textsubscript{board}    (squeaky board)
0 S\textsubscript{ss1}         (magic shield 1 of 4)



NetHack 3.6                   January 27, 2020





NetHack Guidebook                       82



@ S\textsubscript{ss3}         (magic shield 3 of 4)
\begin{itemize}
\item S\textsubscript{ss4}         (magic shield 4 of 4)
\end{itemize}
\^{} S\textsubscript{statue}\textsubscript{trap}     (statue trap)
  S\textsubscript{stone}        (solid rock or unexplored terrain
	      or dark part of a room)
] S\textsubscript{strange}\textsubscript{obj}     (strange object)
\begin{itemize}
\item S\textsubscript{sw}\textsubscript{bc}        (swallow bottom center)
\end{itemize}
$\backslash$ S\textsubscript{sw}\textsubscript{bl}        (swallow bottom left)
/ S\textsubscript{sw}\textsubscript{br}        (swallow bottom right)
\begin{center}
\begin{tabular}{l}
S\textsubscript{sw}\textsubscript{ml}        (swallow middle left)\\
S\textsubscript{sw}\textsubscript{mr}        (swallow middle right)\\
\end{tabular}
\end{center}
\begin{itemize}
\item S\textsubscript{sw}\textsubscript{tc}        (swallow top center)
\end{itemize}
/ S\textsubscript{sw}\textsubscript{tl}        (swallow top left)
$\backslash$ S\textsubscript{sw}\textsubscript{tr}        (swallow top right)
\begin{itemize}
\item S\textsubscript{tdwall}        (wall)
\end{itemize}
\^{} S\textsubscript{teleportation}\textsubscript{trap}  (teleportation trap)
$\backslash$ S\textsubscript{throne}        (opulent throne)
\begin{itemize}
\item S\textsubscript{tlcorn}        (top left corner)
\end{itemize}
\begin{center}
\begin{tabular}{l}
S\textsubscript{tlwall}        (wall)\\
\end{tabular}
\end{center}
( S\textsubscript{tool}         (useful item (pick-axe, key, lamp\ldots{}))
\^{} S\textsubscript{trap}\textsubscript{door}      (trap door)
t S\textsubscript{trapper}       (trapper or lurker above)
\begin{itemize}
\item S\textsubscript{trcorn}        (top right corner)
\end{itemize}

T S\textsubscript{troll}        (troll)
\begin{center}
\begin{tabular}{l}
S\textsubscript{trwall}        (wall)\\
\end{tabular}
\end{center}
\begin{itemize}
\item S\textsubscript{tuwall}        (wall)
\end{itemize}
U S\textsubscript{umber}        (umber hulk)
u S\textsubscript{unicorn}       (unicorn or horse)
< S\textsubscript{upladder}       (ladder up)
< S\textsubscript{upstair}       (staircase up)
V S\textsubscript{vampire}       (vampire)
\begin{center}
\begin{tabular}{l}
S\textsubscript{vbeam}        (vertical beam [zap animation])\\
\end{tabular}
\end{center}

\begin{itemize}
\item S\textsubscript{vcdoor}        (closed door in vertical wall)
\end{itemize}
. S\textsubscript{venom}        (splash of venom)
\^{} S\textsubscript{vibrating}\textsubscript{square}   (vibrating square)
. S\textsubscript{vodbridge}      (vertical lowered drawbridge)
\begin{itemize}
\item S\textsubscript{vodoor}        (open door in vertical wall)
\end{itemize}
v S\textsubscript{vortex}        (vortex)
\begin{center}
\begin{tabular}{l}
S\textsubscript{vwall}        (vertical wall)\\
\end{tabular}
\end{center}
/ S\textsubscript{wand}         (wand)
\} S\textsubscript{water}        (water)
) S\textsubscript{weapon}        (weapon)
" S\textsubscript{web}         (web)
w S\textsubscript{worm}         (worm)
\textasciitilde{} S\textsubscript{worm}\textsubscript{tail}      (long worm tail)
W S\textsubscript{wraith}        (wraith)
x S\textsubscript{xan}         (xan or other extraordinary insect)
X S\textsubscript{xorn}         (xorn)
Y S\textsubscript{yeti}         (apelike creature)
Z S\textsubscript{zombie}        (zombie)
z S\textsubscript{zruty}        (zruty)



NetHack 3.6                   January 27, 2020





NetHack Guidebook                       83



S\textsubscript{pet}\textsubscript{override}     (any pet if ACCESSIBILITY=1 is set)
S\textsubscript{hero}\textsubscript{override}    (hero if ACCESSIBILITY=1 is set)

Notes:

\begin{itemize}
\item Several symbols in this table appear to be blank. They are the
space character, except for S\textsubscript{pet}\textsubscript{override} and S\textsubscript{hero}\textsubscript{override}
which don't have any default value and can only be used if en-
abled in the "sysconf" file.

\item S\textsubscript{rock} is misleadingly named; rocks and stones use S\textsubscript{gem}.
Statues and boulders are the rock being referred to, but since
version 3.6.0, statues are displayed as the monster they de-
pict.  So S\textsubscript{rock} is only used for boulders and not used at all
if overridden by the more specific S\textsubscript{boulder}.
\end{itemize}

9.15. Configuring NetHack for Play by the Blind

   NetHack can be set up to use only standard ASCII characters
for making maps of the dungeons. This makes the MS-DOS versions
of NetHack completely accessible to the blind who use speech
and/or Braille access technologies. Players will require a good
working knowledge of their screen-reader's review features, and
will have to know how to navigate horizontally and vertically
character by character. They will also find the search capabili-
ties of their screen-readers to be quite valuable. Be certain to
examine this Guidebook before playing so you have an idea what
the screen layout is like. You'll also need to be able to locate
the PC cursor. It is always where your character is located.
Merely searching for an @-sign will not always find your charac-
ter since there are other humanoids represented by the same sign.
Your screen-reader should also have a function which gives you
the row and column of your review cursor and the PC cursor.
These co-ordinates are often useful in giving players a better
sense of the overall location of items on the screen.

   NetHack can also be compiled with support for sending the
game messages to an external program, such as a text-to-speech
synthesizer. If the "\#version" extended command shows "external
program as a message handler", your NetHack has been compiled
with the capability. When compiling NetHack from source on Linux
and other POSIX systems, define MSGHANDLER to enable it. To use
the capability, set the environment variable NETHACK\textsubscript{MSGHANDLER}
to an executable, which will be executed with the game message as
the program's only parameter.

   While it is not difficult for experienced users to edit the
defaults.nh file to accomplish this, novices may find this task
somewhat daunting. Included within the "symbols" file of all of-
ficial distributions of NetHack is a symset called NHAccess. Se-
lecting that symset in your configuration file will cause the
game to run in a manner accessible to the blind. After you have
gained some experience with the game and with editing files, you
may want to alter settings via SYMBOLS= and ROGUESYMBOLS= in your


NetHack 3.6                   January 27, 2020





NetHack Guidebook                       84



configuration file to better suit your preferences. See the pre-
vious section for the special symbols S\textsubscript{pet}\textsubscript{override} to force a
consistent symbol for all pets and S\textsubscript{hero}\textsubscript{override} to force a
unique symbol for the player character if accessibility is en-
abled in the sysconf file.

   The most crucial settings to make the game more accessible
are:

symset:NHAccess
 Load a symbol set appropriate for use by blind players.

roguesymset:NHAccess
 Load a symbol set for the rogue level that is appropriate for
 use by blind players.

menustyle:traditional
 This will assist in the interface to speech synthesizers.

nomenu\textsubscript{overlay}
 Show menus on a cleared screen and aligned to the left edge.

number\textsubscript{pad}
 A lot of speech access programs use the number-pad to review
 the screen. If this is the case, disable the number\textsubscript{pad} option
 and use the traditional Rogue-like commands.

autodescribe
 Automatically describe the terrain under the cursor when tar-
 geting.

mention\textsubscript{walls}
 Give feedback messages when walking towards a wall or when
 travel command was interrupted.

whatis\textsubscript{coord}:compass
 When targeting with cursor, describe the cursor position with
 coordinates relative to your character.

whatis\textsubscript{filter}:area
 When targeting with cursor, filter possible locations so only
 those in the same area (eg. same room, or same corridor) are
 considered.

whatis\textsubscript{moveskip}
 When targeting with cursor and using fast-move, skip the same
 glyphs instead of moving 8 units at a time.

nostatus\textsubscript{updates}
 Prevent updates to the status lines at the bottom of the
 screen, if your screen-reader reads those lines. The same in-
 formation can be seen via the "\#attributes" command.




NetHack 3.6                   January 27, 2020





NetHack Guidebook                       85



9.16. Global Configuration for System Administrators

   If NetHack is compiled with the SYSCF option, a system ad-
ministrator should set up a global configuration; this is a file
in the same format as the traditional per-user configuration file
(see above). This file should be named sysconf and placed in the
same directory as the other NetHack support files.  The options
recognized in this file are listed below. Any option not set us-
es a compiled-in default (which may not be appropriate for your
system).

WIZARDS = A space-separated list of user names who are allowed
to play in debug mode (commonly referred to as wizard mode). A
value of a single asterisk (*) allows anyone to start a game in
debug mode.

SHELLERS = A list of users who are allowed to use the shell es-
cape command (!). The syntax is the same as WIZARDS.

EXPLORERS = A list of users who are allowed to use the explore
mode. The syntax is the same as WIZARDS.

MAXPLAYERS = Limit the maximum number of games that can be run-
ning at the same time.

SUPPORT = A string explaining how to get local support (no de-
fault value).

RECOVER = A string explaining how to recover a game on this
system (no default value).

SEDUCE = 0 or 1 to disable or enable, respectively, the SEDUCE
option. When disabled, incubi and succubi behave like nymphs.

CHECK\textsubscript{PLNAME} = Setting this to 1 will make the EXPLORERS, WIZ-
ARDS, and SHELLERS check for the player name instead of the us-
er's login name.

CHECK\textsubscript{SAVE}\textsubscript{UID} = 0 or 1 to disable or enable, respectively, the
UID (used identification number) checking for save files (to
verify that the user who is restoring is the same one who
saved).

The following options affect the score file:

PERSMAX = Maximum number of entries for one person.

ENTRYMAX = Maximum number of entries in the score file.

POINTSMIN = Minimum number of points to get an entry in the
score file.

PERS\textsubscript{IS}\textsubscript{UID} = 0 or 1 to use user names or numeric userids, re-
spectively, to identify unique people for the score file.


NetHack 3.6                   January 27, 2020





NetHack Guidebook                       86



MAX\textsubscript{STATUENAME}\textsubscript{RANK} = Maximum number of score file entries to
use for random statue names (default is 10).

ACCESSIBILITY = 0 or 1 to disable or enable, respectively, the
ability for players to set S\textsubscript{pet}\textsubscript{override} and S\textsubscript{hero}\textsubscript{override}
symbols in their configuration file.

PORTABLE\textsubscript{DEVICE}\textsubscript{PATHS} = 0 or 1 Windows OS only, the game will
look for all of its external files, and write to all of its
output files in one place rather than at the standard loca-
tions.

DUMPLOGFILE = A filename where the end-of-game dumplog is
saved. Not defining this will prevent dumplog from being cre-
ated. Only available if your game is compiled with DUMPLOG. Al-
lows the following placeholders:

\%\% - literal `\%'
\%v - version (eg. "3.6.3-0")
\%u - game UID
\%t - game start time, UNIX timestamp format
\%T - current time, UNIX timestamp format
\%d - game start time, YYYYMMDDhhmmss format
\%D - current time, YYYYMMDDhhmmss format
\%n - player name
\%N - first character of player name

\begin{enumerate}
\item Scoring

NetHack maintains a list of the top scores or scorers on
\end{enumerate}
your machine, depending on how it is set up. In the latter case,
each account on the machine can post only one non-winning score
on this list.  If you score higher than someone else on this
list, or better your previous score, you will be inserted in the
proper place under your current name. How many scores are kept
can also be set up when NetHack is compiled.

   Your score is chiefly based upon how much experience you
gained, how much loot you accumulated, how deep you explored, and
how the game ended. If you quit the game, you escape with all of
your gold intact.  If, however, you get killed in the Mazes of
Menace, the guild will only hear about 90\% of your gold when your
corpse is discovered (adventurers have been known to collect
finder's fees). So, consider whether you want to take one last
hit at that monster and possibly live, or quit and stop with
whatever you have. If you quit, you keep all your gold, but if
you swing and live, you might find more.

   If you just want to see what the current top players/games
list is, you can type nethack -s all on most versions.






NetHack 3.6                   January 27, 2020





NetHack Guidebook                       87



\begin{enumerate}
\item Explore mode

NetHack is an intricate and difficult game.  Novices might
\end{enumerate}
falter in fear, aware of their ignorance of the means to survive.
Well, fear not. Your dungeon comes equipped with an "explore" or
"discovery" mode that enables you to keep old save files and
cheat death, at the paltry cost of not getting on the high score
list.

   There are two ways of enabling explore mode. One is to
start the game with the -X command-line switch or with the play-
mode:explore option.  The other is to issue the "\#exploremode"
extended command while already playing the game. Starting a new
game in explore mode provides your character with a wand of wish-
ing in initial inventory; switching during play does not.  The
other benefits of explore mode are left for the trepid reader to
discover.

11.1. Debug mode

   Debug mode, also known as wizard mode, is undocumented aside
from this brief description and the various "debug mode only"
commands listed among the command descriptions. It is intended
for tracking down problems within the program rather than to pro-
vide god-like powers to your character, and players who attempt
debugging are expected to figure out how to use it themselves.
It is initiated by starting the game with the -D command-line
switch or with the playmode:debug option.

   For some systems, the player must be logged in under a par-
ticular user name to be allowed to use debug mode; for others,
the hero must be given a particular character name (but may be
any role; there's no connection between "wizard mode" and the
Wizard role). Attempting to start a game in debug mode when not
allowed or not available will result in falling back to explore
mode instead.

\begin{enumerate}
\item Credits

The original hack game was modeled on the Berkeley UNIX
\end{enumerate}
rogue game.  Large portions of this paper were shamelessly
cribbed from A Guide to the Dungeons of Doom, by Michael C. Toy
and Kenneth C. R. C. Arnold. Small portions were adapted from
Further Exploration of the Dungeons of Doom, by Ken Arromdee.

   NetHack is the product of literally dozens of people's work.
Main events in the course of the game development are described
below:

   Jay Fenlason wrote the original Hack, with help from Kenny
Woodland, Mike Thome and Jon Payne.

   Andries Brouwer did a major re-write, transforming Hack into
a very different game, and published (at least) three versions


NetHack 3.6                   January 27, 2020





NetHack Guidebook                       88



(1.0.1, 1.0.2, and 1.0.3) for UNIX machines to the Usenet.

   Don G. Kneller ported Hack 1.0.3 to Microsoft C and MS-DOS,
producing PC HACK 1.01e, added support for DEC Rainbow graphics
in version 1.03g, and went on to produce at least four more ver-
sions (3.0, 3.2, 3.51, and 3.6).

   R. Black ported PC HACK 3.51 to Lattice C and the Atari
520/1040ST, producing ST Hack 1.03.

   Mike Stephenson merged these various versions back together,
incorporating many of the added features, and produced NetHack
1.4. He then coordinated a cast of thousands in enhancing and
debugging NetHack 1.4 and released NetHack versions 2.2 and 2.3.

   Later, Mike coordinated a major rewrite of the game, heading
a team which included Ken Arromdee, Jean-Christophe Collet, Steve
Creps, Eric Hendrickson, Izchak Miller, John Rupley, Mike Threep-
oint, and Janet Walz, to produce NetHack 3.0c.

   NetHack 3.0 was ported to the Atari by Eric R. Smith, to
OS/2 by Timo Hakulinen, and to VMS by David Gentzel. The three
of them and Kevin Darcy later joined the main NetHack Development
Team to produce subsequent revisions of 3.0.

   Olaf Seibert ported NetHack 2.3 and 3.0 to the Amiga. Norm
Meluch, Stephen Spackman and Pierre Martineau designed overlay
code for PC NetHack 3.0. Johnny Lee ported NetHack 3.0 to the
Macintosh. Along with various other Dungeoneers, they continued
to enhance the PC, Macintosh, and Amiga ports through the later
revisions of 3.0.

   Headed by Mike Stephenson and coordinated by Izchak Miller
and Janet Walz, the NetHack Development Team which now included
Ken Arromdee, David Cohrs, Jean-Christophe Collet, Kevin Darcy,
Matt Day, Timo Hakulinen, Steve Linhart, Dean Luick, Pat Rankin,
Eric Raymond, and Eric Smith undertook a radical revision of 3.0.
They re-structured the game's design, and re-wrote major parts of
the code. They added multiple dungeons, a new display, special
individual character quests, a new endgame and many other new
features, and produced NetHack 3.1.

   Ken Lorber, Gregg Wonderly and Greg Olson, with help from
Richard Addison, Mike Passaretti, and Olaf Seibert, developed
NetHack 3.1 for the Amiga.

   Norm Meluch and Kevin Smolkowski, with help from Carl Sche-
lin, Stephen Spackman, Steve VanDevender, and Paul Winner, ported
NetHack 3.1 to the PC.

   Jon W\{tte and Hao-yang Wang, with help from Ross Brown, Mike
Engber, David Hairston, Michael Hamel, Jonathan Handler, Johnny
Lee, Tim Lennan, Rob Menke, and Andy Swanson, developed NetHack
3.1 for the Macintosh, porting it for MPW. Building on their


NetHack 3.6                   January 27, 2020





NetHack Guidebook                       89



development, Bart House added a Think C port.

   Timo Hakulinen ported NetHack 3.1 to OS/2. Eric Smith port-
ed NetHack 3.1 to the Atari. Pat Rankin, with help from Joshua
Delahunty, was responsible for the VMS version of NetHack 3.1.
Michael Allison ported NetHack 3.1 to Windows NT.

   Dean Luick, with help from David Cohrs, developed NetHack
3.1 for X11. Warwick Allison wrote a tiled version of NetHack
for the Atari; he later contributed the tiles to the NetHack De-
velopment Team and tile support was then added to other plat-
forms.

   The 3.2 NetHack Development Team, comprised of Michael Alli-
son, Ken Arromdee, David Cohrs, Jessie Collet, Steve Creps, Kevin
Darcy, Timo Hakulinen, Steve Linhart, Dean Luick, Pat Rankin, Er-
ic Smith, Mike Stephenson, Janet Walz, and Paul Winner, released
version 3.2 in April of 1996.

   Version 3.2 marked the tenth anniversary of the formation of
the development team. In a testament to their dedication to the
game, all thirteen members of the original NetHack Development
Team remained on the team at the start of work on that release.
During the interval between the release of 3.1.3 and 3.2, one of
the founding members of the NetHack Development Team, Dr. Izchak
Miller, was diagnosed with cancer and passed away. That release
of the game was dedicated to him by the development and porting
teams.

   During the lifespan of NetHack 3.1 and 3.2, several enthusi-
asts of the game added their own modifications to the game and
made these "variants" publicly available:

   Tom Proudfoot and Yuval Oren created NetHack++, which was
quickly renamed NetHack--. Working independently, Stephen White
wrote NetHack Plus. Tom Proudfoot later merged NetHack Plus and
his own NetHack-- to produce SLASH. Larry Stewart-Zerba and War-
wick Allison improved the spell casting system with the Wizard
Patch. Warwick Allison also ported NetHack to use the Qt inter-
face.

   Warren Cheung combined SLASH with the Wizard Patch to pro-
duce Slash'EM, and with the help of Kevin Hugo, added more fea-
tures.  Kevin later joined the NetHack Development Team and in-
corporated the best of these ideas in NetHack 3.3.

   The final update to 3.2 was the bug fix release 3.2.3, which
was released simultaneously with 3.3.0 in December 1999 just in
time for the Year 2000.

   The 3.3 NetHack Development Team, consisting of Michael Al-
lison, Ken Arromdee, David Cohrs, Jessie Collet, Steve Creps,
Kevin Darcy, Timo Hakulinen, Kevin Hugo, Steve Linhart, Ken Lor-
ber, Dean Luick, Pat Rankin, Eric Smith, Mike Stephenson, Janet


NetHack 3.6                   January 27, 2020





NetHack Guidebook                       90



Walz, and Paul Winner, released 3.3.0 in December 1999 and 3.3.1
in August of 2000.

   Version 3.3 offered many firsts. It was the first version to
separate race and profession. The Elf class was removed in pref-
erence to an elf race, and the races of dwarves, gnomes, and orcs
made their first appearance in the game alongside the familiar
human race. Monk and Ranger roles joined Archeologists, Barbar-
ians, Cavemen, Healers, Knights, Priests, Rogues,  Samurai,
Tourists, Valkyries and of course, Wizards. It was also the
first version to allow you to ride a steed, and was the first
version to have a publicly available web-site listing all the
bugs that had been discovered. Despite that constantly growing
bug list, 3.3 proved stable enough to last for more than a year
and a half.

   The 3.4 NetHack Development Team initially consisted of
Michael Allison, Ken Arromdee, David Cohrs, Jessie Collet, Kevin
Hugo, Ken Lorber, Dean Luick, Pat Rankin, Mike Stephenson, Janet
Walz, and Paul Winner, with Warwick Allison joining just before
the release of NetHack 3.4.0 in March 2002.

   As with version 3.3, various people contributed to the game
as a whole as well as supporting ports on the different platforms
that NetHack runs on:

Pat Rankin maintained 3.4 for VMS.

   Michael Allison maintained NetHack 3.4 for the MS-DOS plat-
form. Paul Winner and Yitzhak Sapir provided encouragement.

   Dean Luick, Mark Modrall, and Kevin Hugo maintained and en-
hanced the Macintosh port of 3.4.

   Michael Allison, David Cohrs, Alex Kompel, Dion Nicolaas,
and Yitzhak Sapir maintained and enhanced 3.4 for the Microsoft
Windows platform. Alex Kompel contributed a new graphical inter-
face for the Windows port. Alex Kompel also contributed a Win-
dows CE port for 3.4.1.

   Ron Van Iwaarden was the sole maintainer of NetHack for OS/2
the past several releases. Unfortunately Ron's last OS/2 machine
stopped working in early 2006. A great many thanks to Ron for
keeping NetHack alive on OS/2 all these years.

   Janne Salmijarvi and Teemu Suikki maintained and enhanced
the Amiga port of 3.4 after Janne Salmijarvi resurrected it for
3.3.1.

   Christian "Marvin" Bressler maintained 3.4 for the Atari af-
ter he resurrected it for 3.3.1.

   The release of NetHack 3.4.3 in December 2003 marked the be-
ginning of a long release hiatus. 3.4.3 proved to be a remarkably


NetHack 3.6                   January 27, 2020





NetHack Guidebook                       91



stable version that provided continued enjoyment by the community
for more than a decade. The NetHack Development Team slowly and
quietly continued to work on the game behind the scenes during
the tenure of 3.4.3. It was during that same period that several
new variants emerged within the NetHack community. Notably
sporkhack by Derek S. Ray, unnethack by Patric Mueller, nitrohack
and its successors originally by Daniel Thaler and then by Alex
Smith, and Dynahack by Tung Nguyen. Some of those variants con-
tinue to be developed, maintained, and enjoyed by the community
to this day.

   In September 2014, an interim snapshot of the code under de-
velopment was released publicly by other parties. Since that code
was a work-in-progress and had not gone through the process of
debugging it as a suitable release, it was decided that the ver-
sion numbers present on that code snapshot would be retired and
never used in an official NetHack release. An announcement was
posted on the NetHack Development Team's official nethack.org
website to that effect, stating that there would never be a
3.4.4, 3.5, or 3.5.0 official release version.

   In January 2015, preparation began for the release of
NetHack 3.6.

   At the beginning of development for what would eventually
get released as 3.6.0, the NetHack Development Team consisted of
Warwick Allison, Michael Allison, Ken Arromdee, David Cohrs,
Jessie Collet, Ken Lorber, Dean Luick, Pat Rankin, Mike Stephen-
son, Janet Walz, and Paul Winner. In early 2015, ahead of the
release of 3.6.0, new members Sean Hunt, Pasi Kallinen, and Derek
S. Ray joined the NetHack Development Team.

   Near the end of the development of 3.6.0, one of the signif-
icant inspirations for many of the humorous and fun features
found in the game, author Terry Pratchett, passed away. NetHack
3.6.0 introduced a tribute to him.

   3.6.0 was released in December 2015, and merged work done by
the development team since the release of 3.4.3 with some of the
beloved community patches. Many bugs were fixed and some code was
restructured.

   The NetHack Development Team, as well as Steve VanDevender
and Kevin Smolkowski, ensured that NetHack 3.6 continued to oper-
ate on various UNIX flavors and maintained the X11 interface.

   Ken Lorber, Haoyang Wang, Pat Rankin, and Dean Luick main-
tained the port of NetHack 3.6 for Mac OSX.

   Michael Allison, David Cohrs, Bart House, Pasi Kallinen,
Alex Kompel, Dion Nicolaas, Derek S. Ray and Yitzhak Sapir main-
tained the port of NetHack 3.6 for Microsoft Windows.




NetHack 3.6                   January 27, 2020





NetHack Guidebook                       92



   Pat Rankin attempted to keep the VMS port running for
NetHack 3.6, hindered by limited access. Kevin Smolkowski has up-
dated and tested it for the most recent version of OpenVMS (V8.4
as of this writing) on Alpha and Integrity (aka Itanium aka IA64)
but not VAX.

   Ray Chason resurrected the msdos port for 3.6 and contrib-
uted the necessary updates to the community at large.

   In late April 2018, several hundred bug fixes for 3.6.0 and
some new features were assembled and released as NetHack 3.6.1.
The NetHack Development Team at the time of release of 3.6.1 con-
sisted of Warwick Allison, Michael Allison, Ken Arromdee, David
Cohrs, Jessie Collet, Pasi Kallinen, Ken Lorber, Dean Luick,
Patric Mueller, Pat Rankin, Derek S. Ray, Alex Smith, Mike
Stephenson, Janet Walz, and Paul Winner.

   In early May 2019, another 320 bug fixes along with some en-
hancements and the adopted curses window port, were released as
3.6.2.

   Bart House, who had contributed to the game as a porting
team participant for decades, joined the NetHack Development Team
in late May 2019.

   NetHack 3.6.3 was released on December 5, 2019 containing
over 190 bug fixes to NetHack 3.6.2.

   NetHack 3.6.4 was released on December 18, 2019 containing a
security fix and a few bug fixes.

   NetHack 3.6.5 was released on January 27, 2020 containing
some security fixes and a small number of bug fixes.

   The official NetHack web site is maintained by Ken Lorber at
\url{https://www.nethack.org/}.

12.1. SPECIAL THANKS

   On behalf of the NetHack community, thank you very much once
again to M. Drew Streib and Pasi Kallinen for providing a public
NetHack server at nethack.alt.org. Thanks to Keith Simpson and
Andy Thomson for hardfought.org. Thanks to all those unnamed dun-
geoneers who invest their time and effort into annual NetHack
tournaments such as Junethack, The November NetHack Tournament
and in days past, devnull.net (gone for now, but not forgotten).

\begin{itemize}
\item - - - - - - - - -
\end{itemize}

   From time to time, some depraved individual out there in
netland sends a particularly intriguing modification to help out
with the game. The NetHack Development Team sometimes makes note
of the names of the worst of these miscreants in this, the list
of Dungeoneers:


NetHack 3.6                   January 27, 2020





NetHack Guidebook                       93



   Adam Aronow      J. Ali Harlow     Mikko Juola
   Alex Kompel       Janet Walz      Nathan Eady
   Alex Smith      Janne Salmijarvi    Norm Meluch
  Andreas Dorn    Jean-Christophe Collet   Olaf Seibert
   Andy Church      Jeff Bailey     Pasi Kallinen
  Andy Swanson      Jochen Erwied      Pat Rankin
  Andy Thomson      John Kallen     Patric Mueller
  Ari Huttunen      John Rupley      Paul Winner
   Bart House       John S. Bien    Pierre Martineau
 Benson I. Margulies     Johnny Lee      Ralf Brown
   Bill Dyer       Jon W\{tte       Ray Chason
 Boudewijn Waijers    Jonathan Handler   Richard Addison
   Bruce Cox      Joshua Delahunty    Richard Beigel
  Bruce Holloway     Karl Garrison    Richard P. Hughey
  Bruce Mewborne     Keizo Yamamoto     Rob Menke
  Carl Schelin      Keith Simpson     Robin Bandy
   Chris Russo       Ken Arnold     Robin Johnson
   David Cohrs      Ken Arromdee    Roderick Schertler
  David Damerell      Ken Lorber     Roland McGrath
  David Gentzel     Ken Washikita    Ron Van Iwaarden
  David Hairston      Kevin Darcy     Ronnen Miller
   Dean Luick       Kevin Hugo      Ross Brown
   Del Lamb       Kevin Sitze     Sascha Wostmann
  Derek S. Ray     Kevin Smolkowski    Scott Bigham
  Deron Meranda      Kevin Sweet     Scott R. Turner
  Dion Nicolaas      Lars Huttar      Sean Hunt
  Dylan O'Donnell     Leon Arnott     Stephen Spackman
   Eric Backus      M. Drew Streib   Stefan Thielscher
 Eric Hendrickson     Malcolm Ryan     Stephen White
  Eric R. Smith     Mark Gooderum     Steve Creps
  Eric S. Raymond     Mark Modrall     Steve Linhart
  Erik Andersen     Marvin Bressler   Steve VanDevender
 Fredrik Ljungdahl     Matthew Day      Teemu Suikki
 Frederick Roeber     Merlyn LeRoy      Tim Lennan
   Gil Neiger      Michael Allison    Timo Hakulinen
   Greg Laskin      Michael Feir      Tom Almy
   Greg Olson      Michael Hamel      Tom West
  Gregg Wonderly     Michael Sokolov    Warren Cheung
  Hao-yang Wang      Mike Engber     Warwick Allison
  Helge Hafting      Mike Gallop     Yitzhak Sapir
Irina Rempt-Drijfhout   Mike Passaretti
  Izchak Miller     Mike Stephenson

   Brand and product names are trademarks or registered trade-
marks of their respective holders.











NetHack 3.6                   January 27, 2020
































\section{Nethack man pages}
\label{sec:org1478834}

\subsection{Nethack(1) man page}
\label{sec:orgca2d871}

\begin{verbatim}
NETHACK(6)                             NETHACK(6)



NAME
    nethack - Exploring The Mazes of Menace

SYNOPSIS
    nethack [ -d directory ] [ -n ] [ -p profession ] [ -r race ] [ -[DX] ]
    [ -u playername ] [ -dec ] [ -ibm ] [ --version[:paste] ]

    nethack [ -d directory ] -s [ -v ] [ -p profession ] [ -r race ] [
    playernames ]

DESCRIPTION
    NetHack is a display oriented Dungeons & Dragons(tm) - like game. The
    standard tty display and command structure resemble rogue.

    Other, more graphical display options exist for most platforms.

    To get started you really only need to know two commands. The command
    ?  will give you a list of the available commands (as well as other
    information) and the command / will identify the things you see on the
    screen.

    To win the game (as opposed to merely playing to beat other people's
    high scores) you must locate the Amulet of Yendor which is somewhere
    below the 20th level of the dungeon and get it out. Few people achieve
    this; most never do. Those who have go down in history as heroes among
    heroes - and then they find ways of making the game even harder. See
    the Guidebook section on Conduct if this game has gotten too easy for
    you.

    When the game ends, whether by your dying, quitting, or escaping from
    the caves, NetHack will give you (a fragment of) the list of top scor-
    ers.  The scoring is based on many aspects of your behavior, but a
    rough estimate is obtained by taking the amount of gold you've found in
    the cave plus four times your (real) experience. Precious stones may
    be worth a lot of gold when brought to the exit.  There is a 10%
    penalty for getting yourself killed.

    The environment variable NETHACKOPTIONS can be used to initialize many
    run-time options. The ? command provides a description of these
    options and syntax. (The -dec and -ibm command line options are equiv-
    alent to the decgraphics and ibmgraphics run-time options described
    there, and are provided purely for convenience on systems supporting
    multiple types of terminals.)

    Because the option list can be very long (particularly when specifying
    graphics characters), options may also be included in a configuration
    file. The default is located in your home directory and named
    .nethackrc on Unix systems. On other systems, the default may be dif-
    ferent, usually NetHack.cnf.  On DOS or Windows, the  name  is
    defaults.nh, while on the Macintosh or BeOS, it is NetHack Defaults.
    The configuration file's location may be specified by setting NETHACK-
    OPTIONS to a string consisting of an @ character followed by the file-
    name.

    The -u playername option supplies the answer to the question "Who are
    you?".  It overrides any name from the options or configuration file,
    USER, LOGNAME, or getlogin(), which will otherwise be tried in order.
    If none of these provides a useful name, the player will be asked for
    one. Player names (in conjunction with uids) are used to identify save
    files, so you can have several saved games under different names. Con-
    versely, you must use the appropriate player name to restore a saved
    game.

    A playername suffix can be used to specify the profession, race, align-
    ment and/or gender of the character. The full syntax of the playername
    that includes a suffix is "name-ppp-rrr-aaa-ggg". "ppp" are at least
    the first three letters of the profession (this can also be specified
    using a separate -p profession option). "rrr" are at least the first
    three letters of the character's race (this can also be specified using
    a separate -r race option). "aaa" are at last the first three letters
    of the character's alignment, and "ggg" are at least the first three
    letters of the character's gender. Any of the parts of the suffix may
    be left out.

    -p profession can be used to determine the character profession, also
    known as the role. You can specify either the male or female name for
    the character role, or the first three characters of the role as an
    abbreviation. -p @ has been retained to explicitly request that a ran-
    dom role be chosen. It may need to be quoted with a backslash (\@) if
    @ is the "kill" character (see "stty") for the terminal, in order to
    prevent the current input line from being cleared.

    Likewise, -r race can be used to explicitly request that a race be cho-
    sen.

    Leaving out any of these characteristics will result in you being
    prompted during the game startup for the information.


    The -s option alone will print out the list of your scores on the cur-
    rent version.  An immediately following -v reports on all versions
    present in the score file. The -s may also be followed by arguments -p
    and -r to print the scores of particular roles and races only. It may
    also be followed by one or more player names to print the scores of the
    players mentioned, by 'all' to print out all scores, or by a number to
    print that many top scores.

    The -n option suppresses printing of any news from the game administra-
    tor.

    The -D or -X option will start the game in a special non-scoring dis-
    covery mode. -D will, if the player is the game administrator, start
    in debugging (wizard) mode instead.

    The -d option, which must be the first argument if it appears, supplies
    a directory which is to serve as the playground.  It overrides the
    value from NETHACKDIR, HACKDIR, or the directory specified by the game
    administrator during compilation (usually /usr/games/lib/nethackdir).
    This option is usually only useful to the game administrator. The
    playground must contain several auxiliary files such as help files, the
    list of top scorers, and a subdirectory save where games are saved.

AUTHORS
    Jay Fenlason (+ Kenny Woodland, Mike Thome and Jon Payne) wrote the
    original hack, very much like rogue (but full of bugs).

    Andries Brouwer continuously deformed their sources into an entirely
    different game.

    Mike Stephenson has continued the perversion of sources, adding various
    warped character classes and sadistic traps with the help of many
    strange people who reside in that place between the worlds, the Usenet
    Zone. A number of these miscreants are immortalized in the historical
    roll of dishonor and various other places.

    The resulting mess is now called NetHack, to denote its development by
    the Usenet. Andries Brouwer has made this request for the distinction,
    as he may eventually release a new version of his own.

FILES
    Run-time configuration options were discussed above and use a platform
    specific name for a file in a platform specific location.  For Unix,
    the name is '.nethackrc' in the user's home directory.

    All  other  files  are  in  the  playground directory, normally
    /usr/games/lib/nethackdir. If DLB was defined during the compile, the
    data files and special levels will be inside a larger file, normally
    nhdat, instead of being separate files.

    nethack           The program itself.
    data, oracles, rumors    Data files used by NetHack.
    quest.dat, bogusmon     More data files.
    engrave, epitaph, tribute  Still more data files.
    symbols           Data file holding sets of specifications
                  for how to display monsters, objects, and
                  map features.
    options           Data file containing a description of the
                  build-time option settings.
    help, hh          Help data files.
    cmdhelp, opthelp, wizhelp  More help data files.
    *.lev            Predefined special levels.
    dungeon           Control file for special levels.
    history           A short history of NetHack.
    license           Rules governing redistribution.
    record           The list of top scorers.
    logfile           An extended list of games played
                  (optional).
    xlogfile          A more detailed version of 'logfile'
                  (also optional).
    paniclog          Record of exceptional conditions
                  discovered during program execution.
    xlock.nn          Description of dungeon level 'nn' of
                  active game 'x' if there's a limit on the
                  number of simultaneously active games.
    UUcccccc.nn         Alternate form for dungeon level 'nn'
                  of active game by user 'UU' playing
                  character named 'cccccc' when there's no
                  limit on number of active games.
    perm            Lock file for xlock.0 or UUcccccc.0.
    bonesDD.nn         Descriptions of the ghost and belongings
                  of a deceased adventurer who met his
                  or her demise on level 'nn'.

    save/            A subdirectory containing saved games.

    sysconf           System-wide options. Required if
                  program is built with 'SYSCF' option
                  enabled, ignored if not.

    The location of 'sysconf' is specified at build time and can't be
    changed except by updating source file "config.h" and rebuilding the
    program.

    In a perfect world, 'paniclog' would remain empty.

ENVIRONMENT
    USER or LOGNAME     Your login name.
    HOME          Your home directory.
    SHELL          Your shell.
    TERM          The type of your terminal.
    HACKPAGER or PAGER   Replacement for default pager.
    MAIL          Mailbox file.
    MAILREADER       Replacement for default reader
                (probably /bin/mail or /usr/ucb/mail).
    NETHACKDIR or HACKDIR  Playground.
    NETHACKOPTIONS     String predefining several NetHack
                options.

    If the same option is specified in both NETHACKOPTIONS and .nethackrc,
    the value assigned in NETHACKOPTIONS takes precedence.

    SHOPTYPE and SPLEVTYPE can be used in debugging (wizard) mode.
    DEBUGFILES can be used if the program was built with 'DEBUG' enabled.

SEE ALSO
    dgn_comp(6), lev_comp(6), recover(6)

BUGS
    Probably infinite.



    Dungeons & Dragons is a Trademark of Wizards of the Coast, Inc.



                7 December 2015           NETHACK(6)
\end{verbatim}

\subsection{Nethack recovery tool}
\label{sec:orge187a43}

\begin{verbatim}



RECOVER(6)          1993           RECOVER(6)



NAME
   recover - recover a NetHack game interrupted by disaster

SYNOPSIS
   recover [ -d directory ] base1 base2 ...

DESCRIPTION
   Occasionally, a NetHack game will be interrupted by disaster
   when the game or the system crashes. Prior to NetHack v3.1,
   these games were lost because various information like the
   player's inventory was kept only in memory. Now, all per-
   tinent information can be written out to disk, so such games
   can be recovered at the point of the last level change.

   The base options tell recover which files to process.  Each
   base option specifies recovery of a separate game.

   The -d option, which must be the first argument if it
   appears, supplies a directory which is the NetHack play-
   ground. It overrides the value from NETHACKDIR, HACKDIR, or
   the directory specified by the game administrator during
   compilation (usually /usr/games/lib/nethackdir).

   For recovery to be possible, nethack must have been compiled
   with the INSURANCE option, and the run-time option check-
   point must also have been on. NetHack normally writes out
   files for levels as the player leaves them, so they will be
   ready for return visits. When checkpointing, NetHack also
   writes out the level entered and the current game state on
   every level change. This naturally slows level changes down
   somewhat.

   The level file names are of the form base.nn, where nn is an
   internal bookkeeping number for the level. The file base.0
   is used for game identity, locking, and, when checkpointing,
   for the game state. Various OSes use different strategies
   for constructing the base name.  Microcomputers use the
   character name, possibly truncated and modified to be a
   legal filename on that system. Multi-user systems use the
   (modified) character name prefixed by a user number to avoid
   conflicts, or "xlock" if the number of concurrent players is
   being limited.  It may be necessary to look in the play-
   ground to find the correct base name of the interrupted
   game.  recover will transform these level files into a save
   file of the same name as nethack would have used.

   Since recover must be able to read and delete files from the
   playground and create files in the save directory, it has
   interesting interactions with game security.  Giving ordi-
   nary players access to recover through setuid or setgid is
   tantamount to leaving the playground world-writable, with
   respect to both cheating and messing up other players. For
   a single-user system, this of course does not change any-
   thing, so some of the microcomputer ports install recover by
   default.

   For a multi-user system, the game administrator may want to
   arrange for all .0 files in the playground to be fed to
   recover when the host machine boots, and handle game crashes
   individually.  If the user population is sufficiently
   trustworthy, recover can be installed with the same permis-
   sions the nethack executable has. In either case, recover
   is easily compiled from the distribution utility directory.

NOTES
   Like nethack itself, recover will overwrite existing save-
   files of the same name. Savefiles created by recover are
   uncompressed; they may be compressed afterwards if desired,
   but even a compression-using nethack will find them in the
   uncompressed form.

SEE ALSO
   nethack(6)

BUGS
   recover makes no attempt to find out if a base name speci-
   fies a game in progress. If multiple machines share a play-
   ground, this would be impossible to determine.

   recover should be taught to use the nethack playground lock-
   ing mechanism to avoid conflicts.

January         Last change: 9             2

\end{verbatim}

\subsection{Nethack data librarian man page}
\label{sec:org46894cd}

\begin{verbatim}

DLB(6)            1993             DLB(6)


NAME
   dlb - NetHack data librarian

SYNOPSIS
   dlb { xct } [ vfIC ] arguments... [ files... ]

DESCRIPTION
   Dlb is a file archiving tool in the spirit (and tradition)
   of tar for NetHack version 3.1 and higher. It is used to
   maintain the archive files from which NetHack reads special
   level files and other read-only information. Note that like
   tar the command and option specifiers are specified as a
   continuous string and are followed by any arguments required
   in the same order as the option specifiers.

   This facility is optional and may be excluded during NetHack
   configuration.

COMMANDS
   The x command causes dlb to extract the contents of the
   archive into the current directory.

   The c command causes dlb to create a new archive from files
   in the current directory.

   The t command lists the files in the archive.

OPTIONS AND ARGUMENTS
   v      verbose output

   f archive  specify the archive. Default if f not specified
   is LIBFILE (usually the nhdat file in the playground).

   I lfile   specify the file containing the list of files to
   put in to or extract from the archive if no files are listed
   on the command line. Default for archive creation if no
   files are listed is LIBLISTFILE.

   C dir    change directory. Changes directory before try-
   ing to read any files (including the archive and the lfile).

EXAMPLES
   Create the default archive from the default file list:
         dlb c

   List the contents of the archive 'foo':
         dlb tf foo

AUTHOR
   Kenneth Lorber

SEE ALSO
   nethack(6), tar(1)

BUGS
   Not a good tar emulation; - does not mean stdin or stdout.
   Should include an optional compression facility. Not all
   read-only files for NetHack can be read out of an archive;
   examining the source is the only way to know which files can
   be.

Oct           Last change: 28            2

\end{verbatim}

\subsection{Nethack dungeon compiler man page}
\label{sec:org06dae48}

\begin{verbatim}

DGN_COMP(6)          1995          DGN_COMP(6)

NAME
   dgn_comp - NetHack dungeon compiler

SYNOPSIS
   dgn_comp [ file ]

   If no arguments are given, it reads standard input.

DESCRIPTION
   Dgn_comp is a dungeon compiler for NetHack version 3.2 and
   higher. It takes a description file as an argument and pro-
   duces a dungeon "script" that is to be loaded by NetHack at
   runtime.

   The purpose of this tool is to provide NetHack administra-
   tors and implementors with a convenient way to create a cus-
   tom dungeon for the game, without having to recompile the
   entire world.

GRAMMAR
   DUNGEON: name bonesmarker ( base , rand ) [ %age ]

   where name is the dungeon name, bonesmarker is a letter for
   marking bones files, ( base , rand ) is the number of lev-
   els, and %age is its percentage chance of being generated
   (if absent, 100% chance).

   DESCRIPTION: tag

   where tag is currently one of HELLISH, MAZELIKE, or ROGUE-
   LIKE.

   ALIGNMENT | LEVALIGN: [ lawful | neutral | chaotic |
   unaligned ]

   gives the alignment of the dungeon/level (default  is
   unaligned).

   ENTRY: level

   the dungeon entry point. The dungeon connection attaches at
   this level of the given dungeon. If the value of level is
   negative, the entry level is calculated from the bottom of
   the dungeon, with -1 being the last level. If this line is
   not present in a dungeon description, the entry level
   defaults to 1.

   PROTOFILE: name

   the prototypical name for dungeon level files in this
   dungeon.  For example, the PROTOFILE name for the dungeon
   Vlad's Tower is tower.

   LEVEL: name bonesmarker @ ( base , rand ) [ %age ]

   where name is the level name, bonesmarker is a letter for
   marking bones files, ( base , rand ) is the location and
   %age is the generation percentage, as above.

   RNDLEVEL: name bonesmarker @ ( base , rand ) [ %age ]
   rndlevs

   where name is the level name, bonesmarker is a letter for
   marking bones files, ( base , rand ) is the location, %age
   is the generation percentage, as above, and rndlevs is the
   number of similar levels available to choose from.

   CHAINLEVEL: name bonesmarker prev_name + ( base , rand ) [
   %age ]

   where name is the level name, bonesmarker is a letter for
   marking bones files, prev_name is the name of a level
   defined previously, ( base , rand ) is the offset from the
   level being chained from, and %age is the generation percen-
   tage.

   RNDCHAINLEVEL: name bonesmarker prev_name + ( base , rand )
   [ %age ] rndlevs

   where name is the level name, bonesmarker is a letter for
   marking bones files, prev_name is the name of a level
   defined previously, ( base , rand ) is the offset from the
   level being chained from, %age is the generation percentage,
   and rndlevs is the number of similar levels available to
   choose from.

   LEVELDESC: type

   where type is the level type, (see DESCRIPTION, above). The
   type is used to override any pre-set value used to describe
   the entire dungeon, for this level only.

   BRANCH: name @ ( base , rand ) [ stair | no_up | no_down |
   portal ] [ up | down ]

   where name is the name of the dungeon to branch to, and (
   base , rand ) is the location of the branch. The last two
   optional arguments are the branch type and branch direction.
   The type of a branch can be a two-way stair connection, a
   one-way stair connection, or a magic portal.  A one-way
   stair is described by the types no_up and no_down which
   specify which stair direction is missing.  The default
   branch type is stair.  The direction for a stair can be
   either up or down; direction is not applicable to portals.
   The default direction is down.

   CHAINBRANCH: name prev_name + ( base , rand ) [ stair |
   no_up | no_down | portal ] [ up | down ]

   where name is the name of the dungeon to branch to,
   prev_name is the name of a previously defined level and (
   base , rand ) is the offset from the level being chained
   from.  The optional branch type and direction are the same
   as described above.

GENERIC RULES
   Each dungeon must have a unique bonesmarker , and each spe-
   cial level must have a bonesmarker unique within its dungeon
   (letters may be reused in different dungeons).  If the
   bonesmarker has the special value "none", no bones files
   will be created for that level or dungeon.

   The value base may be in the range of 1 to MAXLEVEL (as
   defined in global.h ).

   The value rand may be in the range of -1 to MAXLEVEL.

   If rand is -1 it will be replaced with  the  value
   (num_dunlevs(dungeon) - base) during the load process (ie.
   from here to the end of the dungeon).

   If rand is 0 the level is located absolutely at base.

   Branches don't have a probability.  Dungeons do.  If a
   dungeon fails to be generated during load, all its levels
   and branches are skipped.

   No level or branch may be chained from a level with a per-
   centage generation probability.  This is to prevent non-
   resolution during the load. In addition, no branch may be
   made from a dungeon with a percentage generation probability
   for the same reason.

   As a general rule using the dungeon compiler:

   If a dungeon has a protofile name associated with it (eg.
   tower) that file will be used.

   If a special level is present, it will override the above
   rule and the appropriate file will be loaded.

   If neither of the above are present, the standard generator
   will take over and make a "normal" level.

   A level alignment, if present, will override the alignment
   of the dungeon that it exists within.

EXAMPLE
   Here is the current syntax of the dungeon compiler's
   "language":


   #
   #    The dungeon description file for the "standard" original
   #    3.0 NetHack.
   #
   DUNGEON:    "The Dungeons of Doom" "D" (25, 5)
   LEVEL:     "rogue" "none" @ (15, 4)
   LEVEL:     "oracle" "none" @ (5, 7)
   LEVEL:     "bigroom" "B" @ (12, 3) 15
   LEVEL:     "medusa" "none" @ (20, 5)
   CHAINLEVEL:   "castle" "medusa" + (1, 4)
   CHAINBRANCH:  "Hell" "castle" + (0, 0) no_down
   BRANCH:     "The Astral Plane" @ (1, 0) no_down up

   DUNGEON:    "Hell" "H" (25, 5)
   DESCRIPTION:  mazelike
   DESCRIPTION:  hellish
   BRANCH:     "Vlad's Tower" @ (13, 5) up
   LEVEL:     "wizard" "none" @ (15, 10)
   LEVEL:     "fakewiz" "A" @ (5, 5)
   LEVEL:     "fakewiz" "B" @ (10, 5)
   LEVEL:     "fakewiz" "C" @ (15, 5)
   LEVEL:     "fakewiz" "D" @ (20, 5)
   LEVEL:     "fakewiz" "E" @ (25, 5)

   DUNGEON:    "Vlad's Tower" "T" (3, 0)
   PROTOFILE:   "tower"
   DESCRIPTION:  mazelike
   ENTRY:     -1

   DUNGEON:    "The Astral Plane" "A" (1, 0)
   DESCRIPTION:  mazelike
   PROTOFILE:   "endgame"

   NOTES:
   Lines beginning with '#' are considered comments.
   A special level must be explicitly aligned.  The alignment
   of the dungeon it is in only applies to non-special levels
   within that dungeon.

AUTHOR
   M. Stephenson (from the level compiler by Jean-Christophe
   Collet).

SEE ALSO
   lev_comp(6), nethack(6)

BUGS
   Probably infinite.

Dec           Last change: 12            5

\end{verbatim}

\subsection{Nethack special levels compiler man page}
\label{sec:org095b03c}

\begin{verbatim}

LEV_COMP(6)          1996          LEV_COMP(6)


NAME
   lev_comp - NetHack special levels compiler

SYNOPSIS
   lev_comp [ -w ] [ files ]

   If no arguments are given, it reads standard input.

DESCRIPTION
   Lev_comp is a special level compiler for NetHack version 3.2
   and higher.  It takes description files as arguments and
   produces level files that can be loaded by NetHack at run-
   time.

   The purpose of this tool is to provide NetHack administra-
   tors and implementors with a convenient way for adding spe-
   cial levels to the game, or modifying existing ones, without
   having to recompile the entire world.

   The -w option causes lev_comp to perform extra checks on the
   level and display extra warnings, however these warnings are
   sometimes superfluous, so they are not normally displayed.


GRAMMAR
   file      : /* nothing */
           | levels
           ;

   levels     : level
           | level levels
           ;

   level      : maze_level
           | room_level
           ;

   maze_level   : maze_def flags lev_init messages regions
           ;

   room_level   : level_def flags lev_init messages rreg_init rooms corridors_def
           ;

   level_def    : LEVEL_ID ':' string
           ;

   lev_init    : /* nothing */
           | LEV_INIT_ID ':' CHAR ',' CHAR ',' BOOLEAN ',' BOOLEAN ',' light_state ',' walled
           ;

   walled     : BOOLEAN
           | RANDOM_TYPE
           ;

   flags      : /* nothing */
           | FLAGS_ID ':' flag_list
           ;

   flag_list    : FLAG_TYPE ',' flag_list
           | FLAG_TYPE
           ;

   messages    : /* nothing */
           | message messages
           ;

   message     : MESSAGE_ID ':' STRING
           ;

   rreg_init    : /* nothing */
           | rreg_init init_rreg
           ;

   init_rreg    : RANDOM_OBJECTS_ID ':' object_list
           | RANDOM_MONSTERS_ID ':' monster_list
           ;

   rooms      : /* Nothing - dummy room for use with INIT_MAP */
           | roomlist
           ;

   roomlist    : aroom
           | aroom roomlist
           ;

   corridors_def  : random_corridors
           | corridors
           ;

   random_corridors: RAND_CORRIDOR_ID
           ;

   corridors    : /* nothing */
           | corridors corridor
           ;

   corridor    : CORRIDOR_ID ':' corr_spec ',' corr_spec
           | CORRIDOR_ID ':' corr_spec ',' INTEGER
           ;

   corr_spec    : '(' INTEGER ',' DIRECTION ',' door_pos ')'
           ;

   aroom      : room_def room_details
           | subroom_def room_details
           ;

   subroom_def   : SUBROOM_ID ':' room_type ',' light_state ',' subroom_pos ',' room_size ',' string roomfill
           ;

   room_def    : ROOM_ID ':' room_type ',' light_state ',' room_pos ',' room_align ',' room_size roomfill
           ;

   roomfill    : /* nothing */
           | ',' BOOLEAN
           ;

   room_pos    : '(' INTEGER ',' INTEGER ')'
           | RANDOM_TYPE
           ;

   subroom_pos   : '(' INTEGER ',' INTEGER ')'
           | RANDOM_TYPE
           ;

   room_align   : '(' h_justif ',' v_justif ')'
           | RANDOM_TYPE
           ;

   room_size    : '(' INTEGER ',' INTEGER ')'
           | RANDOM_TYPE
           ;

   room_details  : /* nothing */
           | room_details room_detail
           ;

   room_detail   : room_name
           | room_chance
           | room_door
           | monster_detail
           | object_detail
           | trap_detail
           | altar_detail
           | fountain_detail
           | sink_detail
           | pool_detail
           | gold_detail
           | engraving_detail
           | stair_detail
           ;

   room_name    : NAME_ID ':' string
           ;

   room_chance   : CHANCE_ID ':' INTEGER
           ;

   room_door    : DOOR_ID ':' secret ',' door_state ',' door_wall ',' door_pos
           ;

   secret     : BOOLEAN
           | RANDOM_TYPE
           ;

   door_wall    : DIRECTION
           | RANDOM_TYPE
           ;

   door_pos    : INTEGER
           | RANDOM_TYPE
           ;

   maze_def    : MAZE_ID ':' string ',' filling
           ;

   filling     : CHAR
           | RANDOM_TYPE
           ;

   regions     : aregion
           | aregion regions
           ;

   aregion     : map_definition reg_init map_details
           ;

   map_definition : NOMAP_ID
           | map_geometry MAP_ID
           ;

   map_geometry  : GEOMETRY_ID ':' h_justif ',' v_justif
           ;

   h_justif    : LEFT_OR_RIGHT
           | CENTER
           ;

   v_justif    : TOP_OR_BOT
           | CENTER
           ;

   reg_init    : /* nothing */
           | reg_init init_reg
           ;

   init_reg    : RANDOM_OBJECTS_ID ':' object_list
           | RANDOM_PLACES_ID ':' place_list
           | RANDOM_MONSTERS_ID ':' monster_list
           ;

   object_list   : object
           | object ',' object_list
           ;

   monster_list  : monster
           | monster ',' monster_list
           ;

   place_list   : place
           | place ',' place_list
           ;

   map_details   : /* nothing */
           | map_details map_detail
           ;

   map_detail   : monster_detail
           | object_detail
           | door_detail
           | trap_detail
           | drawbridge_detail
           | region_detail
           | stair_region
           | portal_region
           | teleprt_region
           | branch_region
           | altar_detail
           | fountain_detail
           | mazewalk_detail
           | wallify_detail
           | ladder_detail
           | stair_detail
           | gold_detail
           | engraving_detail
           | diggable_detail
           | passwall_detail
           ;

   monster_detail : MONSTER_ID chance ':' monster_c ',' m_name ',' coordinate
           monster_infos
           ;

   monster_infos  : /* nothing */
           | monster_infos monster_info
           ;

   monster_info  : ',' string
           | ',' MON_ATTITUDE
           | ',' MON_ALERTNESS
           | ',' alignment
           | ',' MON_APPEARANCE string
           ;

   object_detail  : OBJECT_ID object_desc
           | COBJECT_ID object_desc
           ;

   object_desc   : chance ':' object_c ',' o_name ',' object_where object_infos
           ;

   object_where  : coordinate
           | CONTAINED
           ;

   object_infos  : /* nothing */
           | ',' curse_state ',' monster_id ',' enchantment optional_name
           | ',' curse_state ',' enchantment optional_name
           | ',' monster_id ',' enchantment optional_name
           ;

   curse_state   : RANDOM_TYPE
           | CURSE_TYPE
           ;

   monster_id   : STRING
           ;

   enchantment   : RANDOM_TYPE
           | INTEGER
           ;

   optional_name  : /* nothing */
           | ',' NONE
           | ',' STRING
           ;

   door_detail   : DOOR_ID ':' door_state ',' coordinate
           ;

   trap_detail   : TRAP_ID chance ':' trap_name ',' coordinate
           ;

   drawbridge_detail: DRAWBRIDGE_ID ':' coordinate ',' DIRECTION ',' door_state
           ;

   mazewalk_detail : MAZEWALK_ID ':' coordinate ',' DIRECTION
           ;

   wallify_detail : WALLIFY_ID
           ;

   ladder_detail  : LADDER_ID ':' coordinate ',' UP_OR_DOWN
           ;

   stair_detail  : STAIR_ID ':' coordinate ',' UP_OR_DOWN
           ;

   stair_region  : STAIR_ID ':' lev_region ',' lev_region ',' UP_OR_DOWN
           ;

   portal_region  : PORTAL_ID ':' lev_region ',' lev_region ',' string
           ;

   teleprt_region : TELEPRT_ID ':' lev_region ',' lev_region teleprt_detail
           ;

   branch_region  : BRANCH_ID ':' lev_region ',' lev_region
           ;

   teleprt_detail : /* empty */
           | ',' UP_OR_DOWN
           ;

   lev_region   : region
           | LEV '(' INTEGER ',' INTEGER ',' INTEGER ',' INTEGER ')'
           ;

   fountain_detail : FOUNTAIN_ID ':' coordinate
           ;

   sink_detail : SINK_ID ':' coordinate
           ;

   pool_detail : POOL_ID ':' coordinate
           ;

   diggable_detail : NON_DIGGABLE_ID ':' region
           ;

   passwall_detail : NON_PASSWALL_ID ':' region
           ;

   region_detail  : REGION_ID ':' region ',' light_state ',' room_type prefilled
           ;

   altar_detail  : ALTAR_ID ':' coordinate ',' alignment ',' altar_type
           ;

   gold_detail   : GOLD_ID ':' amount ',' coordinate
           ;

   engraving_detail: ENGRAVING_ID ':' coordinate ',' engraving_type ',' string
           ;

   monster_c    : monster
           | RANDOM_TYPE
           | m_register
           ;

   object_c    : object
           | RANDOM_TYPE
           | o_register
           ;

   m_name     : string
           | RANDOM_TYPE
           ;

   o_name     : string
           | RANDOM_TYPE
           ;

   trap_name    : string
           | RANDOM_TYPE
           ;

   room_type    : string
           | RANDOM_TYPE
           ;

   prefilled    : /* empty */
           | ',' FILLING
           | ',' FILLING ',' BOOLEAN
           ;

   coordinate   : coord
           | p_register
           | RANDOM_TYPE
           ;

   door_state   : DOOR_STATE
           | RANDOM_TYPE
           ;

   light_state   : LIGHT_STATE
           | RANDOM_TYPE
           ;

   alignment    : ALIGNMENT
           | a_register
           | RANDOM_TYPE
           ;

   altar_type   : ALTAR_TYPE
           | RANDOM_TYPE
           ;

   p_register   : P_REGISTER '[' INTEGER ']'
           ;

   o_register   : O_REGISTER '[' INTEGER ']'
           ;

   m_register   : M_REGISTER '[' INTEGER ']'
           ;

   a_register   : A_REGISTER '[' INTEGER ']'
           ;

   place      : coord
           ;

   monster     : CHAR
           ;

   object     : CHAR
           ;

   string     : STRING
           ;

   amount     : INTEGER
           | RANDOM_TYPE
           ;

   chance     : /* empty */
           | PERCENT
           ;

   engraving_type : ENGRAVING_TYPE
           | RANDOM_TYPE
           ;

   coord      : '(' INTEGER ',' INTEGER ')'
           ;

   region     : '(' INTEGER ',' INTEGER ',' INTEGER ',' INTEGER ')'
           ;

   NOTE:
   Lines beginning with '#' are considered comments.

   The contents of a "MAP" description of a maze is a rectangle
   showing the exact level map that should be used for the
   given part of a maze. Each character in the map corresponds
   to a location on the screen. Different location types are
   denoted using different ASCII characters.  The following
   characters are recognized. To give an idea of how these are
   used, see the EXAMPLE, below. The maximum size of a map is
   normally 76 columns by 21 rows.

   '-'   horizontal wall
   '|'   vertical wall
   '+'   a doorway (state is specified in a DOOR declaration)
   'A'   open air
   'B'   boundary room location (for bounding unwalled irregular regions)
   'C'   cloudy air
   'I'   ice
   'S'   a secret door
   'H'   a secret corridor
   '{'   a fountain
   '\'   a throne
   'K'   a sink (if SINKS is defined, else a room location)
   '}'   a part of a moat or other deep water
   'P'   a pool
   'L'   lava
   'W'   water (yes, different from a pool)
   'T'   a tree
   'F'   iron bars
   '#'   a corridor
   '.'   a normal room location (unlit unless lit in a REGION declaration)
   ' '   stone

EXAMPLE
   Here is an example of a description file (a very simple
   one):

   MAZE : "fortress", random
   GEOMETRY : center , center
   MAP
   }}}}}}}}}
   }}}|-|}}}
   }}|-.-|}}
   }|-...-|}
   }|.....|}
   }|-...-|}
   }}|-.-|}}
   }}}|-|}}}
   }}}}}}}}}
   ENDMAP
   MONSTER: '@', "Wizard of Yendor", (4,4)
   OBJECT: '"', "Amulet of Yendor", (4,4)
   # a hell hound flanking the Wiz on a random side
   RANDOM_PLACES: (4,3), (4,5), (3,4), (5,4)
   MONSTER: 'd', "hell hound", place[0]
   # a chest on another random side
   OBJECT: '(', "chest", place[1]
   # a sack on a random side, with a diamond and maybe a ruby in it
   CONTAINER: '(', "sack", place[2]
   OBJECT: '*', "diamond", contained
   OBJECT[50%]: '*', "ruby", contained
   # a random dragon somewhere
   MONSTER: 'D', random, random
   # 3 out of 4 chance for a random trap in the EAST end
   TRAP[75%]: random, (6,4)
   # an electric eel below the SOUTH end
   MONSTER: ';', "electric eel", (4,8)
   # make the walls non-diggable
   NON_DIGGABLE: (0,0,8,8)
   TELEPORT_REGION: levregion(0,0,79,20), (0,0,8,8)

   This example will produce a file named "fortress" that can
   be integrated into one of the numerous mazes of the game.

   Note especially the final, TELEPORT_REGION specification.
   This  says that level teleports or other non-stairway
   arrivals on this level can land anywhere on the level except
   the area of the map. This shows the use of the ``levre-
   gion'' prefix allowed in certain region specifications.
   Normally, regions apply only to the most recent MAP specifi-
   cation, but when prefixed with ``levregion'', one can refer
   to any area of the level, regardless of the placement of the
   current MAP in the level.

AUTHOR
   Jean-Christophe Collet, David Cohrs.

SEE ALSO
   dgn_comp(6), nethack(6)

BUGS
   Probably infinite. Most importantly, still needs additional
   bounds checking.

May           Last change: 16            11

\end{verbatim}

\section{Index}
\label{sec:org8b52945}
\end{document}
